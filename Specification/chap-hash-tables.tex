\chapter{Hash tables}

\section{Classes}

Every class in this section has \texttt{built-in-class} as its
metaclass.

\Defclass {hash-table}

This class is the base class of all hash tables.  It is a subclass of
the class \texttt{t}.

\Defclass {eq-hash-table-mixin}

This mixin class is a superclass of every hash tables class that uses
\texttt{eq} as its test function.  It is a subclass of the class
\texttt{t}.

\Defclass {eql-hash-table-mixin}

This mixin class is a superclass of every hash tables class that uses
\texttt{eql} as its test function.  It is a subclass of the class
\texttt{t}.

\Defclass {equal-hash-table-mixin}

This mixin class is a superclass of every hash tables class that uses
\texttt{equal} as its test function.  It is a subclass of the class
\texttt{t}.

\Defclass {equalp-hash-table-mixin}

This mixin class is a superclass of every hash tables class that uses
\texttt{equalp} as its test function.  It is a subclass of the class
\texttt{t}.

\Defclass {standard-hash-table}

This class is a subclass of the class \texttt{hash-table}.

\Definitarg {:contents}

This initialization argument is accepted by all instances of
\texttt{standard-hash-table}

\Defgeneric {contents} {standard-hash-table}

Given an instance of the class \texttt{standard-hash-table}, this
generic function returns the value that was supplied as the
\texttt{:contents} initialization argument when the instance was
created.
