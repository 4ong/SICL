\chapter{Hash tables}

\section{Package}

The package for all symbols in this chapter is \texttt{sicl-hash-table}.

\section{Protocol}

Every class in this section has \texttt{built-in-class} as its
metaclass.

\Defclass {hash-table}

This class is the base class of all hash tables.  It is a subclass of
the class \texttt{t}.


{\small\Defgeneric {hash-table-p} {hash-table}
}

This generic function returns a generalized Boolean value, where a
\textsl{true} value indicates that \textit{hash-table} is an instance
of the class \texttt{hash-table}, and \textsl{false} indicates that
\textit{hash-table} is \emph{not} an instance of the class
\texttt{hash-table}.

{\small\Defgeneric {hash-table-count} {hash-table}
}

This generic function returns a non-negative integer indicating the
number of entries in \textit{hash-table}.

{\small\Defgeneric {gethash} {key hash-table \optional default}
}

This generic function returns two values.  The first value is the
value in \textit{hash-table} associated with \textit{key}, or
\texttt{nil} if no value in \textit{hash-table} is associated with
\textit{key}.  The second value is a generalized Boolean value, where
a \textsl{true} value indicates that the first value is indeed present
in \textit{hash-table}, and \textsl{false} indicates that no value is
associated with \textit{key} in \textit{hash-table}.

{\small\Defgeneric {(setf gethash)} {new-value key hash-table \optional default}
}

{\small\Defgeneric {hash-table-test} {hash-table}
}

\section{Implementation}

\Defclass {eq-hash-table-mixin}

This mixin class is a superclass of every hash tables class that uses
\texttt{eq} as its test function.  It is a subclass of the class
\texttt{t}.

{\small\Defmethod {hash-table-test} {(hash-table {\tt eq-hash-table-mixin})}
}

This method returns the symbol \texttt{eq}.

\Defclass {eql-hash-table-mixin}

This mixin class is a superclass of every hash tables class that uses
\texttt{eql} as its test function.  It is a subclass of the class
\texttt{t}.

{\small\Defmethod {hash-table-test} {(hash-table {\tt eql-hash-table-mixin})}
}

This method returns the symbol \texttt{eql}.

\Defclass {equal-hash-table-mixin}

This mixin class is a superclass of every hash tables class that uses
\texttt{equal} as its test function.  It is a subclass of the class
\texttt{t}.

{\small\Defmethod {hash-table-test} {(hash-table {\tt equal-hash-table-mixin})}
}

This method returns the symbol \texttt{equal}.

\Defclass {equalp-hash-table-mixin}

This mixin class is a superclass of every hash tables class that uses
\texttt{equalp} as its test function.  It is a subclass of the class
\texttt{t}.

{\small\Defmethod {hash-table-test} {(hash-table {\tt equalp-hash-table-mixin})}
}

This method returns the symbol \texttt{equalp}.

\Defclass {standard-hash-table}

This class is a subclass of the class \texttt{hash-table}.

\Definitarg {:contents}

This initialization argument is accepted by all instances of
\texttt{standard-hash-table}

{\small\Defgeneric {contents} {standard-hash-table}
}

Given an instance of the class \texttt{standard-hash-table}, this
generic function returns the value that was supplied as the
\texttt{:contents} initialization argument when the instance was
created.

\Defclass {list-hash-table}

This class is a subclass of the class \texttt{standard-hash-table}.
It provides and implementation of the protocol where the entries are
stored as an association list where the key is the \texttt{car} of the
element in the list and the value is the \texttt{cdr} of the element
in the list.
