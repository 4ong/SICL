\chapter{Garbage collector}

To fully appreciate the contents of this chapter, the reader should
have some basic knowledge of the usual techniques for garbage
collection.  We recommend ``The Garbage Collection Handbook''
\cite{Jones:2011:GCH:2025255} to acquire such basic knowledge.  We
also recommend Paul Wilson's excellent survey paper
\cite{Wilson:1992:UGC:645648.664824}.

We think it would be good to use a per-thread nursery combined with a
global allocator for older objects.

In \refChap{chap-data-representation} we suggest a data representation
where every heap allocated object has a two-word header object.  If it
is a \texttt{cons} cell, then that is the entire object.  For other
heap-allocated objects (called \emph{genral instances}, the first of
the two words is a tagged pointer to the class object, and the second
of the two words is a raw pointer to the \emph{rack}.  In
this chapter, we describe the consequences of this suggested
representation to the garbage collector.

\section{Global collector}
\subsection{General description}

No implementation work has been done yet on the global collector.

The global collector is a concurrent collector, i.e., it runs in
parallel with the mutator threads.  With modern processors, it is
probably practical to assign at least one core more or less
permanently to the global collector.  According to current thinking,
the global collector will be a combination of a mark-and-sweep
collector and a  traditional memory allocator as implemented by
\texttt{malloc()/free()} in a \clanguage{} environment.

We define a global heap divided into two parts.  The first part is a
single vector consisting of two-word blocks.  This is where
\texttt{cons} cells and the \emph{headers} of general instances are
allocated.  The second part of the global heap is organized as the
space managed by an ordinary memory allocator, for example the one
created by Doug Lea.

Since all general instances have a two-word header and \texttt{cons}
cells consist of two words as well, we can use a mark-and-sweep
collector for these objects without suffering any fragmentation.  The
advantage of a mark-and-sweep collector is that objects will never
move, which is an advantage when they are used as keys in hash tables
and when they are used to communicate with code in foreign languages
that assume that an address of an object is fixed once and for all.

The racks are allocated in the second part of the global heap.  As a
consequence, the racks also do not move once allocated.  This
fixed position is advantageous for code on some architectures.  For
example, the correspondence between source code location and values of
the program counter does not have to be updated as a result of code
being moved by the garbage collector.

Another great advantage of racks being in a permanent position is that
mutator threads can cache a pointer without the necessity of this
pointer having to be updated as a result of a garbage collection in
the global heap.  Garbage collection in the global heap can therefore
be done in parallel with the execution of the mutator threads.

The global collector is subject to a write barrier, so that if an
attempt is made to store a reference to an object in a nursery in an
object in the global heap, then the object being referred to and all
the objects in its transitive closure are first moved to the global
heap.  As a result, there are no references from the global heap to
objects in any nursery.  The write barrier is implemented as a test,
emitted by the compiler, to determine:

\begin{enumerate}
\item whether the object written to is indeed an object in the global
  heap, and
\item whether the datum being written is reference to a heap-allocated
  object (as opposed to an immediate object).
\end{enumerate}

In many cases, this test can be omitted as a result of \emph{type
  inference}, for instance if the datum being written can be
determined at compile time to be an immediate object.

The write barrier is tripped whether the reference to be stored is to
an object in the global heap or to an object in the local heap.  In
the first case, the write barrier is used to make sure there is not a
reference to a white object stored in a black object.  In the second
case, the write barrier is used to trigger a migration of local
objects to the global heap.

Objects in the global heap are \emph{allocated black}, i.e. it is
assumed that they are referenced.  When the global heap is about to be
collected, threads are first asked to garbage collect their nurseries
and then to report any references to objects in the global heap.  Once
every thread has finished this task, collection in the global heap can
start.  The first part of the heap is scanned and a free list of
objects that are no longer references is established.  Then, this free
list is traversed, effectively executing a \texttt{free()} of the rack
in the second part of the heap.

In order for marking by the global collector to be done in parallel
with mutator activity, we use a standard three-color marking
algorithm.  Again, we use a write barrier, so that when a reference to
a white object is about to be stored in a black object, we either one
of the two gray.  The choice can be based on the type of objects in
question, or it can be a global choice.  The write barrier is
implemented as additional instructions emitted by the compiler.  These
additional instructions can be avoided in many cases, and in
particular when the object being stored is known to be an immediate
object (fixnum, character, etc) or when it an unboxed object to be
stored in a specialized array.  The type inferencer can thus be used
to determine whether the write barrier can be elided.

\section{Nursery collector}
The directory \texttt{Code/Garbage-collector} contains some code that
uses a different data representation from what we now suggest.  

For the nursery, we suggest using a copying collector to manage small
(a few megabytes) linear space.  Instead of promoting objects that
survive a collection, we suggest using a \emph{sliding collector} in
the nursery.  Such a collector gives a very precise idea of the age of
different objects, so objects would always be promoted in the order of
the oldest to the youngest.  This technique avoids the problem where
the allocation of some intermediate objects is immediately followed by
a collection, so these objects are promoted even though they are
likely to die soon after the collection.  In a sliding collector,
promotion will happen only when a collection leaves insufficient space
in the nursery, at which point only the number of objects required to
free up enough memory would be promoted, and in the strict order of
oldest to youngest.

We suggest allocating header objects from the beginning of the nursery
and racks from the end.  A garbage collection is triggered
when the two free pointers meet.  We preserve the relative order both
of header objects and of racks. 

In the \emph{mark phase}, a single mark bit per header object is
used.  Header objects and racks are scanned, but only
header objects are marked.  

In the \emph{header compaction phase}, live header objects are compacted
toward the beginning of the heap.  A \emph{source} and a \emph{target}
pointer follow blocks in the bitmap and live words are moved. 

In the \emph{table build phase}, the bitmap is used to construct an
\emph{offset table} at the end of the lower part of the heap.  The
offset table contains a sequence of pairs $<a,o>$ meaning that a
pointer with an address less than or equal to $a$ (and greater than
the corresponding field of the previous pair) should be adjusted by
offset $o$.  There is always enough room for this table, because it
can contain at most as many entries as there are dead header objects.  

In the \emph{adjust phase}, header objects and racks are
scanned, and fields containing pointers are adjusted according to the
offset table.  The offset table is searched using binary search,
except that a simple caching scheme is used to avoid a full binary
search in nearly all cases. 

In the \emph{rack compaction phase}, header objects are scanned in
order of increasing addresses, and the corresponding racks
are compacted towards the end of the heap.

\section{Promotion}

Objects are typically promoted when the space recovered in the nursery
as a result of a thread-local garbage collection is deemed too small.%
\footnote{Here ``too small'' means that another garbage collection
  would be triggered very soon again, which is not desirable.}  When
this happens, a \emph{prefix} of the nursery (and the associated
racks in the suffix) is moved to the space managed by the
global collector.  Then, pointers are adjusted according to the size
of the prefix and the suffix. 

Promotion could happen for reasons other than age.  Objects that are
too large for the nursery would be allocated directly in the global
allocator.  Objects to which references from foreign code are
about to be created would first be promoted to the global collector
where they would no longer move.  The same is true for an object that
is about to be used as key in an \texttt{eq} hash table.  Objects that
can be expected to have a significant life span (such as symbols)
might also be allocated directly in the global collector. 

\section{Implementation}

In most systems, the garbage collector is implemented in some language
other than \commonlisp{}.  However, we imagine using \commonlisp{}
together with some additional low-level primitives for accessing
memory by address instead.

%%  LocalWords:  mutator
