\chapter{Processing arguments}

In this chapter, we describe how processing arguments is accomplished
by inserting HIR instructions immediately after HIR code is generated
from an abstract syntax tree.  By doing it this way, we obtain several
advantages:

\begin{itemize}
\item We simplify the translation of HIR code to LIR later on the
  translation process.
\item HIR transformations such as constant hoisting and
  \texttt{fdefinition} hoisting can be applied to the
  argument-processing code, thereby simplifying this code.
\item The HIR instructions introduced are subject to various HIR
  transformations such as value numbering, constant propagation,
  etc.
\end{itemize}

Each type of parameter is is handled by a different module.  In
addition, because of the complexity of initializing keyword
parameters, that module is is further subdivided.

Two new HIR instructions are used in order to accomplish the argument
processing:

\begin{itemize}
\item The \texttt{compute-argument-count-instruction} has no inputs,
  and a single output.  It is responsible for computing the total
  number of arguments passed to the function.
\item The \texttt{argument-instruction} has one input and one output.
  The input is a datum that must be a non-negative fixnum.  The output
  is the argument with an index represented by the value of the input,
  starting at $0$ for the first argument.
\end{itemize}

The overall organization of the modules for initializing parameters is
shown in \refFig{fig-process-arguments}.

\begin{figure}
\begin{center}
\inputfig{fig-process-arguments.pdf_t}
\end{center}
\caption{\label{fig-process-arguments}
Processing all arguments.}
\end{figure}

\section{Calling \texttt{error}}

To minimize the clutter in many of the figures in this chapter, when
an error situation is detected, a box labeled ``error'' is shown.  But
this box is slightly more complicated than a single instruction.  The
complete sequence of instructions is shown in
\refFig{fig-call-error}.


\begin{figure}
\begin{center}
\inputfig{fig-call-error.pdf_t}
\end{center}
\caption{\label{fig-call-error}
Calling \texttt{error}.}
\end{figure}

As \refFig{fig-call-error} shows, the sequences consists of two
instructions.

The first instruction is a \texttt{funcall-instruction} that calls the
function \texttt{error} as computed by the first instruction in
\refFig{fig-process-arguments}.  This instruction takes more arguments
to be passed to \texttt{error}.  The first additional argument is a
constant input containing the name of the condition class to signal.
Remaining additional arguments are initialization arguments for the
particular condition class.

The second instruction is an \texttt{unreachable-instruction}.  An
additional instruction is required, because the
\texttt{funcall-instruction} has a single successor.  The
\texttt{unreachable-instruction} indicates that control can not reach
this point.  This instruction has no successors.

\section{Checking the minimum argument count}

Unless there are no required parameters, there is a minimum allowed
value for the argument count, and it is equal to the number of
required parameters.  The HIR code for this check is simple and is
illustrated by \refFig{fig-check-minimum-argument-count}.

\begin{figure}
\begin{center}
\inputfig{fig-check-minimum-argument-count.pdf_t}
\end{center}
\caption{\label{fig-check-minimum-argument-count}
Checking minimum argument count.}
\end{figure}

As figure \refFig{fig-process-arguments} shows, this test is only
inserted when the number of required parameters is strictly greater
than $0$.  The input labeled ``Nr'' in
\refFig{fig-check-minimum-argument-count} is a constant input (known
at compile time) representing the number of required parameters, and
the input labeled ``AC'' is the lexical location computed by the
\texttt{compute-argument-count} instruction in
\refFig{fig-process-arguments}.

\section{Checking the maximum argument count}

Unless there is a \texttt{\&rest} parameter or there are
\texttt{\&key} parameters, there is a maximum allowed value for the
argument count, and it is equal to the sum of the number of required
parameters and the number of optional parameters.  The HIR code for
this check is also simple, and is illustrated by
\refFig{fig-check-maximum-argument-count}.

\begin{figure}
\begin{center}
\inputfig{fig-check-maximum-argument-count.pdf_t}
\end{center}
\caption{\label{fig-check-maximum-argument-count}
Checking maximum argument count.}
\end{figure}

As figure \refFig{fig-process-arguments} shows, this test is only
inserted when there is no \texttt{\&rest} parameter and no
\texttt{\&key} parameters. The input labeled ``Nr+No'' in
\refFig{fig-check-maximum-argument-count} is a constant input (known
at compile time) representing the number of required parameters plus
the number of optional parameters, and the input labeled ``AC'' is
again the lexical location computed by the
\texttt{compute-argument-count} instruction in
\refFig{fig-process-arguments}.

\section{Initializing required parameters}

\refFig{fig-initialize-required-parameters} illustrates the HIR code
for initializing required parameters.

\begin{figure}
\begin{center}
\inputfig{fig-initialize-required-parameters.pdf_t}
\end{center}
\caption{\label{fig-initialize-required-parameters}
Initializing required parameters.}
\end{figure}

The number of stages is determined at compile time and is equal to the
number of required parameters.  For each parameter, the next argument
(starting with $0$) is assigned to that parameter.

As \refFig{fig-process-arguments} suggests, this procedure can be
applied even when there are no required parameters.  In this case, the
result consists only of the final \texttt{nop-instruction}.

\section{Initializing optional parameters}

\refFig{fig-initialize-optional-parameters} illustrates the HIR code
for initializing optional parameters.

\begin{figure}
\begin{center}
\inputfig{fig-initialize-optional-parameters.pdf_t}
\end{center}
\caption{\label{fig-initialize-optional-parameters}
Initializing optional parameters.}
\end{figure}

As \refFig{fig-initialize-optional-parameters} shows, there are two
main branches in this code, the \emph{left} branch and the
\emph{right} branch.  The control flow starts in the \emph{right}
branch, and continues in that branch if until there are no more
arguments.  At that point, control is transferred to the \emph{left}
branch where the remaining optional parameters are initialized to
\texttt{nil}.

As \refFig{fig-initialize-optional-parameters} also shows, the code
ends with a test to check whether there are no more argument, even
though this test is not required in order to decide how any more
optional parameters should be initialized (because there are no more
at that point).  The reason for the existence of this test is so that
the left branch (no more arguments) can be used to determine whether
all keyword parameters should be initialized to \texttt{nil}, as
illustrated by \refFig{fig-process-arguments}.

As with the required parameters, this procedure can be applied even
when there are no optional parameters.  In this case, it degenerates
into the last test, i.e. it determines whether there are more
arguments than required parameters.

\section{Initializing keyword parameters to \texttt{nil}}

When there are no more arguments after all the required and all the
optional parameters have been initialized, then, if there are keyword
parameters, then they can all be initialized to \texttt{nil}.
This procedure is shown in \refFig{fig-no-more-arguments}.

\begin{figure}
\begin{center}
\inputfig{fig-no-more-arguments.pdf_t}
\end{center}
\caption{\label{fig-no-more-arguments}
Initializing keyword parameters to \texttt{nil}.}
\end{figure}

As \refFig{fig-process-arguments} shows, this HIR code is added only
if the lambda list contains \texttt{\&key} parameters.

\section{Creating the \texttt{\&rest} parameter}

\refFig{fig-create-rest-parameter} illustrates how the \texttt{\&rest}
parameter is created.

\begin{figure}
\begin{center}
\inputfig{fig-create-rest-parameter.pdf_t}
\end{center}
\caption{\label{fig-create-rest-parameter}
Creating the rest parameter.}
\end{figure}

As \refFig{fig-create-rest-parameter} shows, there is a loop that
starts with the last argument and ends with the first remaining
argument after the required and the optional parameters have been
initialized.  In each iteration of the loop, the function
\texttt{cons} is called in order to add another argument to the
beginning of the list.

As shown in \refFig{fig-process-arguments}, the \texttt{\&rest}
parameter, if present, has already been initialized to \texttt{nil}
before this HIR code is executed.

\section{Initializing keyword parameters}

\refFig{fig-process-keyword-arguments} illustrates how keyword
parameters are initialized.

\begin{figure}
\begin{center}
\inputfig{fig-process-keyword-arguments.pdf_t}
\end{center}
\caption{\label{fig-process-keyword-arguments}
Processing keyword arguments.}
\end{figure}

As \refFig{fig-process-keyword-arguments} shows, we subdivide the
initialization of keyword arguments further.  We start by checking
that there is an even number of keyword argument.  After that, each
keyword parameter is initialized in turn by a separate traversal of
the remaining arguments.  Finally, if required, we check whether the
keyword \texttt{:allow-other-keys} is present in the argument list,
and if so, what value it has.  We end the process by checking the
validity of each keyword argument, again only if required.

\subsection{Checking that the number of arguments is even}

\begin{figure}
\begin{center}
\inputfig{fig-check-even-keyword-arguments.pdf_t}
\end{center}
\caption{\label{fig-check-even-keyword-arguments}
Checking that there is an even number of keyword arguments.}
\end{figure}

\subsection{Initializing a single keyword parameter}

\begin{figure}
\begin{center}
\inputfig{fig-initialize-one-keyword-parameter.pdf_t}
\end{center}
\caption{\label{fig-initialize-one-keyword-parameter}
Initializing one keyword parameter.}
\end{figure}

\subsection{Checking the presence of \texttt{:allow-other-keys}}

\begin{figure}
\begin{center}
\inputfig{fig-check-allow-other-keys.pdf_t}
\end{center}
\caption{\label{fig-check-allow-other-keys}
Checking for \texttt{:allow-other-keys}}
\end{figure}

\subsection{Checking the validity of every keyword}

\begin{figure}
\begin{center}
\inputfig{fig-check-every-keyword.pdf_t}
\end{center}
\caption{\label{fig-check-every-keyword}
Checking the validity of every keyword.}
\end{figure}
