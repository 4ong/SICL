\chapter{Setf expanders}

The \hs{} requires%
\footnote{See figure 5.7 in section 5.1.2.2 in the \hs{}} 
the following function call forms to have a corresponding
\texttt{setf} form:

\begin{itemize}
\item Accessors for parts of a list: 
  car, cdr, caar, cadr, cdar, cddr, caaar, caadr, cadar, caddr,
  cdaar, cdadr, cddar, cdddr, caaaar, caaadr, caadar, caaddr, 
  cadaar, cadadr, caddar, cadddr, 
  cdaaar, cdaadr, cdadar, cdaddr, 
  cddaar, cddadr, cdddar, cddddr, 
  first, second, third, fourth, fifth,
  sixth, seventh, eighth, ninth, tenth
  rest, nth.
\item Array element accessors: 
  aref, row-major-aref, char, schar, bit, sbit, svref.
\item Other array accessors: fill-pointer
\item Sequence element accessors: elt.
\item Other sequence accessors: subseq.
\item Symbol properties: symbol-plist.
\item Environment accessors:
  symbol-function, symbol-value, fdefinition, 
  macro-function, compiler-macro-function.
\item Hash table accessors: gethash.
\item CLOS-related accessors: class-name, slot-value, find-class.
\item Miscellaneous: documentation,
  logical-pathname-translations, get, 
  readtable-case.
\end{itemize}

The \hs{} also gives the implementer a choice concerning the
implementation of \texttt{setf} forms either as functions or as
\texttt{setf} expanders.  For \sysname{} we always choose a function
whenever possible.  Consequently, every \texttt{setf} form in
the list above is implemented as a function.


