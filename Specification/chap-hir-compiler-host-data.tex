\chapter{HIR compiler with host data}

\section{Characteristics}

This backend is characterized by the fact that source code is compiled
to the high-level intermediate representation (HIR), and then
translated to very rudimentary \commonlisp{} which is compiled using
the host compiler.  Recall that the high-level intermediate code only
manipulates \commonlisp{} objects so that all address calculations are
implicit.  All objects such as symbols, numbers, packages, etc. are
host objects.

At the center of this backend is a \sysname{} first-class global
environment represented as a host class instance.  This environment
contains mappings from names to all objects that are part of any
global environment, such as functions, macros, variables, classes,
packages, types, \texttt{setf} expanders, etc.

\section{Bootstrapping}

A host function can be imported into the target environment.  However,
such imports must be temporary, because if a host function detects an
error, it will call the host function \texttt{error} which uses the
binding stack of the host to search for handlers.  Therefore, in the
complete backend, as many host functions as possible must be replaced
by equivalent target functions.

Notice that it is not generally possible to import host \emph{macros}
into the target environment.  The reason is that host macros might
expand into host-specific code, which the target environment can not
process.  Some host macros (such as some simple conditional macros)
expand into completely portable code, but since this set can vary
between different hosts, to be on the safe side, we use only target
macros.

This backend is created as follows:

\begin{enumerate}
\item All standard \commonlisp{} functions are imported from the host
  and given the same name in the target environment.
\item Similarly, all generic functions that are part of the protocol
  for first-class global environments are imported.  The names of
  these functions are in the package named
  \texttt{sicl-global-environment}, aliased \texttt{sicl-env}.
\item The target macro \texttt{defmacro} is created artificially by
  creating an explicit host macro function and associating it with the
  name \texttt{defmacro} in the target environment.
\item Using the now-existing macro \texttt{defmacro}, other macros
  that are needed early are created, such as \texttt{defun},
  \texttt{defparameter}, etc.  If any new macros need support code in
  the form of functions, these functions are imported first as host
  functions.
\item Remaining macros such as the conditional macros and
  \texttt{setf}-related macros, are created the same way.
\item Functions that manipulate the environment are defined as target
  functions using the \texttt{defun} macro.  Examples of such
  functions are \texttt{fdefinition}, \texttt{(setf fdefinition)},
  \texttt{fboundp}, etc.  These functions call the host functions
  imported in step 2.
\item A temporary simplified version of the function
  \texttt{ensure-generic-function} is created as a target function.
  We can not use the equivalent host function, because the
  specifications for this function include the creation of a function
  binding in the environment.  In some sense, it is therefore in the
  same category as the other environment-manipulating functions.  This
  version always creates a new generic function by calling the
  imported host function \texttt{make-instance} on the class named
  \texttt{standard-generic-function}.
\end{enumerate}
