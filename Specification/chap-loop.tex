\chapter{The \texttt{loop} macro} 

Our \texttt{loop} module uses all available Common Lisp functions for
its analysis of syntax and semantics.  We believe this is not a
problem, even though we assume the existence of \texttt{loop} for many
other modules, because the code in this module will be executed during
macro-expansion time, and for a new Common Lisp system, it would be
executed during cross compilation by another full Common Lisp
implementation.

Our \texttt{loop} module uses only standard Common Lisp code in its
expansion, so that macro-expanded uses of \texttt{loop} will not
require any other \sysname{} module in order to work.

The code for the \sysname{} \texttt{loop} macro is located in the
directory \texttt{Code/Loop}.  

For parsing a \texttt{loop} expression, we use a technique called
\emph{combinatory parsing}, except that we do not handle arbitrary
backtracking.  Luckily, arbitrary backtracking is not required to
parse the fairly simple syntax of \texttt{loop} clauses.

\section{Current state}

All \texttt{loop} clauses have been tested with the test cases
provided by Paul Dietz' ANSI \commonlisp{} test suite.

Future work includes providing an alternative parser to be used when
the normal parser fails.  The purpose of the alternative parser is to
provide good error messages to the programmer.

\section{Protocol}

\Defgeneric{bound-variables} {clause}

The purpose of this generic function is to generate a list of all
bound variables in a clause.  The same variable occurs as many times
in the list as the number of times it is bound in the clause.

\Defgeneric{accumulation-variables} {clause}

The purpose of this generic function is to generate a list of all the
accumulation variables in a clause.  Each element of the list is
itself a list of three elements.  The first element is the name of a
variable used in an \texttt{into} clause, or \texttt{nil} if the
clause has no \texttt{into}.  The second element determines the kind
of accumulation, and can be one of the symbols \texttt{list},
\texttt{count/sum}, or \texttt{max/min}.  The third element is a type
specifier which can be \texttt{t}.

\Defgeneric{declarations} {clause}

The purpose of this generic function is to extract a list of
declaration specifiers from the clause.  Notice that it is a list of
declaration specifiers, not a list of declarations.  In other words,
the symbol \texttt{declare} is omitted.

\Defgeneric {initial-bindings} {clause}

The purpose of this generic function is to extract the outermost level
of bindings required by the clause.
