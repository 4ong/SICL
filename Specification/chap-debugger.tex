\chapter{Debugger}
\label{chap-debugger}

Part of the reason for SICL is to have a system that provides
excellent debugging facilities for the programmer.  The kind of
debugger we plan to support is described in a separate repository.%
\footnote{See https://github.com/robert-strandh/Clordane}  In this
chapter, we describe only the support that \sysname{} contains in
order to make such a debugger possible.

The execution of every function starts by testing a \emph{flag} passed
in a register.%
\footnote{For the x86-64 platform, it is register RAX.}
This flag indicates whether the current thread is being debugged.  The
function contains two versions of the code, called the \emph{normal}
version and the \emph{debugging} version.%
\footnote{This idea was suggested by Michael Raskin.}
The flag determines which version is chosen.  Thus, when the thread is
not being debugged, the only overhead is this single test at the
beginning of the function.  Furthermore, once the test is done, the
register is no longer needed for this purpose, and is at the disposal
of the register allocator for the remainder of the function body.

The normal version is used when the thread is not run under the
control of the debugger, so this version does not contain any code for
communicating with the debugger.  Furthermore, this version is highly
optimized.  In particular, variables that occur in the source code may
have been eliminated by various optimization passes.  A function call
in the normal version clears the flag register, so that the callee can
choose its normal version as well.

The debugging version starts by examining a special variable that
contains information about the current thread.  In particular, this
information includes a table in the form of a bit vector containing
summary information about breakpoints.  In this version of the
function body the compiler inserts a call to a small subroutine before
and after every form to be evaluated.  The subroutine does not use the
full \commonlisp{} function-call protocol.  Instead, it is just a very
fast call that can be done with a \texttt{jsr} instruction (or
equivalent) on must architectures.

The subroutine does a test in two steps.  In the first step, the value
of the program counter is taken modulo some reasonably large value
such as 256, and a the bit vector is queried to see whether the
corresponding entry is a 1.  If it is 0, the subroutine simply
returns.  This first step will slow down every debugged thread a
little bit, but most of the time, the value will be 0, and then,
normal function execution is resumed.

If the entry in the bit vector turns out to be 1, then the final test
is made.  The program counter is checked against a hash table in the
thread instance to see whether some action must be taken.  If so, the
thread gives up control to the debugger.

A function call in the debugging version sets the flag register to 1,
so that the callee can choose its debugging version as well.  The
debugging version does not have optimizations applied to it that may
make debugging harder.  Lexical variables that appear in source code
may be kept, or code may be included that can compute their values
from the lexical variables that \emph{are} kept, for the duration of
their scope.

%% For each possible breakpoint, the system must keep a description of
%% the lexical environment.  This includes mappings from variable names
%% to registers or stack locations, information about liveness of
%% registers and stack location, how a variable is stored in a location
%% (immediate value, pointer, with or without type tag, etc).

