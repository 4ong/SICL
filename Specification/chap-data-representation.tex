\chapter{Data representation}
\label{chap-data-representation}

\section{Low-level tag bits}

The two least significant bits of a machine word are used to represent
four different \emph{tags} as follows:

\begin{itemize}
\item \texttt{00}.  This tag is used for fixnums.  The remaining bits
  represent a signed integer in two's complement representation.  On a
  32-bit machine, a fixnum is thus in the interval $[2^{-29}, 2^{29} -
    1]$.  On a 64-bit machine, a fixnum is in the interval $[2^{-61}, 2^{61} -
    1]$.  
\item \texttt{01}.  This tag is used for \texttt{cons} cells.  A
  pointer to a \texttt{cons} cell is thus a pointer aligned to a
  double word to which the machine integer $1$ has been added.  On a
  32-bit machine, the bit representing $2^2$ will always have the
  value $0$.  On a 64-bit machine, the bits representing $2^2$ and
  $2^3$ will always have the value $0$.
\item \texttt{10}.  This tag is used for various \emph{immediate}
  objects.  In particular for \emph{characters}, but also a
  distinguished bit pattern that is used to initialize \emph{unbound}
  variables and slots.  On 64-bit platforms \emph{single-floats} are
  represented as immediate values. 
\item \texttt{11}.  This tag is used for all heap-allocated objects
  other than \texttt{cons} cells.
\end{itemize}

\section{Immediate objects}

\subsection{Characters}

\subsection{Unbound}

\subsection{Short floats}

\subsection{Single floats}

\section{Representation of \texttt{cons} cells}

A \texttt{cons} cell is represented as two consecutive machine
words aligned on a double-word boundary.

\section{Representation of other heap objects}

All heap objects other than \texttt{cons} cells are represented in (at
least) two parts, a \emph{header} object and a \emph{contents} vector.
The header object always consists of two consecutive words aligned on
a double-word boundary (just like \texttt{cons} cells).  The first
word always contains a tagged pointer to a \emph{class} object (which
is another heap object).  The second word contains a \emph{raw
  pointer} to the first word of the contents vector.  In the following
sections we give the details of the representation for all possible
heap objects.

\section{Subclasses of \texttt{standard-object}}

\section{Arrays}

An array being a heap object, it is represented as a two-word header
object and a contents vector.  For a \emph{simple} array the contents
vector contains:

\begin{itemize}
\item A word containing a fixnum that represents the \emph{rank} of
  the array.
\item A word containing a fixnum for each dimension of the array.
\item A number of consecutive words holding the contents of the
  array. 
\end{itemize}

The element type of the array is determined by the exact class of the
array. 

\section{Strings}

A string is a subtype of \texttt{array}.  Tentatively, we think that
there is no need to optimize strings that contain only characters that
could be represented in a single byte.  Thus the contents vector of a
\emph{simple} string is represented as follows:

\begin{itemize}
\item A fixnum with the value of 1.
\item A fixnum containing the number of characters in the string.
\item A number of consecutive words, each holding a tagged immediate
  object representing a Unicode character.
\end{itemize}

\section{Symbols}

\section{Packages}

\section{Hash tables}

\section{Streams}



