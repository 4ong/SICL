\chapter{Data representation}
\label{chap-data-representation}

\section{Low-level tag bits}

On 32-bit backends, the two least significant bits of a machine word
are used to represent four different \emph{tags} as follows:

\begin{itemize}
\item \texttt{00}.  This tag is used for fixnums.  The remaining bits
  represent a signed integer in two's complement representation.  On a
  32-bit machine, a fixnum is thus in the interval $[2^{-29}, 2^{29} -
    1]$.
\item \texttt{01}.  This tag is used for \texttt{cons} cells.  A
  pointer to a \texttt{cons} cell is thus a pointer aligned to a
  double word to which the machine integer $1$ has been added.  The
  bit representing $2^2$ will always have the value $0$.
\item \texttt{10}.  This tag is used for various \emph{immediate}
  objects.  In particular for \emph{characters}, but also a
  distinguished bit pattern that is used to initialize \emph{unbound}
  variables and slots.  Floats of type \texttt{short-float} are
  represented as immediate values. 
\item \texttt{11}.  This tag is used for all heap-allocated objects
  other than \texttt{cons} cells.  A heap allocated object like this
  is called a \emph{general instance}.
\end{itemize}

On 64-bit backends, the three least significant bits are used as tag
bits as follows:

\begin{itemize}
\item \texttt{000}, \texttt{010}, \texttt{100}, \texttt{110}.  These
  tags are used for fixnums.  The bits except the last one represent
  integer in two's complement representation.  On a 64-bit machine, a
  fixnum is thus in the interval $[2^{-62}, 2^{62} - 1]$.
\item \texttt{001}.  This tag is used for \texttt{cons} cells.  A
  pointer to a \texttt{cons} cell is thus a pointer aligned to a
  double word to which the machine integer $1$ has been added.  The
  bit representing will always have the value $0$.
\item \texttt{011}, \texttt{101}.  These tags are used for various
  \emph{immediate} objects.  In particular for \emph{characters}, but
  also a distinguished bit pattern that is used to initialize
  \emph{unbound} variables and slots.  Floats of type
  \texttt{short-float} and \texttt{single-float} are represented as
  immediate values.
\item \texttt{111}.  This tag is used for all heap-allocated objects
  other than \texttt{cons} cells.  A heap allocated object like this
  is called a \emph{general instance}.
\end{itemize}

\section{Immediate objects}

On 32-bit backends, immediate objects are all objects with \texttt{10}
in the lower two bits.  Two more bits are used to distinguish between
different kinds of immediate objects, giving the following four low
bits:

\begin{itemize}
\item \texttt{0010}.  This tag is used for Unicode characters.  When
  shifted four positions to the right, the value gives the Unicode
  code point. 
\item \texttt{0110}.  This tag is used short floats.  
\item \texttt{1010}.  This tag is used for single floats (64-bit
  platforms only). 
\item \texttt{1110}.  This tag is used for various unique immediate
  objects, in particular for the object that is stored in various
  locations to indicate that the location is \emph{unbound}. 
\end{itemize}

\subsection{Characters}

As indicated above, the low four bits of a character have the value
\texttt{0010}, and the corresponding Unicode code point is obtained by
shifting the value of the character four bits to the right. 

We currently do not plan to supply a module for Unicode support.
Instead we are relying on the support available in the Unicode library
by Edi Weitz.

\subsection{Short floats}

A short float is an immediate floating-point value that does not fall
into any of the IEEE floating-point categories.  On a 32-bit platform,
it is similar to a single-precision IEEE floating-point value, with
the difference that four fewer bits are used for the mantissa.  On a
65-bit platform, it is similar to a double-precision IEEE
floating-point value, again with the difference that four fewer bits
are used for the mantissa.  Notice that on a 64-bit platform, a short
float is more precise than a single float. 

\subsection{Single floats}

A single float is an IEEE single-precision (32-bit) floating-point
value stored in the most significant half of a 64-bit word.  The least
significant half of the word is unused except for the tag bits. 

\subsection{Unbound}
\label{data-representation-unbound}

The unbound object is represented as a word in which every bit except
the least significant one is equal to 1.  As a signed machine integer,
this value is equal to $-2$.  The unbound immediate object is used
as the value of special variables and of slots in arrays and instances
of (subclasses of) \texttt{standard-object}.  It is \emph{not} used to
indicate that a \emph{function name} is unbound.  For that purpose, a
special function that signals an error is used instead. 

\section{Representation of \texttt{cons} cells}

A \texttt{cons} cell is represented as two consecutive machine
words aligned on a double-word boundary.

\section{Representation of general instances}
\label{sec-data-representation-general-instances}

Recall that a \emph{general instance} is a heap allocated object that
is not a \texttt{cons} cell.  
All general instances are represented in (at
least) two parts, a \emph{header object} and a \emph{rack}.
The header object always consists of two consecutive words aligned on
a double-word boundary (just like \texttt{cons} cells).  The first
word always contains a tagged pointer to a \emph{class} object (which
is another general instance).  The second word contains a \emph{raw
  pointer} to the first word of the rack.  In most cases,
the only place that can contain a raw pointer to a rack is
the second word of a header object.  However, in some cases (for
performance reasons) a raw pointer to a rack can be the
value of some slot.  In this case, we say that we have a
\emph{disembodied} pointer to a rack.  An object containing
such a disembodied pointer must be known to the garbage collector, and
the object must also contain (directly or indirectly) an ordinary
tagged \cl{} pointer to a header object containing that rack..%
\footnote{The reason for this restriction is that it must always be possible
  to find the header object of a rack so as to determine
  the class of which the object is an instance.  If we could have
  an object X that contained only the disembodied raw pointer to the
  rack, then it may be the case that the pointer to the
  header object is no longer referenced by any other object, whereas
  the disembodied pointer in X is still reachable.  Then it would be
  impossible to determine the class of the object.}

The first entry of the rack of every general instance is a
small fixnum called the \emph{stamp} of the general instance.  The
stamp is the \emph{unique class number} of the class of the general
instance as it was when the instance was created.  The main purpose of
this information is to be used in \emph{generic function dispatch}.
It is also used to determine whether a general instance is an obsolete
instance (in this case the stamp of the general instance will not be
the same as the \emph{current} unique class number of the class of the
general instance).  Standard instances that can become obsolete are
said to be \emph{flexible}.
\seesec{sec-data-representation-standard-objects}

One advantage of representing general instances this way is that the
rack is \emph{internally consistent}.  To explain what we
mean by this concept, let us take an \emph{adjustable array} as an
example.  The implementation of \texttt{aref} must check that the
indices are valid, compute the offset of the element and then access
the element.  But in the presence of threads, between the index check
and the access, some other thread might have adjusted the array so
that the indices are no longer valid.  In most implementations, to
ensure that \texttt{aref} is \emph{thread safe}, it is necessary to
prevent other threads from intervening between the index check and the
access, for instance by using a \emph{lock}.  In \sysname{}, adjusting
the array involves creating a new rack with new dimensions,
and then with a single store instruction associate the new rack
with the array.  The implementation of \texttt{aref} would then
initially obtain a pointer to the rack and then do the
index check, the computation of the offset, and the access without
risking any interference.  No locking is therefore required.  Another
example is a \emph{generic function}.  When a method is added or
deleted, or when a new sequence of argument classes is seen, the
generic function must be destructively updated.  Normally, this
operation would require some locking primitive in order to prevent
other threads from invoking a partially updated generic function.  In
\sysname{}, to update a generic function this way, a new rack
would be allocated and the modifications would be made there,
leaving the original generic function intact until the final
instruction to store a reference to the the new rack in the
header object. 

A general instance can be \emph{rigid} or \emph{flexible}.  A rigid
instance is an instance of a class that can not change, typically a
built-in class.  A flexible instance is an instance of a class that
may be modified, makings its instances \emph{obsolete}.  Standard
objects are flexible instances, but other instances might be flexible
too.  In \sysname{}, structure objects are flexible too, for
instance. 

In the following sections we give the details of the representation
for all possible general instances.

\section{Flexible instances}
\label{sec-data-representation-flexible-instances}

a flexible instance must allow for its class to be redefined, making
it necessary for the instance to be updated before used again.  The
standard specifically allows for these updates to be delayed and not
happen as a direct result of the class redefinition.  They must happen
before an attempt is made to access some slot (or some other
information about the slots) of the instance.  It is undesirable to
make the all instances directly accessible from the class, because
such a solution would waste space and would have to make sure that
memory leaks are avoided.  We must thus take into account the presence
of \emph{obsolete instance} in the system, i.e., instances that must
be \emph{updated} at some later point in time.

The solution is to store some kind of \emph{version} information in
the rack so that when an attempt is made to access an
obsolete instance, the instance can first be updated to correspond to
the current definition of its class.  This version information must
allow the system to determine whether any slots have been added or
removed since the instance was created.  Furthermore, if the garbage
collector traces an obsolete instance, then it must either first
update it, or the version information must allow the garbage collector
to trace the obsolete version of the instance.  Our solution allows
both.  We simply store a reference to the \emph{list of effective
  slots} that the class of the instance defined when the instance was
created.  This reference is stored as the \emph{second} word of the
rack (recall that the first word is taken up by the
\emph{stamp}). \seesec{sec-data-representation-general-instances}

This solution makes it possible to determine the layout of the
rack of an obsolete instance, so that it can be traced by
the garbage collector when necessary.  This solution also allows the
system to determine which slots have been added and which slots have
been removed since the instance was created.  As indicated in
\refSec{sec-data-representation-general-instance}, in order to detect
whether an object is obsolete, the contents of the first word of the
rack (i.e,, the \emph{stamp}) is compared to the
\emph{class unique number} of the class of the object.  However, this
test is performed automatically in most cases, because when an
obsolete object is passed as an argument to a generic function, the
\emph{automation} of the discriminating function of the generic
function will fail to find an effective method, triggering an update
of the object.
\seesec{sec-generic-function-dispatch-the-discriminating-function}

\section{Standard objects}
\label{sec-data-representation-standard-objects}

By definition, a \emph{standard object} is an instance of a subclass
of the class with the name \texttt{standard-object}.  Standard objects
are \emph{flexible instances}.
\seesec{sec-data-representation-flexible-instances} 

Subclasses of \texttt{standard-object} must allow for the class of an
instance to be changed to some other class (\texttt{change-class}) and
for the definition of the class of the instance to be modified.
Changing the class of an instance is fairly straightforward because
the instance is then passed as an argument and the slots of the
instance can be updated as appropriate.


\section{Funcallable standard objects}
\label{sec-data-representation-funcallable-standard-objects}

By definition, a \emph{funcallable standard object} is an instance of
a subclass of the class \texttt{funcallable-standard-object} which is
itself a subclass of the class \texttt{standard-object}
\seesec{sec-data-representation-standard-objects} and of the class
\texttt{function}. \seesec{sec-data-representation-functions}

To make function invocation fast, we want every subclass of the class
\texttt{function} to be invoked in the same way, i.e. by loading the
\emph{static environment} into a register and then transferring
control to the \emph{entry point} of the function. The static
environment and the entry point are stored in the function object
and are loaded into registers as part of the function-call protocol.

When the funcallable standard object is a generic function, invoking
it amounts to transferring control to the \emph{discriminating
  function}.  However, the discriminating function can not \emph{be}
the generic function, because the \clos{} specification requires that
the discriminating function of a generic function can be replaced,
without changing the identity of the generic function itself.
Furthermore, the discriminating function does not have to be stored in
a slot of the generic function, because once it is computed and
installed, it is no longer needed.  In order for the generic function
itself to behave in exactly the same way as its discriminating
function, whenever a new discriminating function is \emph{installed},
the \emph{entry point} and the \emph{static environment} are copied
from the discriminating function to the corresponding slots of the
generic function itself.

\section{Rigid instances}
\label{sec-data-representation-rigid-instances}

Contrary to \emph{flexible instances}, a rigid instance is an instance
of a class that is not allowed to change after the first instance is
created.  The class definition might change as long as there are
no instances, but the consequences are undefined if a built-in class
is changed after it has been instantiated.

As a consequence, in rigid instances it is not necessary to keep a
copy of the class slots in the rack.  However, the first
cell of the rack of such instances still contain the
\emph{stamp}, so that every general instance can be treated the same
by all generic functions, thus avoiding a test for a special case.

\section{Instance of built-in classes}

Instances of a built-in classes are \emph{rigid instances}. 
\seesec{sec-data-representation-rigid-instances}

\subsection{Instances of \texttt{sequence}}

The system class \texttt{sequence} can not be directly instantiated.
Instead, it serves as a superclass for the classes \texttt{list} and
\texttt{vector}.  

The \hs{} is a bit contradictory here, because
in some places it says that \texttt{list} and \texttt{vector}
represent an exhaustive partition of \texttt{sequence}%
\footnote{See for instance section 17.1}
but in other places it explicitly allows for other subtypes of
\texttt{sequence}.%
\footnote{See the definition of the system class \texttt{sequence}.}
The general consensus seems to be that other subtypes are allowed. 


\subsection{Arrays}
\label{sec-data-representation-arrays}

An array being a general instance, it is represented as a two-word
header object and a rack.  The second word of the rack
contains a proper list of dimensions.  The length of the list
is the \emph{rank} of the array.

If the array is \emph{simple}, the second word is directly followed
by the elements of the array.  If the array is \emph{not} simple, then
it is either displaced to another array, or it has a fill pointer, or
both.  If it has a fill pointer, then it is stored in the third word
of the rack.  Finally, if the array is displaced to another
array, the rack contains two words with the array to which
this one is displaced, and the displacement offset.  If the array is
not displaced, then the elements of the array follow.  The size of the
rack is rounded up to the nearest multiple of a word. 

The exact class of the array differs according to whether the array is
simple, has a fill pointer, or is displaced. 

All arrays are \emph{adjustable} thanks to the split representation
with a header object and a rack.  Adjusting the array
typically requires allocating a new rack. 

The element type of the array is determined by the exact class of the
array. 

We suggest providing specialized arrays for the following data types:

\begin{itemize}
\item \texttt{double-float}
\item \texttt{single-float}
\item \texttt{(unsigned-byte 64)}.
\item \texttt{(signed-byte 64)}.
\item \texttt{(unsigned-byte 32)}.
\item \texttt{(signed-byte 32)}.  
\item \texttt{(unsigned-byte 8)}, used for code, interface with the
  operating system, etc. 
\item \texttt{character} (i.e., strings) as required by the \hs{}.
\item \texttt{bit}, as required by the \hs{}.
\end{itemize}

Since the element type determines where an element is located and how
to access it, \texttt{aref} and \texttt{(setf aref)} are \emph{generic
  functions} that specialize on the type of the array. 

\subsubsection{System class \texttt{vector}}

A vector is a one-dimensional array.  As such, a vector has a rack
where the second word is a proper list of a single element,
namely the \emph{length} of the vector represented as a fixnum. 

The remaining words of the rack contain an optional fill
pointer, and then either the elements of the vector or displacement
information as indicated above. 

\subsubsection{System class \texttt{string}}

A string is a subtype of \texttt{array}.  Tentatively, we think that
there is no need to optimize strings that contain only characters that
could be represented in a single byte.  Thus the rack of a
\emph{simple} string is represented as follows:

\begin{itemize}
\item The obligatory \emph{stamp}.
\item A list of a single element corresponding to the \emph{length} of
  the string. 
\item A number of consecutive words, each holding a tagged immediate
  object representing a Unicode character.
\end{itemize}

\subsection{Symbols}

A symbol is represented with a two-word header (as usual) and a
rack of four consecutive words.  The four words contain:

\begin{enumerate}
\item The obligatory \emph{stamp}.
\item The \emph{name} of the symbol.  The value of this slot is a
  string.
\item The \emph{package} of the symbol.  The value of this slot is a
  package or \texttt{NIL} if this symbol does not have a package.
\item The \emph{plist} of the symbol.  The value of this slot is a
  property list.
\end{enumerate}

Notice that the symbol does not contain its \emph{value} as a global
variable, nor does it contain its definition as a \emph{function} in
the global environment.  Instead, this information is contained in an
explicit \emph{global environment} object.

\subsection{Packages}

A package is represented with a two-word header (as usual) and a
rack of 8 words:

\begin{enumerate}
\item The obligatory \emph{stamp}.
\item The \emph{name} of the package.  The value of this slot is a
  string.
\item The \emph{nicknames} of the package.  The value of this slot is
  a list of strings. 
\item The \emph{use list} of the package.  The value of this slot is a
  proper list of packages that are used by this package. 
\item The \emph{used-by list} of the package.  The value of this slot
is a proper list of packages that use this package. 
\item The \emph{external symbols} of the package.  The value of this
  slot is a proper list of symbols that are both present in and
  exported from this package.
\item The \emph{internal symbols} of the package.  The value of this
  slot is a proper list of symbols that are present in the package but
  that are not exported.
\item The \emph{shadowing symbols} of the package.  The value of this
  slot is a proper list of symbols. 
\end{enumerate}

\subsection{Hash tables}

\subsection{Streams}

\subsection{Functions}
\label{sec-data-representation-functions}

The \emph{class} of a function is a subclass of \texttt{function}.

In order to obtain reasonable performance, we represent functions in a
somewhat complex way, as illustrated by
\refFig{fig-function-representation}. 

\begin{figure}
\begin{center}
\inputfig{fig-function-representation.pdf_t}
\end{center}
\caption{\label{fig-function-representation}
Representation of functions.}
\end{figure}

\refFig{fig-function-representation} shows two functions.  The two
functions were created from the same compilation unit, because they
share the last \texttt{cons} cell of the static environment.

A function is represented as a two-word header (as usual) and a
rack with three slots:

\begin{enumerate}
\item The obligatory \emph{stamp}.
\item An \emph{environment}, which is the local environment in which
  the function was defined.  The local environment is represented as a
  proper list of simple vectors, one for each \emph{level} of the
  environment according to the nesting depth of the function.  The
  last element of the list contains the \emph{linkage vector} which
  contains \emph{constants} and \emph{storage cells of global
    functions} that are referenced by the function.  The first element
  of the linkage vector is the \emph{code object} of the function (see
  below).  Several functions may share the same linkage vector. 
\item The \emph{entry point}.  The each entry
  point is a \emph{raw address} of an aligned word in the vector
  containing the instructions of the function.  The garbage collector
  must update the value of this slot if it decides to move the
  rack of the code vector.
\end{enumerate}

As indicated above, the \emph{linkage vector} of a function is a
simple vector, and it is the last element of the proper list that
makes up the \emph{static environment} of the function.  The most
important purpose of the linkage vector is to contain \emph{constants}
and \emph{storage cells} for global functions that are referenced by
the function.  

Since raw addresses are word aligned, they show up as \texttt{fixnum}s
when inspected by tools that are unaware of their special
signification.  

A \emph{code object} is represented as a two-word header (as usual)
and a rack containing (in addition to inherited slots):

\begin{enumerate}
\item The \emph{linkage vector} which is an ordinary \cl{} simple
  vector.
\item The \emph{code vector} which is an ordinary \cl{} vector with
  elements of type \texttt{(unsigned-byte 8)} containing the machine
  instructions for one or more functions.
\item Correspondence between values of the program counter
  (represented as offsets into the byte vector of instructions) and
  source code locations.  
\item Correspondence between values of the program counter and the
  contents of the local environment.  This information is used by the
  garbage collector and by the debugger. 
\end{enumerate}

By having several functions share the same linkage vector, we can
simplify calls between top-level functions in the same compilation
unit, because the caller and the callee would then share the
same static environment.  In contrast, a call from a function in one
compilation unit to a function in a different compilation unit
involves accessing the static environment of the callee and storing it
in a register.

For a \emph{generic function}, the description of the slots above
applies both to the generic function object itself and to the
\emph{discriminating function} of the generic function.  In addition
to these slots, a generic function also contains other slots holding
the list of its methods, and other information.

When a function is called, there are several possible situations that
can occur:

\begin{itemize}
\item The most general case is when an object of unknown type is given
  as an argument to \texttt{funcall}.  Then no optimization is
  attempted, and \texttt{funcall} is responsible for determining
  whether the object is a function, a name that designates a function,
  or an object that does not designate a function in which an error is
  signaled. 
\item When it can be determined statically that the object called is a
  function (i.e. its class is a subclass of the class
  \texttt{function}), but nothing else is know about it, then an
  \emph{external call} is made.  Such a call consists of copying the
  contents of the \emph{static environment} slot to the predetermined
  place specific to the backend, and then to issue a \emph{call}
  instruction (or equivalent) to the address indicated by the
  \emph{entry point} slot of the function object.  When the
  call is to a \emph{global function}, then the linkage vector contains
  a \texttt{cons} cell whose \texttt{car} is guaranteed to contain a
  function object, so this situation is applicable in this case.
\item When it can not only be determined statically that the object
  being called is a function, but also that the number of arguments
  that will be passed to it is acceptable, then the alternative entry
  point can be used.  This is often the case when the function being
  called is a \emph{global system function}.  Then the alternative
  entry point is computed as a constant offset from the normal entry
  point.
\item When it can be determined statically that the object being
  called is a function object in the \emph{same compilation unit} as
  the caller, then we can make an \emph{internal call}.  If both the
  caller and the callee are \emph{global functions} (so that the
  static environment contain only the linkage vector for both), then
  it suffices to issue a \texttt{call} instruction (or equivalent) to
  a relative address that can be determined statically.  The static
  environment does not need to be passed because it is shared between
  the caller and the callee.  The relative address can be chosen so as
  to avoid type checking of arguments with known types.
\item Even when it can not be determined statically that the object
  being called is a function, there are some situations that can be
  optimized.  For instance, when it is likely that the object is going
  to be called multiple times with the same number of arguments, then
  it may pay off to start by checking that the object is a function,
  then accessing its \emph{lambda list} to determine that the number
  of arguments is acceptable, and then cache the static environment
  and the alternative entry point in \emph{local registers}.  A
  typical use for this situation is in the \emph{sequence functions}
  for the keyword arguments \texttt{:test}, \texttt{:test-not}, and
  \texttt{:key}.
\end{itemize}
