\chapter{Data representation}
\label{chap-data-representation}

\section{Low-level tag bits}

The two least significant bits of a machine word are used to represent
four different \emph{tags} as follows:

\begin{itemize}
\item \texttt{00}.  This tag is used for fixnums.  The remaining bits
  represent a signed integer in two's complement representation.  On a
  32-bit machine, a fixnum is thus in the interval $[2^{-29}, 2^{29} -
    1]$.  On a 64-bit machine, a fixnum is in the interval $[2^{-61}, 2^{61} -
    1]$.  
\item \texttt{01}.  This tag is used for \texttt{cons} cells.  A
  pointer to a \texttt{cons} cell is thus a pointer aligned to a
  double word to which the machine integer $1$ has been added.  On a
  32-bit machine, the bit representing $2^2$ will always have the
  value $0$.  On a 64-bit machine, the bits representing $2^2$ and
  $2^3$ will always have the value $0$.
\item \texttt{10}.  This tag is used for various \emph{immediate}
  objects.  In particular for \emph{characters}, but also a
  distinguished bit pattern that is used to initialize \emph{unbound}
  variables and slots.  On 64-bit platforms \emph{single-floats} are
  represented as immediate values. 
\item \texttt{11}.  This tag is used for all heap-allocated objects
  other than \texttt{cons} cells.
\end{itemize}

\section{Immediate objects}

Immediate objects are all objects with \texttt{10} in the lower two
bits.  Two more bits are used to distinguish between different kinds
of immediate objects, giving the following four low bits:

\begin{itemize}
\item \texttt{0010}.  This tag is used for Unicode characters.  When
  shifted four positions to the right, the value gives the Unicode
  code point. 
\item \texttt{0110}.  This tag is used short floats.  
\item \texttt{1010}.  This tag is used for single floats (64-bit
  platforms only). 
\item \texttt{1110}.  This tag is used for various unique immediate
  objects, in particular for the object that is stored in various
  locations to indicate that the location is \emph{unbound}. 
\end{itemize}

\subsection{Characters}

As indicated above, the low four bits of a character have the value
\texttt{0010}, and the corresponding Unicode code point is obtained by
shifting the value of the character four bits to the right. 

We currently do not plan to supply a module for Unicode support.
Instead we are relying on the support available in the Unicode library
by Edi Weitz.

\subsection{Short floats}

A short float is an immediate floating-point value that does not fall
into any of the IEEE floating-point categories.  On a 32-bit platform,
it is similar to a single-precision IEEE floating-point value, with
the difference that four fewer bits are used for the mantissa.  On a
65-bit platform, it is similar to a double-precision IEEE
floating-point value, again with the difference that four fewer bits
are used for the mantissa.  Notice that on a 64-bit platform, a short
float is more precise than a single float. 

\subsection{Single floats}

A single float is an IEEE single-precision (32-bit) floating-point
value stored in the most significant half of a 64-bit word.  The least
significant half of the word is unused except for the tag bits. 

\subsection{Unbound}
\label{data-representation-unbound}

The unbound object is represented as a word in which every bit except
the least significant one is equal to 1.  As a signed machine integer,
this value is equal to $-2$.  The unbound immediate object is used
as the value of special variables and of slots in arrays and instances
of (subclasses of) \texttt{standard-object}.  It is \emph{not} used to
indicate that a \emph{function name} is unbound.  For that purpose, a
special function that signals an error is used instead. 

\section{Representation of \texttt{cons} cells}

A \texttt{cons} cell is represented as two consecutive machine
words aligned on a double-word boundary.

\section{Representation of other heap objects}

All heap objects other than \texttt{cons} cells are represented in (at
least) two parts, a \emph{header object} and a \emph{contents vector}.
The header object always consists of two consecutive words aligned on
a double-word boundary (just like \texttt{cons} cells).  The first
word always contains a tagged pointer to a \emph{class} object (which
is another heap object).  The second word contains a \emph{raw
  pointer} to the first word of the contents vector.  In most cases,
the only place that can contain a raw pointer to a contents vector is
the second word of a header object.  However, in some cases (for
performance reasons) a raw pointer to a contents vector can be the
value of some slot.  In this case, we say that we have a
\emph{disembodied} pointer to a contents vector.  An object containing
such a disembodied pointer must be known to the garbage collector, and
the object must also contain (directly or indirectly) an ordinary
tagged \cl{} pointer to a header object containing that contents
vector.%
\footnote{The reason for this restriction is that it must always be possible
  to find the header object of a contents vector so as to determine
  the class of which the object is an instance.  If we could have
  an object X that contained only the disembodied raw pointer to the
  contents vector, then it may be the case that the pointer to the
  header object is no longer referenced by any other object, whereas
  the disembodied pointer in X is still reachable.  Then it would be
  impossible to determine the class of the object.}

In the following sections we give the details of the representation
for all possible heap objects.

\section{Instances of subclasses of \texttt{standard-object}}

Subclasses of \texttt{standard-object} must allow for must allow for
the class of an instance to be changed to some other class
(\texttt{change-class}) and for the definition of the class of the
instance to be modified.  Changing the class of an instance is fairly
straightforward because the instance is then passed as an argument and
the slots of the instance can be updated as appropriate.

Allowing for the class of an instance to be redefined is significantly
more complicated.  The main reason for that is that when a class is
redefined, the existing instances must be updated.  The standard
specifically allows for these updates to be delayed and not happen as
a direct result of the class redefinition.  They must happen before an
attempt is made to access some slot (or some other information about
the slots) of the instance.  It is undesirable to make the all
instances directly accessible from the class, because such a solution
would waste space and would have to make sure that memory leaks are
avoided.  We must thus take into account the presence of
\emph{obsolete instance} in the system, i.e., instances that must be
\emph{updated} at some later point in time. 

The solution is to store some kind of \emph{version} information in
the contents vector so that when an attempt is made to access an
obsolete instance, the instance can first be updated to correspond to
the current definition of its class.  This version information must
allow the system to determine whether any slots have been added or
removed since the instance was created.  Furthermore, if the garbage
collector traces an obsolete instance, then it must either first
update it, or the version information must allow the garbage collector
to trace the obsolete version of the instance.  Our solution allows
both.  We simply store a reference to the \emph{list of effective
  slots} that the class of the instance defined when the instance was
created.  This reference is stored as the first word of the contents
vector.  

This solution makes it possible to determine the layout of the
contents vector of an obsolete instance, so that it can be traced by
the garbage collector when necessary.  Furthermore, it is easy to
determine whether some instance is obsolete, simply by comparing the
list of effective slots in its \emph{class} (as contained in the first
word of the header object) to the list of effective slots stored in
the contents vector.  This comparison is fast, because it can be done
using \texttt{eq}.  Clearly, this solution also allows the system to
determine which slots have been added and which slots have been
removed since the instance was created. 

\section{Instance of built-in classes}

Contrary to instances of subclasses of \texttt{standard-object}, we do
not allow for a built-in class to change after it has been
instantiated.  The class definition might change as long as there are
no instances, but the consequences are undefined if a built-in class
is changed after it has been instantiated.

As a consequence, it is not necessary to keep version information in
the contents vector of an instance of a built-in class, and it is not
necessary to check whether the instance is obsolete before accessing
its contents.  

\subsection{Instances of \texttt{sequence}}

The system class \texttt{sequence} can not be directly instantiated.
Instead, it serves as a superclass for the classes \texttt{list} and
\texttt{vector}.  

The \hs{} is a bit contradictory here, because
in some places it says that \texttt{list} and \texttt{vector}
represent an exhaustive partition of \texttt{sequence}%
\footnote{See for instance section 17.1}
but in other places it explicitly allows for other subtypes of
\texttt{sequence}.%
\footnote{See the definition of the system class \texttt{sequence}.}
The general consensus seems to be that other subtypes are allowed. 


\subsection{Arrays}
\label{sec-data-representation-arrays}

An array being a heap object, it is represented as a two-word header
object and a contents vector.  The first word of the contents vector
contains a proper list of dimensions.  The length of the list is the
\emph{rank} of the array.

If the array is \emph{simple}, the initial word is directly followed
by the elements of the array.  If the array is \emph{not} simple, then
it is either displaced to another array, or it has a fill pointer, or
both.  If it has a fill pointer, then it is stored in the second word
of the contents vector.  Finally, if the array is displaced to another
array, the contents vector contains two words with the array to which
this one is displaced, and the displacement offset.  If the array is
not displaced, then the elements of the array follow.  The size of the
contents vector is rounded up to the nearest multiple of a word. 

The exact class of the array differs according to whether the array is
simple, has a fill pointer, or is displaced. 

All arrays are \emph{adjustable} thanks to the split representation
with a header object and a contents vector.  Adjusting the array
typically requires allocating a new contents vector. 

The element type of the array is determined by the exact class of the
array. 

We suggest providing specialized arrays for the following data types:

\begin{itemize}
\item \texttt{double-float}
\item \texttt{single-float}
\item \texttt{(unsigned-byte 64)}.
\item \texttt{(signed-byte 64)}.
\item \texttt{(unsigned-byte 32)}.
\item \texttt{(signed-byte 32)}.  
\item \texttt{(unsigned-byte 8)}, used for code, interface with the
  operating system, etc. 
\item \texttt{character} (i.e., strings) as required by the \hs{}.
\item \texttt{bit}, as required by the \hs{}.
\end{itemize}

Since the element type determines where an element is located and how
to access it, \texttt{aref} and \texttt{(setf aref)} are \emph{generic
  functions} that specialize on the type of the array. 

\subsubsection{System class \texttt{vector}}

A vector is a one-dimensional array.  As such, a vector has a contents
vector where the first word is a proper list of a single element,
namely the \emph{length} of the vector represented as a fixnum. 

The remaining words of the contents vector contain an optional fill
pointer, and then either the elements of the vector or displacement
information as indicated above. 

\subsubsection{System class \texttt{string}}

A string is a subtype of \texttt{array}.  Tentatively, we think that
there is no need to optimize strings that contain only characters that
could be represented in a single byte.  Thus the contents vector of a
\emph{simple} string is represented as follows:

\begin{itemize}
\item A list of a single element corresponding to the \emph{length} of
  the string. 
\item A number of consecutive words, each holding a tagged immediate
  object representing a Unicode character.
\end{itemize}

\subsection{Symbols}

A symbol is represented with a two-word header (as usual) and a
contents vector of three consecutive words.  The tree words contain:

\begin{enumerate}
\item The \emph{name} of the symbol.  The value of this slot is a
  string.
\item The \emph{package} of the symbol.  The value of this slot is a
  package or \texttt{NIL} if this symbol does not have a package.
\item The \emph{plist} of the symbol.  The value of this slot is a
  property list.
\end{enumerate}

Notice that the symbol does not contain its \emph{value} as a global
variable, nor does it contain its definition as a \emph{function} in
the global environment.  Instead, this information is contained in an
explicit \emph{global environment} object.

\subsection{Packages}

A package is represented with a two-word header (as usual) and a
contents vector of 7 words:

\begin{enumerate}
\item The \emph{name} of the package.  The value of this slot is a
  string.
\item The \emph{nicknames} of the package.  The value of this slot is
  a list of strings. 
\item The \emph{use list} of the package.  The value of this slot is a
  proper list of packages that are used by this package. 
\item The \emph{used-by list} of the package.  The value of this slot
is a proper list of packages that use this package. 
\item The \emph{external symbols} of the package.  The value of this
  slot is a proper list of symbols that are both present in and
  exported from this package.
\item The \emph{internal symbols} of the package.  The value of this
  slot is a proper list of symbols that are present in the package but
  that are not exported.
\item The \emph{shadowing symbols} of the package.  The value of this
  slot is a proper list of symbols. 
\end{enumerate}

\subsection{Hash tables}

\subsection{Streams}

\subsection{Functions}

The \emph{class} of a function is a subclass of \texttt{function}.

In order to obtain reasonable performance, we represent functions in a
somewhat complex way, as illustrated by
\refFig{fig-function-representation}. 

\begin{figure}
\begin{center}
\inputfig{fig-function-representation.pdf_t}
\end{center}
\caption{\label{fig-function-representation}
Representation of functions.}
\end{figure}

A function is represented as a two-word header (as usual) and a
contents vector with five slots in addition to the inherited slots.

\begin{enumerate}
\item The \emph{code object} for this function (see below).
\item An \emph{environment}, which is the local environment in which
  the function was defined. If the function does not close over any
  variables, then the value of this slot is \texttt{NIL}.
\item The \emph{linkage vector} for this function, which is a raw
  disembodied contents vector pointer pointing to the contents vector
  of the linkage vector object for this function.  The ordinary \cl{}
  object containing this contents vector is referred to from the code
  object in the slot described previously.  The garbage collector must
  update the value of this slot if it decides to move the contents
  vector of the linkage vector object.
\item A \emph{raw address} to the entry point for the function that
  also checks argument count.  The garbage collector must update the
  value of this slot if it decides to move the contents vector of the
  code vector.
\item A \emph{raw address} to the entry point for the function that
  omits the checks for argument count.  The garbage collector must
  update the value of this slot if it decides to move the contents
  vector of the code vector.
\end{enumerate}

Since raw addresses are word aligned, they show up as \texttt{fixnum}s
when inspected by tools that are unaware of their special
signification.  

A \emph{code object} is represented as a two-word header (as usual)
and a contents vector containing (in addition to inherited slots):

\begin{enumerate}
\item The \emph{linkage vector} which is an ordinary \cl{} simple
  vector.
\item The \emph{code vector} which is an ordinary \cl{} vector with
  elements of type \texttt{(unsigned-byte 8)} containing the machine
  instructions for one or more functions.
\item Correspondence between values of the program counter
  (represented as offsets into the byte vector of instructions) and
  source code locations.  
\item Correspondence between values of the program counter and the
  contents of the local environment.  This information is used by the
  garbage collector and by the debugger. 
\end{enumerate}

By having several functions share the same linkage vector, we can
simplify calls between functions in the same compilation unit, because
the caller and the callee would then share the same linkage vector.
In contrast, a call from a function in one compilation unit to a
function in a different compilation unit involves (in the most general
case) accessing the linkage vector of the callee and storing it in a
register.

For a \emph{generic function}, the description of the slots above
applies to the \emph{discriminating function} of the generic function.
In addition to these slots, a generic function also contains other
slots holding the list of its methods, and other information.

When a function is called, there are several possible situations that
can occur:

\begin{itemize}
\item The most general case is when an object of unknown type is given
  as an argument to \texttt{funcall}.  Then no optimization is
  attempted, and \texttt{funcall} is responsible for determining
  whether the object is a function, a name that designates a function,
  or an object that does not designate a function in which an error is
  signaled. 
\item When it can be determined statically that the object called is a
  function (i.e. its class is a subclass of the class
  \texttt{function}), but nothing else is know about it, then an
  \emph{external call} is made.  Such a call consists of copying the
  contents of the \emph{static environment} slot and the \emph{linkage
    vector} slot to the predetermined place specific to the backend,
  and then to issue a \emph{call} instruction (or equivalent) to the
  address indicated by the most general \emph{entry point} slot of the
  function object.  When the call is to a \emph{global function}, then
  the linkage vector contains a \texttt{cons} cell whose \texttt{car}
  is guaranteed to contain a function object, so this situation is
  applicable in this case. 
\item When it can not only be determined statically that the object
  being called is a function, but also that the number of arguments
  that will be passed to it is acceptable, then the alternative entry
  point can be used.  This is often the case when the function being
  called is a \emph{global system function}. 
\item When it can be determined statically that the object being
  called is a function object in the \emph{same compilation unit} as
  the caller, then we can make an \emph{internal call}.  If both the
  caller and the callee are \emph{global functions} (so that the
  static environment is NIL for both), then it suffices to issue a
  \texttt{call} instruction (or equivalent) to a relative address that
  can be determined statically.  The linkage vector does not need to
  be passed because it is shared between the caller and the callee.
  The relative address can be chosen so as to avoid type checking of
  arguments with known types. 
\item Even when it can not be determined statically that the object
  being called is a function, there are some situations that can be
  optimized.  For instance, when it is likely that the object is going
  to be called multiple times with the same number of arguments, then
  it may pay off to start by checking that the object is a function,
  then accessing its \emph{lambda list} to determine that the number
  of arguments is acceptable, and then cache the static environment,
  the linkage vector, and the alternative entry point in \emph{local
    registers}.   A typical use for this situation is in the
  \emph{sequence functions} for the keyword arguments \texttt{:test},
  \texttt{:test-not}, and \texttt{:key}.  
\end{itemize}
