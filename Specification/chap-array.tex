\chapter{Array}
\label{chap-array}

There are a few things that can be done in a portable module for
arrays. 

Clearly, \texttt{array-dimensions} can be defined in terms of
\texttt{array-rank} and \texttt{array-dimension}.  On the other hand,
it can also be done the other way around, i.e., \texttt{array-rank}
can be computed as the length of the list returned by
\texttt{array-dimensions}, and \texttt{array-dimension} can be defined
using the list returned by \texttt{array-dimensions} and
\texttt{elt}, or \texttt{nth}.  Whether one or the other is chosen
depends on how arrays are represented.  In \sysname{} the list of
dimensions is already stored in the array
\seesec{sec-data-representation-arrays}, so the second solution is
better, and it also avoids consing. 

\texttt{array-total-size} can be defined in terms of
\texttt{array-rank} and \texttt{array-dimension}, or in terms of and
\texttt{array-dimensions} depending on what solution was chosen above.

Furthermore, \texttt{aref} and
\texttt{(setf aref)} can be defined in terms of
\texttt{row-major-aref} and \texttt{(setf row-major-aref)}. 

It might be worthwhile defining a compiler macro for \texttt{aref}. 

At the moment, there is some experimental code for arrays in the
directory \texttt{Code/Array}, but it is probably best to do it over
from scratch. 

