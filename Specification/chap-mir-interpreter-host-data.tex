\chapter{MIR interpreter with host data}

This backend is an interpreter for the medium-level intermediate
representation (MIR) presented in \refChap{chap-mir}.  All objects
such as symbols, numbers, packages, etc. are host objects.  There are
three kinds of functions:

\begin{itemize}
\item Primitive functions.  These are essentially host functions that
  have been wrapped so that they obey the function-call protocol of
  the backend. 
\item Interpreted ordinary functions.  An interpreted ordinary
  function contains a lexical environment and a MIR program
  representing a function.  
\item Interpreted generic functions.  Such a function contains an
  interpreted ordinary function for the discriminating function, and
  methods that also contain ordinary interpreted functions. 
\end{itemize}

Primitive functions can be arbitrarily complex, allowing us to
gradually test \sysname{} functions by incrementally moving code from
complex primitive functions to collections of interpreted functions.
So for instance, the functions \texttt{read} and \texttt{print} will
initially be primitive functions, and then gradually replaced by
target implementations as collections of interpreted functions.

Macros such as global standard macros and macros introduced by
\texttt{macrolet} have macro functions represented as primitive
functions.  Only macros introduced by executing compiled calls to
\texttt{(setf macro-function)} will have macro functions represented
as interpreted ordinary functions.  This way, we can pre-load the
global environment with standard macros, allowing us to turn source
code into abstract syntax trees and then to MIR code. 
