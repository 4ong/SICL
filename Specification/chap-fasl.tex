\chapter{Compiled files}
\label{chap-compiled-files}

When a compiled file (a so called \emph{fasl} file) is loaded, some
arbitrary code may have to be executed.  For that reason, we think
that it is best to make the \emph{fasl} file format as simple as
possible so as to simplify the \emph{fasl} loader.

We think a \emph{fasl} file should consist of a one-word \emph{version
  identifier} which also contains the \emph{magic number} of the
file.  Starting immediately after the first word is \emph{executable
  native code} that has been compiled to be \emph{position
  independent}.  Loading a \emph{fasl} file is then just a matter of:

\begin{enumerate}
\item Calling \texttt{mmap} to map the file to some arbitrary place in
  the address space.
\item Verifying the magic number and the version to make sure it is
  compatible with the current version of the system. 
\item Using a \texttt{call} instruction (or equivalent) to call the
  code of the file as a subroutine.
\item Calling \texttt{munmap} to unmap the file from the address
  space. 
\end{enumerate}

Since a \emph{fasl} file can contain arbitrary code to be executed at
load time, the entire system can be bootstrapped by loading a sequence
of \emph{fasl} files. 

Most ``normal'' \emph{fasl} files will need to call some functions in
order to accomplish the processing that must happen at load time.
They do this by executing the following steps:

\begin{enumerate}
\item Call \texttt{make-string} to allocate a string to contain the
  name of a package.
\item Call \texttt{(setf schar)} multiple times to set the characters
  of the string to the name of the desired package.
\item Call \texttt{find-package} to find a package with that name.
\item Call \texttt{make-string} again to allocate a string to contain
  the name of a symbol.
\item Call \texttt{(setf schar)} multiple times to set the characters
  of the string to the name of the desired symbol.
\item Call \texttt{intern} with the name of the symbol and the package
  in order to obtain a symbol.
\item Call \texttt{fdefinition} with the symbol to obtain the desired
  function. 
\end{enumerate}

If the name of the function that is wanted is of the form
\texttt{(setf \emph{name})}, then the function named \texttt{CONS}
first has to be found with the method indicated above.  Then the 
symbols \texttt{SETF}, and \texttt{NIL}, must also be found using the
same method as above.  The function \texttt{CONS} must then be called
in order to build the list that gives the name of the desired
function.  Finally, \texttt{fdefinition} is called with that list. 

Clearly, the functions \texttt{make-string}, \texttt{(setf schar)},
\texttt{find-package}, and \texttt{intern} must be found through a
different mechanism than the normal one.  For that reason, these
functions are referred to from memory locations with fixed addresses
in the system.  


