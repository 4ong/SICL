\chapter{Compiled files}
\label{chap-compiled-files}

In order to simplify \sysname{} as much as possible, we will use the
external format of the \cleavir{} abstract syntax tree as our so
called \emph{\textsc{fasl}} format.  The \cleavir{} compiler already contains
code that makes it possible to read and write abstract syntax trees,
so with this decision, there is no need to design an additional file
format.

The external format for abstract syntax trees can be read using the
\commonlisp{} standard \texttt{read} function with a single additional
reader macro, also provided by \cleavir{}.  Since the \commonlisp{}
reader will be present in the initial executable \sysname{} system,
there is no special code needed in order to read a \textsc{fasl} file.

Furthermore, the compiler will also be present in the initial initial
executable \sysname{} system, so the code for converting an abstract
syntax tree into native code is also present.

The \commonlisp{} standard requires compiled files to be at least
\emph{minimally compiled}, and the abstract syntax tree format
fulfills the requirement for minimal compilation.

The main downside of using this format for \textsc{fasl} files is decreased
performance compared to a format containing native code.  However,
loading \textsc{fasl} files is typically only done during the development phase
of some software, and almost never at run-time.  The additional delay
required when loading an abstract syntax tree as a result of
converting it to intermediate code and then to native code is likely
to be barely noticeable during development.
