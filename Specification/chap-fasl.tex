\chapter{Compiled files}
\label{chap-compiled-files}

When a compiled file (a so called \emph{fasl} file) is loaded, some
arbitrary code may have to be executed.  For that reason, we think
that it is best to make the \emph{fasl} file format as simple as
possible so as to simplify the \emph{fasl} loader.

We think a \emph{fasl} file should consist of a one-word \emph{version
  identifier} which also contains the \emph{magic number} of the
file.  Starting immediately after the first word is \emph{executable
  native code} that has been compiled to be \emph{position
  independent}.  Loading a \emph{fasl} file is then just a matter of:

\begin{enumerate}
\item Calling \texttt{mmap} to map the file to some arbitrary place in
  the address space.
\item Verifying the magic number and the version to make sure it is
  compatible with the current version of the system. 
\item Using a \texttt{call} instruction (or equivalent) to call the
  code of the file as a subroutine.
\item Calling \texttt{munmap} to unmap the file from the address
  space. 
\end{enumerate}

The executable code in the \emph{fasl} file can be seen as a \commonlisp{}
function that takes 3 functions as arguments, namely
\texttt{copy-string}, \texttt{intern}, and \texttt{fstorage}, where
\texttt{copy-string} is a function of a single argument that returns a
copy of a string passed as an argument, and \texttt{fstorage} is a
function of a single argument (a function name) that returns a
\texttt{cons} cell containing the global definition of the function
with that name.

In order to create objects and modify the environment, the code in the
\emph{fasl} file executes the following steps:

\begin{enumerate}
\item Call \texttt{copy-string}, passing it a string object local to
  the \emph{fasl} file to create a string to contain the
  name of a package.
\item Call \texttt{copy-string} again to create a string to contain
  the name of a symbol.
\item Call \texttt{intern} with the name of the symbol and the name of
  the package in order to obtain a symbol.
\item Call \texttt{fstorage} with the symbol to obtain the storage
  cell for the desired function.
\item Take the \texttt{car} of that cell to get the desired function.
\item Use the function to create the desired object.
\end{enumerate}

If the name of the function that is wanted is of the form
\texttt{(setf \emph{name})}, then the function named \texttt{cons}
first has to be found with the method indicated above.  Then the 
symbols \texttt{setf}, and \texttt{NIL}, must also be found using the
same method as above.  The function \texttt{cons} must then be called
in order to build the list that gives the name of the desired
function.  Finally, \texttt{fstorage} is called with that list. 
