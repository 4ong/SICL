\chapter{x86-64}
\label{chapter-backend-x86-64}

\section{Register usage}

The standard calling conventions defined by the vendors, and used by
languages such as \clanguage{} use the registers as follows:

\begin{tabular}{|l|l|l|}
\hline
Name & Used for & Saved by\\
\hline
\hline
RAX & First return value & Caller\\
RBX & Optional base pointer & Callee\\
RCX & Fourth argument & Caller \\
RDX & Third argument, second return value & Caller\\
RSP & Stack pointer &\\
RBP & Frame pointer & Callee\\
RSI & Second argument & Caller\\
RDI & First argument & Caller\\
R8 & Fifth argument & Caller\\
R9 & Sixth argument & Caller\\
R10 & Temporary, static chain pointer & Caller\\
R11 & Temporary & Caller\\
R12 & Temporary & Callee\\
R13 & Temporary & Callee\\
R14 & Temporary & Callee\\
R15 & Temporary & Callee\\
\hline
\end{tabular}

We mostly respect this standard, and define the register allocation as
follows:

\begin{tabular}{|l|l|l|}
\hline
Name & Used for & Saved by\\
\hline
\hline
RAX & First return value & Caller\\
RBX & Dynamic environment & Callee\\
RCX & Fourth argument, third return value & Caller \\
RDX & Third argument, second return value & Caller\\
RSP & Stack pointer &\\
RBP & Frame pointer & Caller\\
RSI & Second argument, fourth return value & Caller\\
RDI & First argument, value count & Caller\\
R8  & Fifth argument & Caller\\
R9  &  Argument count, return value pointer& Caller\\
R10 & Static env. argument & Caller\\
R11 & Scratch & Caller\\
R12 & Register variable & Callee\\
R13 & Register variable & Callee\\
R14 & Register variable & Callee\\
R15 & Register variable & Callee\\
\hline
\end{tabular}

\section{Representation of function objects}

\begin{itemize}
\item A static environment.
\item The entry point of the function as a raw machine address.  Since
  entry points are word aligned, this value looks like a fixnum.
\end{itemize}

\section{Calling conventions}

\refFig{fig-x86-64-stack-frame} shows the layout of a stack frame.

\begin{figure}
\begin{center}
\inputfig{fig-x86-64-stack-frame.pdf_t}
\end{center}
\caption{\label{fig-x86-64-stack-frame}
Stack frame for the x86-64 backend.}
\end{figure}

Normal call to external function, passing at most $5$ arguments:

\begin{enumerate}
\item Compute the callee function object and the arguments into
  temporary locations.
\item Store the arguments in RDI, RSI, RDX, RCX, and R8.
\item Store the argument count in R9 as a fixnum.
\item Load the static environment of the callee from the callee
  function object into R10.
\item Push the value of RBP on the stack.
\item Copy the value of RSP into RBP, establishing the
  (empty) stack frame for the callee.
\item Load the entry point address of the callee from the callee
  function object into an available scratch register, typically RAX.
\item Use the CALL instruction with that register as an argument,
  pushing the return address on the stack and transferring control to
  the callee.
\end{enumerate}

Normal call to external function, passing at more than $5$ arguments:

\begin{enumerate}
\item Compute the callee function object and the arguments into
  temporary locations.
\item Subtract $8(N - 3)$ from RSP, where $N$ is the number of
  arguments to pass, thus leaving room in the callee stack frame for
  the $N - 5$ arguments, the return address, and the caller RBP.
\item Store the first $5$ arguments in RDI, RSI, RDX, RCX, and R8.
\item Store the remaining arguments in [RSP$ + 0$], [RSP$ + 8$],
  $\ldots$, [RSP$ + 8(N - 6)$] in that order, so that the sixth
  argument is on top of the stack.
\item Store the argument count in R9 as a fixnum.
\item Load the static environment of the callee from the callee
  function object into R10.
\item Store the value of RBP into [RSP + 8(N - 4)]
\item Copy the value of RSP$ + 8(N - 4)$ into RBP, establishing the
  stack frame for the callee.  The instruction LEA can be used for
  this purpose.
\item Load the entry point address of the callee from the callee
  function object into an available scratch register, typically RAX.
\item Use the CALL instruction with that register as an argument,
  pushing the return address on the stack and transferring control to
  the callee.
\end{enumerate}

\refFig{fig-x86-64-stack-frame-at-entry} shows the layout of the stack
upon entry to a function when more than $5$ arguments are passed.
Notice that the return address is not in its final place, and the
final place for the return address is marked ``unused'' in
\refFig{fig-x86-64-stack-frame-at-entry}.

\begin{figure}
\begin{center}
\inputfig{fig-x86-64-stack-frame-at-entry.pdf_t}
\end{center}
\caption{\label{fig-x86-64-stack-frame-at-entry}
Stack frame at entry with more than 5 arguments.}
\end{figure}

Tail call to external function, passing at most $5$ arguments:

\begin{enumerate}
\item Compute the callee function object and the arguments into
  temporary locations.
\item Store the arguments in RDI, RSI, RDX, RCX, and R8.
\item Store the argument count in R9 as a fixnum.
\item Load the static environment of the callee from the callee
  function object into R10.
\item Copy the value of RBP$ - 8$ to RSP, establishing the stack frame
  for the callee, containing only the return address.  The LEA
  instruction can be used for this purpose.
\item Load the entry point address of the callee from the callee
  function object into an available scratch register, typically RAX.
\item Use the JMP instruction with that register as an argument,
  transferring control to the callee.
\end{enumerate}

Tail call to external function, passing at more than $5$ arguments:%
\fixme{Determine the protocol.}

For \emph{internal calls} there is greater freedom, because the caller
and the callee were compiled simultaneously.  In particular, the
caller might copy some arbitrary \emph{prefix} of the code of the
callee in order to optimize it in the context of the caller.  This
prefix contains argument count checking and type checking of
arguments.  The address to use for the call is computed statically as
an offset from the current program counter, so that a CALL instruction
with a fixed relative address can be used.  Furthermore, the caller
might be able to avoid loading the static environment if it is known
that the callee uses the same static environment as the caller.

Upon function entry, when the entry point is the one used for ordinary
calls with more than $5$ arguments, the callee must pop the return
address off the top of the stack and store it in its final location.
This can be done with a single POP instruction, using [RBP$ - 8$] as
the destination.   For all other entry points, the return address is
already in the right place.

Return from callee to caller with no values:

\begin{enumerate}
\item Store NIL in RAX.
\item Store $0$ in RDI, represented as a fixnum.
\item Store the value of RBP$ - 8$ in RSP so that the stack frame
  contains only the return address.  To accomplish this effect, the
  callee can use the LEA instruction.
\item Return to the caller by executing the RET instruction.
\end{enumerate}

Return from callee to caller with a $1 - 4$ values:

\begin{enumerate}
\item Store the values to return in RAX, RDX, RCX, RSI.
\item Store the number of values in RDI, represented as a fixnum.
\item Store the value of RBP$ - 8$ in RSP so that the stack frame
  contains only the return address.  To accomplish this effect, the
  callee can use the LEA instruction.
\item Return to the caller by executing the RET instruction.
\end{enumerate}

Return from callee to caller with more than $4$ values:

\begin{enumerate}
\item Store the first $4$ values to return in RAX, RDX, RCX, RSI.
\item Access the \emph{thread} object in the dynamic environment in
  order to obtain an address to be used for the remaining return
  values and put that address in R8.
\item Store the number of values in RDI, represented as a fixnum.
\item Store the value of RBP$ - 8$ in RSP so that the stack frame
  contains only the return address.  To accomplish this effect, the
  callee can use the LEA instruction.
\item Return to the caller by executing the RET instruction.
\end{enumerate}

\section{Parsing keyword arguments}

When the callee accepts keyword arguments, it is convenient to have
all the arguments in a properly-ordered sequence somewhere in memory.
We obtain this sequence by pushing the register arguments to the stack
in reverse order, so that the first argument is at the top of the
stack.  When more than $5$ arguments are passed, the program counter
is popped off the top of the stack, thereby moving it to its final
destination before the register arguments are pushed.

%%  LocalWords:  callee
