\chapter{x86-64}
\label{chapter-backend-x86-64}

The standard calling conventions use the registers as follows

\begin{tabular}{|l|l|l|}
\hline
Name & Used for & Saved by\\
\hline
\hline
RAX & First return value & Caller\\
RBX & Optional base pointer & Callee\\
RCX & Fourth argument & Caller \\
RDX & Third argument, second return value & Caller\\
RSP & Stack pointer &\\
RBP & Frame pointer & Callee\\
RSI & Second argument & Caller\\
RDI & First argument & Caller\\
R8 & Fifth argument & Caller\\
R9 & Sixth argument & Caller\\
R10 & Temporary, static chain pointer & Caller\\
R11 & Temporary & Caller\\
R12 & Temporary & Callee\\
R13 & Temporary & Callee\\
R14 & Temporary & Callee\\
R15 & Temporary & Callee\\
\hline
\end{tabular}

We mostly respect this standard, and allocate as follows:

\begin{tabular}{|l|l|l|}
\hline
Name & Used for & Saved by\\
\hline
\hline
RAX & First return value & Caller\\
RBX & Dynamic environment & Callee\\
RCX & Fourth argument, third return value & Caller \\
RDX & Third argument, second return value & Caller\\
RSP & Stack pointer &\\
RBP & Frame pointer & Caller\\
RSI & Second argument, fourth return value & Caller\\
RDI & First argument, value count & Caller\\
R8 &  Argument count, return value pointer& Caller\\
R9 & Linkage vector & Caller\\
R10 & Static env. argument & Caller\\
R11 & Scratch & Caller\\
R12 & Register variable & Callee\\
R13 & Register variable & Callee\\
R14 & Register variable & Callee\\
R15 & Register variable & Callee\\
\hline
\end{tabular}

\refFig{fig-x86-64-stack-frame} shows the layout of a stack frame. 

\begin{figure}
\begin{center}
\inputfig{fig-x86-64-stack-frame.pdf_t}
\end{center}
\caption{\label{fig-x86-64-stack-frame}
Stack frame for the x86-64 backend.}
\end{figure}
