\chapter{x86-64}
\label{chapter-backend-x86-64}

\section{Register usage}

The standard calling conventions defined by the vendors, and used by
languages such as \clanguage{} use the registers as follows:

\begin{tabular}{|l|l|l|}
\hline
Name & Used for & Saved by\\
\hline
\hline
RAX & First return value & Caller\\
RBX & Optional base pointer & Callee\\
RCX & Fourth argument & Caller \\
RDX & Third argument, second return value & Caller\\
RSP & Stack pointer &\\
RBP & Frame pointer & Callee\\
RSI & Second argument & Caller\\
RDI & First argument & Caller\\
R8 & Fifth argument & Caller\\
R9 & Sixth argument & Caller\\
R10 & Temporary, static chain pointer & Caller\\
R11 & Temporary & Caller\\
R12 & Temporary & Callee\\
R13 & Temporary & Callee\\
R14 & Temporary & Callee\\
R15 & Temporary & Callee\\
\hline
\end{tabular}

We mostly respect this standard, and define the register allocation as
follows:

\begin{tabular}{|l|l|l|}
\hline
Name & Used for & Saved by\\
\hline
\hline
RAX & First return value & Caller\\
RBX & Dynamic environment & Callee\\
RCX & Fourth argument, third return value & Caller \\
RDX & Third argument, second return value & Caller\\
RSP & Stack pointer &\\
RBP & Frame pointer & Caller\\
RSI & Second argument, fourth return value & Caller\\
RDI & First argument, value count & Caller\\
R8  & Fifth argument & Caller\\
R9  &  Argument count, return value pointer& Caller\\
R10 & Static env. argument & Caller\\
R11 & Scratch & Caller\\
R12 & Register variable & Callee\\
R13 & Register variable & Callee\\
R14 & Register variable & Callee\\
R15 & Register variable & Callee\\
\hline
\end{tabular}

\section{Representation of function objects}

\begin{itemize}
\item A static environment.
\item The entry point of the function as a raw machine address.  Since
  entry points are word aligned, this value looks like a fixnum.
\end{itemize}

\section{Calling conventions}

\refFig{fig-x86-64-stack-frame} shows the layout of a stack frame.

\begin{figure}
\begin{center}
\inputfig{fig-x86-64-stack-frame.pdf_t}
\end{center}
\caption{\label{fig-x86-64-stack-frame}
Stack frame for the x86-64 backend.}
\end{figure}

Normal call to external function, passing at most $5$ arguments:

\begin{enumerate}
\item Compute the callee function object and the arguments into
  temporary locations.
\item Store the arguments in RDI, RSI, RDX, RCX, and R8.
\item Store the argument count in R9 as a fixnum.
\item Load the static environment of the callee from the callee
  function object into R10.
\item Push the value of \texttt{RBP} on the stack.
\item Copy the value of RSP into \texttt{RBP}, establishing the
  (empty) stack frame for the callee.
\item Load the entry point address of the callee from the callee
  function object into an available scratch register, typically RAX.
\item Use the CALL instruction with that register as an argument,
  pushing the return address on the stack and transferring control to
  the callee.
\end{enumerate}

Normal call to external function, passing at more than $5$ arguments:

\begin{enumerate}
\item Compute the callee function object and the arguments into
  temporary locations.
\item Subtract $8(N - 3)$ from RSP, where $N$ is the number of
  arguments to pass, thus leaving room in the callee stack frame for
  the $N - 5$ arguments, the return address, and the caller \texttt{RBP}.
\item Store the first $5$ arguments in RDI, RSI, RDX, RCX, and R8.
\item Store the remaining arguments in [RSP$ + 0$], [RSP$ + 8$],
  $\ldots$, [RSP$ + 8(N - 6)$] in that order, so that the sixth
  argument is on top of the stack.
\item Store the argument count in R9 as a fixnum.
\item Load the static environment of the callee from the callee
  function object into R10.
\item Store the value of \texttt{RBP} into [RSP + 8(N - 4)]
\item Copy the value of RSP$ + 8(N - 4)$ into \texttt{RBP}, establishing the
  stack frame for the callee.  The instruction LEA can be used for
  this purpose.
\item Load the entry point address of the callee from the callee
  function object into an available scratch register, typically RAX.
\item Use the CALL instruction with that register as an argument,
  pushing the return address on the stack and transferring control to
  the callee.
\end{enumerate}

By using a \texttt{CALL}/\texttt{RET} pair instead of (say) the caller
storing the return address in its final place using some other method,
we make sure that the predictor for the return address of the
processor makes the right guess about the eventual address to be used.

\refFig{fig-x86-64-stack-frame-at-entry} shows the layout of the stack
upon entry to a function when more than $5$ arguments are passed.
Notice that the return address is not in its final place, and the
final place for the return address is marked ``unused'' in
\refFig{fig-x86-64-stack-frame-at-entry}.

\begin{figure}
\begin{center}
\inputfig{fig-x86-64-stack-frame-at-entry.pdf_t}
\end{center}
\caption{\label{fig-x86-64-stack-frame-at-entry}
Stack frame at entry with more than 5 arguments.}
\end{figure}

Tail call to external function, passing at most $5$ arguments:

\begin{enumerate}
\item Compute the callee function object and the arguments into
  temporary locations.
\item Store the arguments in RDI, RSI, RDX, RCX, and R8.
\item Store the argument count in R9 as a fixnum.
\item Load the static environment of the callee from the callee
  function object into R10.
\item Copy the value of \texttt{RBP}$ - 8$ to RSP, establishing the stack frame
  for the callee, containing only the return address.  The LEA
  instruction can be used for this purpose.
\item Load the entry point address of the callee from the callee
  function object into an available scratch register, typically RAX.
\item Use the JMP instruction with that register as an argument,
  transferring control to the callee.
\end{enumerate}

Tail call to external function, passing at more than $5$ arguments:%
\fixme{Determine the protocol.}

For \emph{internal calls} there is greater freedom, because the caller
and the callee were compiled simultaneously.  In particular, the
caller might copy some arbitrary \emph{prefix} of the code of the
callee in order to optimize it in the context of the caller.  This
prefix contains argument count checking and type checking of
arguments.  The address to use for the call is computed statically as
an offset from the current program counter, so that a CALL instruction
with a fixed relative address can be used.  Furthermore, the caller
might be able to avoid loading the static environment if it is known
that the callee uses the same static environment as the caller.

Upon function entry after an ordinary call, when more than $5$
arguments are passed, the callee must pop the return address off the
top of the stack and store it in its final location.  This can be done
with a single POP instruction, using [\texttt{RBP}$ - 8$] as the
destination.  When fewer than $5$ arguments are passed, the return
address is already in the right place.

Return from callee to caller with no values:

\begin{enumerate}
\item Store NIL in RAX.
\item Store $0$ in RDI, represented as a fixnum.
\item Store the value of \texttt{RBP}$ - 8$ in RSP so that the stack frame
  contains only the return address.  To accomplish this effect, the
  callee can use the LEA instruction.
\item Return to the caller by executing the RET instruction.
\end{enumerate}

Return from callee to caller with a $1 - 4$ values:

\begin{enumerate}
\item Store the values to return in RAX, RDX, RCX, RSI.
\item Store the number of values in RDI, represented as a fixnum.
\item Store the value of \texttt{RBP}$ - 8$ in RSP so that the stack frame
  contains only the return address.  To accomplish this effect, the
  callee can use the LEA instruction.
\item Return to the caller by executing the RET instruction.
\end{enumerate}

Return from callee to caller with more than $4$ values:

\begin{enumerate}
\item Store the first $4$ values to return in RAX, RDX, RCX, RSI.
\item Access the \emph{thread} object in the dynamic environment in
  order to obtain an address to be used for the remaining return
  values and put that address in R8.
\item Store the number of values in RDI, represented as a fixnum.
\item Store the value of \texttt{RBP}$ - 8$ in RSP so that the stack frame
  contains only the return address.  To accomplish this effect, the
  callee can use the LEA instruction.
\item Return to the caller by executing the RET instruction.
\end{enumerate}

\section{Use of the dynamic environment}

The dynamic environment is a simply linked sequence of entries
allocated in the stack rather than on the heap.  The following entry
types exist:

\begin{itemize}
  \item An entry representing the binding of a special variable.  This
    entry contains two fields; the symbol to be bound and the value.

  \item An entry representing a \texttt{catch} tag.  The entry
    contains a single field containing the \texttt{catch} tag.

  \item An entry representing a \texttt{block} or a \texttt{tagbody}.
    It is similar to the entry representing a \texttt{catch} tag.  The
    entry contains a single field containing a time stamp represented
    as a fixnum.  This time stamp becomes part of the static
    environment of any function nested inside the \texttt{block} or
    \texttt{tagbody} form that contains a \texttt{return-from} or a
    \texttt{go} to this form.

  \item An entry representing an \texttt{unwind-protect} form.  This
    entry contains a thunk containing the cleanup forms of the
    \texttt{unwind-protect} form.
\end{itemize}

In addition to their own fields, each entry contains a pointer to the
next entry in the sequence, and a field indicating what type of entry
it is.

Of the three types of entries above, the \texttt{catch} and the
\texttt{block}/\texttt{tagbody} entries represent exit points.  When
an exit point expires, the entry must be invalidated.  An entry
representing a \texttt{block} or \texttt{tagbody} is invalidated by
storing 0 as the value of the time stamp.  An entry repesenting a
\texttt{catch} tag is invalidated by changing its type field so that
it is a an entry representing a \texttt{block} or \texttt{tagbody} and
storing 0 in the field for the \texttt{catch} tag.

\texttt{catch} is implemented as a call to a function.  This function
establishes a \texttt{catch} tag and calls a thunk containing the body
of the \texttt{catch} form.  The \texttt{catch} tag is established by
allocating (as dynamic local data on the stack) an entry of the second
type as shown in \refFig{fig-x86-64-catch}

\begin{figure}
\begin{center}
\inputfig{fig-x86-64-catch.pdf_t}
\end{center}
\caption{\label{fig-x86-64-catch}
\texttt{catch} tag for the x86-64 backend.}
\end{figure}

\texttt{throw} searches the dynamic environment for an entry with the
right \texttt{catch} tag.  If one is found, then it must also be
valid.  The point to which control is to be transferred is stored as
the return value of the stack frame containing the \texttt{catch} tag.

A \texttt{block} form may establish an exit point.  In the most
general case, a \texttt{return-from} is executed from a function
lexically-enclosed inside the \texttt{block} with an arbitrary number
of intervening stack frames.  When this is the case, upon entry to the
\texttt{block} form, a \texttt{block}/\texttt{tagbody} entry with a
fresh time stamp is established.  When a \texttt{return-from} is
executed, the point to which control is to be transferred is known
statically.  The time stamp is also stored in a lexical variable in
the static environment of closures established inside the
\texttt{block} form that may execute a corresponding
\texttt{return-from} form.  \refFig{fig-x86-64-block-tag} shows this
situation.

\begin{figure}
\begin{center}
\inputfig{fig-x86-64-block-tag.pdf_t}
\end{center}
\caption{\label{fig-x86-64-block-tag}
\texttt{block} tag for the x86-64 backend.}
\end{figure}

\texttt{return-from} accesses the time stamp from the static
environment and then searches the dynamic environment for a
corresponding time stamp in a \texttt{block}/\texttt{tagbody} entry.
If one with the right time stamp is found, the lexical environment is
restored, and control is transferred to the statically known address.

A \texttt{tagbody} may establish several exit points.  In the most
general case, a \texttt{go} is executed from a function
lexically-enclosed inside the \texttt{tagbody} with an arbitrary
number of intervening stack frames.  When this is the case, upon entry
to the \texttt{tagbody} form, a \texttt{block}/\texttt{tagbody} entry
with a fresh time stamp is established.  When a \texttt{go} is
executed, the point to which control is to be transferred is known
statically.  The time stamp is also stored in a lexical variable in
the static environment of closures established inside the
\texttt{tagbody} form that may execute a corresponding \texttt{go}
form.  \refFig{fig-x86-64-block-tag} shows this situation.

\texttt{go} accesses the time stamp from the static environment and
then searches the dynamic environment for a corresponding time stamp
in a \texttt{block}/\texttt{tagbody} entry.  If one with the right
time stamp is found, the lexical environment is restored, and control
is transferred to the statically known address.

\section{Transfer of control to an exit point}

Whenever transfer of control to an exit point is initiated, the exit
point is first searched for.  If no valid exit point can be found, an
error is signaled.  If a valid exit point is found, the stack must
then be unwound.  First, the dynamic environment is traversed for any
intervening exit points, and they are marked as invalid as indicated
above.  Traversal stops when the stack frame of the valid exit point
is reached.  Unwinding now begins.  The dynamic environment is
traversed again and thunks in \texttt{unwind-protect} entries are
executed.  The traversal again stops when the stack frame of the valid
exit point is reached.

\begin{figure}
\begin{center}
\inputfig{fig-x86-64-unwind-protect.pdf_t}
\end{center}
\caption{\label{fig-x86-64-unwind-protect}
\texttt{unwind-protect} tag for the x86-64 backend.}
\end{figure}

\begin{figure}
\begin{center}
\inputfig{fig-x86-64-dynamic-binding.pdf_t}
\end{center}
\caption{\label{fig-x86-64-dynamic-binding}
Dynamic variable binding for the x86-64 backend.}
\end{figure}

\section{Parsing keyword arguments}

When the callee accepts keyword arguments, it is convenient to have
all the arguments in a properly-ordered sequence somewhere in memory.
We obtain this sequence by pushing the register arguments to the stack
in reverse order, so that the first argument is at the top of the
stack.  When more than $5$ arguments are passed, the program counter
is popped off the top of the stack, thereby moving it to its final
destination before the register arguments are pushed.

%%  LocalWords:  callee
