\chapter{x86-32}
\label{chap-backend-x86}

We are in the process of removing this chapter.  Some of the
information here will be transferred to
\refChap{chapter-backend-x86-64}, perhaps with modifications.

\section{Calling conventions}

The calling conventions above are optimized for functions with no
\&rest and no \&key parameters, and with parameters having dynamic
extent.  That case is very common, and the use of compiler macros
makes it even more common that what is apparent from reading source
code, because many calls to functions with \&rest and \&key parameters
can be replaced by calls to specialized functions with only required
parameters.  For this special case, the arguments supplied by a caller
are already in their ``home'' position on the stack.  When \&rest and
\&key parameters are present, a possibly fairly complex process must
traverse the arguments and establish a local environment.  Parameters
with dynamic extent can be allocated on the stack, whereas parameters
with indefinite extent must be placed in heap-allocated objects.
\seesec{backends-x86-32-static-environment}  The presence of
\&optional parameters represents an intermediate case.  In most cases
the default values of \&optional parameters are simple constants.  In
that case, the callee can simply push additional values on the stack
(provided the corresponding parameter has dynamic extent).  But
default values may require the evaluation of arbitrary expressions,
and those expressions might refer to the values of required parameters
or of \&optional arguments further left in the lambda list.  Such
expression can also refer to local variables in enclosing
environments.

The first return value is passed in \texttt{EAX}, and the number of
return values is passed in \texttt{EBX} as a \texttt{fixnum}.  If the
number of return values is $0$, then the callee must make sure
\texttt{EAX} contains \texttt{nil} when it returns.  This convention
makes it unnecessary for the caller that expects a single return value
to check the count.  Furthermore, this situation is probably the most
common one, so that in almost all cases, checking the number of return
values becomes unnecessary.

\section{Argument parsing}

By \emph{argument parsing}, we mean the process of analyzing the
arguments that were passed to a function, and using those arguments to
initialize variables corresponding to the names of the parameters of
that function.

Because of the presence of \emph{optional arguments} and \emph{keyword
  arguments} that arbitrary forms to evaluate, argument parsing is a
fairly complicated process.  The process is further complicated by the
fact that some of the parameters of the function might be
\emph{special variables} rather than \emph{lexical variables}.
However, the evaluation of default forms and the binding of special
variables is handled by code in the body of the function.  This code
is generated by the \cleavir{} compiler framework.

Most of the time, there are lexical variables and temporaries that are
not closed over.  Such variables need not be allocated in a level of
the lexical runtime environment, but can be allocated on the stack.
In this case, the place on the stack that these variables will occupy
overlaps with the place that the initial arguments are passed.  For
that reason, the arguments are first moved, leaving a hole in the
stack frame corresponding to the number of lexical variables and
temporaries that are not closed over.  Then, in each step, argument
parsing will either initialize one of these variables, or one of the
variables in the level of the lexical runtime environment.  Finally,
the arguments are removed from the stack, leaving only the lexical
variables and temporaries that are not closed over.  However, things
are complicated by the fact that as part of the argument-parsing
process, entries for dynamic variables may have been created.
Removing the arguments, then, consists of moving those entries to
clobber the arguments on the stack, and adjusting the dynamic
environment word to reflect this move.

As a special case of the scenario in the previous paragraph,
\emph{all} lexical variables and temporaries can be allocated on the
stack.  Then, no level of the lexical runtime environment will be
created at all, and the hole in the stack frame will be big enough to
hold all such variables.

Also as a special case of the scenario in that same paragraph, it is
common that some \emph{prefix} of the parameters of the function
contains variables that are not closed over.  The variables in this
prefix are already in their ultimate places, so only the remaining
arguments need to be moved.

When a function takes only required parameters and none of those
parameters are closed over, then the argument-parsing process is
reduced to \emph{nothing}, and all that needs to be done upon function
entry is to allocate space in the stack frame for temporaries.

