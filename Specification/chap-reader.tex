\chapter{Reader}

The reader can be implemented in an almost entirely portable way.  

The code for the reader is located in the \texttt{Code/Reader}
directory.  There are two subdirectories: \texttt{Code/Reader/Fast}
and \texttt{Code/Reader/Simple}.  

\section{The fast reader}

The directory \texttt{Code/Reader/Fast} contains the code for the
reader that we ultimately plan to make the main reader for \sysname{}.
However, at the moment, the code for the fast reader is very
incomplete and quite difficult to understand, mainly because it is
over-engineered.  

The main reason for the reader to be over-engineered is that we
wanted to create a very fast implementation (and we still do).  The
way we imagined doing this was to create special versions of the
reader according to important parameters such as the input base and
the readtable case.  We still think this is a good idea, but it might
be better to do it in different phases.  Phase 1 would be to create a
reader that works for any value of those parameters.  Phase 2 would be
to create a special version for the most common parameter combination
such as input base 10 and readtable case of \texttt{:upcase}.  Other
specialized versions could be developed later.  The main method of
obtaining good performance is to  provide a very fast tokenizer in the
form of a state machine that builds up the token while it is being
read.  

\section{The simple reader}

The directory \texttt{Code/Reader/Simple} contains the code for a
simple, straightforward reader.  It is currently fairly complete.
Only the \texttt{\#S} macro is missing.

We have made no attempt to make this reader fast.  Its main virtue is
its simplicity.  We are in the process of extracting the simple reader
into a separate repository.  Its name is \textbf{Eclector}.

\section{Portability problems}
\label{sec-reader-portability-problems}

The main problem as far as portability is concerned is with the reader
macros \texttt{\#\#} and \texttt{\#=}.  When an object labeled with
\texttt{\#=} is being read (and so has not been constructed yet) and a
reference to it is being encountered (thus creating a circular
structure), some temporary value must be put in until the object has
been constructed.  Then the object must be scanned for that temporary
value.  However scanning arbitrary objects for some substructure
cannot be done portably.  We use the system \texttt{closer-mop} to
scan standard objects, but we have no way of scanning structs yet.

Another (minor) problem with respect to portability has to do with the
\emph{backquote} facility.  The \hs{} explicitly encourages
implementations to use more specific constructs than the standard
\commonlisp{} functions \texttt{append}, \texttt{list}, and
\texttt{cons} as a result of parsing the \emph{backquote} reader
macro.  The reason for is so that the \emph{pretty printer} can print
such forms in a more convenient way (i.e. using the backquote and
comma characters).  However, the result is often an
implementation-specific encoding that would have to be translated into
standard \commonlisp{} constructs by the compiler, for instance by
writing macros for this encoding.  Such macros would then necessarily
be implementation specific.  The simple reader calls a generic
function that can be customized by client code whenever it needs to
wrap some code to represent the characters defined by the backquote
reader macro.
