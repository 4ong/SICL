\chapter{Low-level primitives}
\label{chap-low-level-primitives}

A large number of the operators of the \cl{} language are directly
defined in terms of operators at an even lower level.  At the lowest
level of abstraction, we find operations such as \texttt{car} and
\texttt{svref} that can be defined in terms of lower level operations
only if the internal representation of objects such as \texttt{cons}
cells and simple vectors is known.

For a particular implementation such as \sysname{}, however, this
representation is known, so it is reasonable to attempt to define the
lowest-level \cl{} operations in terms of even more primitive
operations. 

For \sysname{}, we define a collection of operators that are used the
in the same way that ordinary \cl{} functions are used, i.e., they are
represented in source code as S-expressions.  The symbols that name
these operators all live in a package called \texttt{sicl-word}.  

\section{Definition of low-level primitives}

\subsection{\texttt{memref}, reference memory}

This opereration takes a raw pointer and returns the contents of
memory at the location represented by that pointer.

\subsection{\texttt{memset}, set memory}

This operation takes a raw pointer and an arbitrary word and stores
the words at the memory location indicated by the raw pointer.

\subsection{\texttt{word}, constant machine integer}

This operation is used only with a signed constant integer as an
argument.  The value is the argument represented as a machine word in
2's complement representation.

\subsection{\texttt{u+}, unsigned addition}

This operation takes two word operands, interpreted as unsigned
integers.  It returns two values: the sum of the two operands modulo
the word size, and a Boolean value which is \emph{true} if and only if
the addition resulted in an overflow (carry).  

\subsection{\texttt{u-}, unsigned subtraction}

This operation takes two word operands, interpreted as unsigned
integers.  It returns two values: the difference between the two
operands modulo the word size, and a Boolean value which is
\emph{true} if and only if the addition resulted in an underflow
(carry).

\subsection{\texttt{s+}, signed addition}

This operation takes two word operands, interpreted as signed
integers.  It returns two values: the sum of the two operands modulo
the word size, and a Boolean value which is \emph{true} if and only if
the addition resulted in an overflow or an underflow.  If the two
operands were both positive and an overflow occurred, then the first
value is a negative value.  If the two operands were both negative and
an underflow occurred then the first value is a non-negative value. 

\subsection{\texttt{s-}, signed subtraction}
\subsection{\texttt{neg}, negation}
\subsection{\texttt{u*}, unsigned multiplication}
\subsection{\texttt{s*}, signed multiplication}

\subsection{\texttt{==}, equality}

Return \texttt{T} if the two word operands are the same.  Otherwise
return \texttt{NIL}.

\subsection{\texttt{u<}, unsigned less than}

Return \texttt{T} if the first word operand is strictly less than the
second one when they are both interpreted as unsigned integers.
Otherwise return \texttt{NIL}.

\subsection{\texttt{u<=}, unsigned less than or equal to}

Return \texttt{T} if the first word operand is less than or equal to
the second one when they are both interpreted as unsigned integers.
Otherwise return \texttt{NIL}.

\subsection{\texttt{s<}, signed less than}

Return \texttt{T} if the first word operand is strictly less than the
second one when they are both interpreted as signed integers.
Otherwise return \texttt{NIL}.

\subsection{\texttt{s<=}, signed less than or equal to}

Return \texttt{T} if the first word operand is less than or equal to
the second one when they are both interpreted as signed integers.
Otherwise return \texttt{NIL}.

\subsection{\texttt{\&}, bit-wise and}

Return a word that is the bitwise logical \texttt{and} between the two
word operands. 

\subsection{\texttt{ior}, bit-wise or}

Return a word that is the bitwise logical \texttt{or} (inclusive)
between the two word operands.

\subsection{\texttt{xor}, bit-wise exclusive or}

Return a word that is the bitwise logical \texttt{exclusive or}
between the two word operands.

\subsection{\texttt{\~}, bit-wise not}

Return a word that is the bitwise logical \texttt{not} of its operand.

\subsection{\texttt{<\thinspace <}, left shift}


\subsection{\texttt{>\thinspace >}, right shift}

\section{Examples of use}

\begin{codefragment}
\inputcode{code-car-1.code}
\caption{\label{code-car-1}
Example implementation of \texttt{car}.}
\end{codefragment}
