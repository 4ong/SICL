\chapter{Bootstrapping}

An image I is said to be \emph{viable} if and only if it is possible
to obtain a complete \cl{} system by starting with I and loading a
sequence of ordinary compiled files.

There are many ways of making a viable image.  We are interested in
making a \emph{small} viable image.  In particular, we want the
initial image \emph{not to contain the compiler}, because we want to
be able to load the compiler as a bunch of compiled files. 

By requiring the viable image not to contain the compiler, we place
some restrictions on the code in it.  

One such restriction is that creating the discriminating function of a
generic function is not allowed to invoke the compiler.  Since doing
so is typical in any CLOS implementation, we must find a different
way.

Another such restriction concerns the creation of classes.  Typically,
when a class is created, the reader and writer methods are created by
invoking the compiler.  Again, a different way must be found.

In order to create an instance of a class, the following functions
are called, so they must exist and be executable:

\begin{itemize}
\item \texttt{make-instance}.  Called directly to create the instance.
\item \texttt{allocate-instance}.  Called by \texttt{make-instance} to
  allocate storage for the instance.
\item \texttt{initialize-instance}.  Called by \texttt{make-instance}
  to initialize the slots of the instance.
\item \texttt{shared-initialize}.  Called by
  \texttt{initialize-instance}.
\item \texttt{slot-boundp-using-class}.  Called by \texttt{shared-initialize}.
\item \texttt{(setf slot-value-using-class)}.  Called by
  \texttt{shared-initialize}.
\end{itemize}

Most of these functions are generic functions, so the functionality
that implements generic function dispatch must also be in place,
including:

\begin{itemize}
\item \texttt{compute-discriminating-function}.
\item \texttt{compute-applicable-methods}.
\item \texttt{compute-applicable-methods-using-classes}.
\item \texttt{compute-effective-method}.  This function returns a form
  to be compiled, but since we do not have the compiler, we instead
  use some more direct method similar to what is shown in chapter 1 of
  the AMOP.  Furthermore, in order to take into account different
  method combinations, we also need the compiler, so only the method
  combinations used by the compiler will be considered at
  bootstrapping.
\end{itemize}

\section{Bootstrapping stages}

\subsection{Stage 1, bootstrapping CLOS}

\subsubsection{Definitions}

\begin{definition}
A \emph{host class} is a class in the host system.  If it is an
instance of the host class \texttt{host:standard-class}, then it is
typically created by \texttt{host:defclass}.
\end{definition}

\begin{definition}
A \emph{host instance} is an instance of a host class.  If it is an
instance of the class \texttt{host:standard-object}, then it is
typically created by a call to \texttt{host:make-instance} using a
host class or the name of a host class.
\end{definition}

\begin{definition}
An \emph{ersatz instance} is a target instance represented as a host
data structure, using a host structure instance to represent the
\emph{header} and a host simple vector to represent the \emph{rack}.
\end{definition}

\begin{definition}
An \emph{ersatz class} is an ersatz instance that can be instantiated
to obtain another ersatz instance.
\end{definition}

\begin{definition}
A \emph{bridge class} is a representation of a target class as a host
instance.  A bridge class can be instantiated to obtain a bridge
instance. 
\end{definition}

\begin{definition}
A \emph{bridge instance} is a target instance represented just like an
ersatz instance, except that the class object stored in the header is
not an ersatz class, but a bridge class. 
\end{definition}

\begin{definition}
A \emph{host generic function} is a generic function created by
\texttt{host:defgeneric}, so it is a host instance of the host class
\texttt{host:generic-function}.  Arguments to the discriminating
function of such a generic function are host instances.
\texttt{host:class-of} is called on some required arguments in order
to determine what methods to call.
\end{definition}

\begin{definition}
A \emph{host method} is a method created by \texttt{host:defmethod},
so it is a host instance of the host class \texttt{host:method}.  The
class specializers of such a method are host classes.
\end{definition}

\begin{definition}
An \emph{ersatz generic function} is an instance of the ersatz class
\texttt{target:generic-function}.  An ersatz generic function can not
be executed.  It can only be translated into a target generic function.
The methods on an ersatz generic function are ersatz methods. 
\end{definition}

\begin{definition}
An \emph{ersatz method} is an instance of the ersatz class
\texttt{target:method}.  The class specializers of such a method are
ersatz classes.
\end{definition}

\begin{definition}
A \emph{bridge generic function} is a generic function which is a host
instance of the host class \texttt{target:bridge-generic-function}.
This class is a subclass of \texttt{host:funcallable-standard-object}
and of the bridge class \texttt{target:standard-generic-function}.
A bridge generic function is executable just like any host function.

Arguments to a bridge generic function are ersatz instances.  The
bridge generic function uses the 
\emph{stamp}
\seesec{sec-generic-function-dispatch-the-discriminating-function} of
the required arguments to dispatch on. 

The methods on a bridge generic function are bridge methods.
\end{definition}

\begin{definition}
A \emph{bridge method} is a a method which is a host instance of the host
class \texttt{target:method}.  The class specializers of such a method are
bridge classes.  The \emph{method function} of a bridge method is an
ordinary host function.
\end{definition}

\subsubsection{Phase 1}

Phase 1 is characterized by the fact that the MOP names are names of
host classes and host generic functions.  

We start by defining several packages.

The package named
\texttt{sicl-boot-common} contains and exports symbols that are used
the same way throughout the boot process.  In particular, it contains
symbols for allocating and accessing ersatz instances.  This package
is used by the phase-specific packages.  

The package named \texttt{aspiring-sicl-clos} contains symbols that
will eventually have the package \texttt{sicl-clos} as their home
package.  This package can not be named \texttt{sicl-clos} at this
point, because that name is going to be used as a nickname for
phase-specific packages during the boot process.  Furthermore, this
package initially only contains the names of classes.  

The package named \texttt{sicl-boot-phase1} is the phase-specific
package for phase 1.  During phase 1, it will have the nickname
\texttt{sicl-clos}, but this nickname is removed at the end of phase
1.   This package shadows symbols that are defined in the package
\texttt{common-lisp} and that name MOP-specified names.  This package
also shadows the symbol \texttt{defclass}, for reasons explained
below. 

In the following steps, we define macros \texttt{defclass} and
\texttt{define-built-in-class} in preparation for loading the MOP
class hierarchy.  In order to be able to use the file of MOP classes
unmodified, we define a version if \texttt{defclass} that expand into
\texttt{host:defclass} except that \texttt{T} is removed if it is a
superclass.  We also define a version of
\texttt{define-built-in-class} that expands into \texttt{defclass}
except that if the class to be defined is named \texttt{T} it expands
to \texttt{nil} instead.  The reason for this trickery is that we do
not want to shadow the symbol \texttt{T} since it is used for other
purposes in the MOP class hierarchy.  Since we do not shadow
\texttt{T}, we can not define a class with that name.  Furthermore,
\texttt{T} will not be the superclass of any of the classes defined in
phase 1.  It is however not required to have \texttt{T} as a
superclass for these classes.  All that matters is that these classes
have the right slots, and \texttt{T} does not supply any slots anyway.

In the next step, we load the file containing the MOP class hierarchy.
The names of the host classes created will be
\texttt{target:standard-object}, \texttt{target:metaobject}, etc.
These host classes created here have the analogous superclasses
compared their corresponding target MOP classes, and they define the
analogous slots.  Their metaclasses are all wrong of course, because
each one of them has \texttt{host:standard-class} as metaclass.  The
class finalization protocol of the host will make sure that instances
of these host classes have the effective slots determined by class
inheritance.  In other words, each of the classes define slots with
the right names, the right initargs, etc.

We define \emph{databases} and associated query functions for
\emph{bridge classes} and \emph{bridge generic function}, because when
one of the MOP classes defined in phase 1 is instantiated, it
explicitly creates a bridge class, and it implicitly creates bridge
generic functions and bridge methods for the accessors of those
classes.  Other host instances are created here, in particular
instances of \texttt{target:standard-direct-slot-definition}, but
these instances do not require a separate database, because they are
referred to by the bridge classes that are created. 

During phase 1, we also establish the \texttt{:after} methods on
\texttt{host:initialize-instance} that implement the initialization
protocols for class metaobjects, generic function metaobjects, etc.
We define special versions of \texttt{ensure-class} and
\texttt{ensure-generic-function} that call \texttt{host:make-instance}
to create bridge classes and bridge generic functions.

Finally, during phase 1, we define functions that will ultimately be
used for manipulating the bridge classes and bridge generic functions
that are created during phase 1, including computing the
discriminating function of bridge generic functions and creating
instances of bridge classes.  For functions like
\texttt{make-instance} that we can not create in phase 1 because they
would class with the host function of the same name, we only define
the supporting ordinary functions and we stop short of defining the
main functions.  We define all this machinery in phase 1, because it
will use the accessors that are implicitly defined when MOP classes
are created here. 

The support code for \texttt{shared-initialize} for the purpose of
creating ersatz instances is defined here, because shared initialize
accesses the \emph{class} of the instance to initialize, and the class
if an ersatz instance is a bridge instance.  So the generic functions
called by shared-initialize when an ersatz instance is created are
host generic functions taking bridge classes as arguments. 

\subsubsection{Phase 2}

Phase 2 is characterized by the fact that MOP names are names of
bridge classes and bridge generic functions.  For that purpose, we
define the specified macros \texttt{defclass}, \texttt{defgeneric},
and \texttt{defmethod}.  These macros expand to calls to corresponding
\texttt{ensure-...} functions of phase 1.  These macros need some
support code that is also defined here, such as the canonicalization
functions for \texttt{defclass} and \texttt{make-method-lambda} for
\texttt{defmethod}.

Next, we create bridge classes corresponding to the classes of the MOP
hierarchy.  When a bridge class is created, it will automatically
create bridge generic functions corresponding to slot readers and
writers.  This is done by calling \texttt{ensure-generic-function} of
phase 1.

Creating bridge classes this way will also instantiate the host class
\texttt{target:direct-slot-definition}, so that our bridge classes
contain host instances bridge in their slots. 

In this phase, we also prepare for the creation of ersatz instances.
We establish a database of ersatz classes and a database of ersatz
generic functions.


\subsubsection{Phase 3}

The purpose of this phase is to create ersatz generic functions and
ersatz classes, by instantiating bridge classes.  

At the end of this phase, we have a set of ersatz instances, some of
which are ersatz classes, except that the \texttt{class} slot of the
header object of every such instance is a bridge class.  We also have
a set of ersatz generic functions (mainly accessors) that are ersatz
instances like all the others. 

\subsubsection{Phase 4}

The first step of this phase is to finalize all the ersatz classes
that were created in phase 3.  Finalization will create ersatz
instances of bridge classes corresponding to effective slot
definitions. 

The second step is to \emph{patch} all the ersatz instances allocated
so far.  There are two different ways in which ersatz instances must
be patched.  First of all, all ersatz instances have a bridge class in
the \texttt{class} slot of the header object.  We patch this
information by accessing the \emph{name} of the bridge class and
replacing the slot contents by the ersatz class with the same name.
Second, ersatz instances of the class \texttt{standard-object} contain
a list of effective slot definition objects in the second word of the
rack, except that those effective slot definition objects
are bridge instances, because they were copied form the
\texttt{class-slots} slot of the bridge class when the bridge class
was instantiated to obtain the ersatz instance.  Since all ersatz
classes were finalized during the first step of this phase, they now
all have a list of effective slot definition objects, and those
objects are ersatz instances.  Patching consists of replacing the
second word of the rack of all instances of
\texttt{standard-object} by the contents of the \texttt{class-slots}
slot of the class object of the instance, which is now a ersatz
class. 

The final step in this phase is to \emph{install} some of the
remaining bridge generic functions so that they are the
\texttt{fdefinition}s of their respective names.  We do not install
all remaining bridge generic functions, because some of them would
clobber host generic functions with the same name that are still
needed.  

At the end of this phase, we have a complete set of bridge generic
functions that operate on ersatz instances.  We still need bridge
classes to create ersatz instances, because the \emph{initfunction}
needs to be executed for slots that require it, and only host
functions are executable at this point.

\subsubsection{Phase 5}

The purpose of this phase is to create ersatz instances for all
objects that are needed in order to obtain a viable image, including: 

\begin{itemize}
\item ersatz built-in classes such as \texttt{package}, \texttt{symbol},
  \texttt{string}, etc., 
\item ersatz instances of those classes, such as the required
  packages, the symbols contained in those packages, the names of
  those symbols, etc.
\item ersatz standard classes for representing the global environment
  and its contents.
\item ersatz instances of those classes.
\end{itemize}

\subsubsection{Phase 6}

The purpose of this phase is to replace all the host instances that
have been used so far as part of the entire ersatz structure, such as
symbols, lists, and integers by their ersatz counterparts.

\subsubsection{Phase 7}

The purpose of this phase is to take the simulated graph of objects
used so far and transfer it to a \emph{memory image}.  

\subsubsection{Phase 8}

Finally, the memory image is written to a binary file. 


\subsection{Stage 2, compiling macro definitions}

Stage 1 of the bootstrapping process consists of using the cross
compiler to compile files containing definitions of standard macros
that will be needed for compiling other files. 

When a \texttt{defmacro} form is compiled by the cross compiler, we
distinguish between the two parts of that defmacro form, namely the
\emph{expander code} and the \emph{resulting expansion code}.  The
\emph{expander code} is the code that will be executed in order to
compute the resulting expansion code when the macro is invoked.  The
\emph{resulting expansion code} is the code that replaces the macro
call form. 

As an example, consider the hypothetical definition of the
\texttt{and} macro shown in \refCode{code-defmacro-and}.

\begin{codefragment}
\inputcode{code-defmacro-and.code}
\caption{\label{code-defmacro-and}
Example implementation of the \texttt{and} macro.}
\end{codefragment}

In \refCode{code-defmacro-and}, the occurrences of \texttt{car},
\texttt{cdr}, \texttt{null}, and \texttt{cond} are part of the
\emph{expander code} whereas the occurrence of \texttt{when}, of
\texttt{t}, and the occurrence of \texttt{and} in the last line are
part of the resulting expansion code. 

The result of expanding the \texttt{defmacro} form in
\refCode{code-defmacro-and} is shown in
\refCode{code-macro-expansion-and}. 

\begin{codefragment}
\inputcode{code-macro-expansion-and.code}
\caption{\label{code-macro-expansion-and}
Expansion of the macro call.}
\end{codefragment}

Thus, when the code in \refCode{code-defmacro-and} is compiled by the
cross compiler, it is first expanded to the code in
\refCode{code-macro-expansion-and}, and the resulting code is compiled
instead.  Now \refCode{code-macro-expansion-and} contains an
\texttt{eval-when} at the top level with all three situations (i.e.,
\texttt{:compile-toplevel}, \texttt{:load-toplevel}, and
\texttt{:execute}.  As a result, two things happen to the
\texttt{funcall} form of \refCode{code-macro-expansion-and}:

\begin{enumerate}
\item It is \emph{evaluated} by the \emph{host function}
  \texttt{eval}.
\item It is \emph{minimally compiled} by the cross compiler.
\end{enumerate}

In order for the evaluation by the host function \texttt{eval} to be
successful, the following must be true:

\begin{itemize}
\item All the \emph{functions} and \emph{macros} that are
  \emph{invoked} as a result of the call to \texttt{eval} must exist.
  In the case of \refCode{code-macro-expansion-and}, the function
  \texttt{(setf sicl-environment::macro-function)} must exist, and that
    is all.
\item All the \emph{macros} that occur in macro forms that are
  \emph{compiled} as a result of the call to \texttt{eval} must
  exist.  These are the macros of the expansion code; in our example
  only \texttt{cond}.  Clearly, if only standard \cl{} macros are
  used in the expansion code of macros, this requirement is
  automatically fulfilled.
\item It is \emph{preferable}, though not absolutely necessary for the
  \emph{functions} that occur in function forms that are
  \emph{compiled} as a result of the call to \texttt{eval} exist.  If
  they do not exist, the compilation will succeed, but a warning will
  probably be issued.  These functions are the functions of the
  expansion code; in our example \texttt{car}, \texttt{cdr}, and
  \texttt{null}.  Again, if only standard \cl{} function are used in
  the expansion code of macros, this requirement is automatically
  fulfilled.  It is common, however, for other functions to be used as
  well.  In that case, those functions should preferably have been
  loaded into the host environment first. 
\end{itemize}

In order for the minimal compilation by the cross compiler to be
successful, the following must be true:

\begin{itemize}
\item All the \emph{macros} that occur in macro forms that are
  \emph{minimally compiled} by the cross compiler must exist.  These
  are again the macros of the expansion code; in our example only
  \texttt{cond}.  Now, the cross compiler uses the \emph{compilation
    environment} of the \emph{target} when looking up macro
  definitions.  Therefore, in order for the example in
  \refCode{code-defmacro-and} to work, a file containing the
  definition of the macro \texttt{cond} must first be compiled by the
  cross compiler. 
\item While it would have been desirable for the \emph{functions} that
  occur in function forms that are \emph{minimally compiled} by the
  cross compiler to exist, this is typically not the case.%
  \fixme{Investigate the possibility of first compiling a bunch of
    \texttt{declaim} forms containing type signatures of most
    standard \cl{} functions used in macro expansion code.}
  As a
  result, the cross compiler will emit warnings about undefined
  functions.  The generated code will still work, however.
\end{itemize}

Of the constraints listed above, the most restrictive is the one that
imposes an order between the files to be cross compiled, i.e., that
the macros of the expansion code must be cross compiled first.  It is
possible to avoid this restriction entirely by using \emph{auxiliary
  functions} rather than macros.  The alternative implementation of
the \texttt{and} macro in \refCode{code-defmacro-and-2} shows how this
is done in the extreme case.

\begin{codefragment}
\inputcode{code-defmacro-and-2.code}
\caption{\label{code-defmacro-and-2}
Alternative implementation of the \texttt{and} macro.}
\end{codefragment}

We use the technique of \refCode{code-defmacro-and-2} only when the
expansion code is fairly complex.  An example of a rather complex
expansion code is that of the macro \texttt{loop} which uses mutually
recursive functions and fairly complex data structures.  When this
technique is used, we can even use a macro to implement its own
expansion code.  For instance, nothing prevents us from using
\texttt{loop} to implement the functions of the expander code of
\texttt{loop}, because when the \texttt{loop} macro is used to expand
code in the cross compiler, the occurrences of \texttt{loop} in the
functions called by the expander code are executed by the host.  As it
turns out, we do not do that, because we would like for the \sysname{}
implementation of \texttt{loop} to be used as a drop-in extension in
implementations other than \sysname{}.


