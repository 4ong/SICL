\chapter{Bootstrapping}

\section{General technique}

\sysname{} is bootstrapped from an existing \commonlisp{}
implementation that, in addition to the functionality required by the
standard, also contains the library \texttt{closer-mop}.  This
\commonlisp{} system is called the \emph{host}.  The result of the
bootstrapping process is an \emph{image} in the form of an executable
file containing a \sysname{} system.  This system is called the
\emph{target}.  The target image does not contain a complete
\commonlisp{} system.  It only contains enough functionality to load
the remaining system from compiled files.
\seesec{sec-bootstrapping-viable-image}.

In general, the target image can be thought of as containing
\emph{graph} of \commonlisp{} objects that have been placed in memory
according to the spaces managed by the memory manager.  To create this
graph, we first generate an \emph{isomorphic} graph of host objects in
the memory of an executing host system.  To generate the target image,
the isomorphic host graph is traversed, creating a target version of
each object in the host graph, and placing that object on an
appropriate address in the target image.

The isomorphic host graph contains objects that are \emph{analogous}
to their target counterparts as follows:

\begin{itemize}
\item A target \texttt{fixnum} is represented as a host integer.
  Whether the integer is a host fixnum or not depends on the fixnum
  range of the host.
\item A target \texttt{character} is represented as a host character.
\item A target \texttt{cons} cell is represented as a host
  \texttt{cons} cell.
\item A target general instance is represented as a host
  \texttt{standard-object} for the \emph{header} and a host
  \texttt{simple-vector} for the \emph{rack}.
\item Target objects such as bignums or floats are not needed at this
  stage of bootstrapping, so they do not have any representation as
  host objects.
\end{itemize}

\section{Global environments for bootstrapping}

During different stages of bootstrapping, a particular \emph{name} (of
a function, class, etc) must be associated with different objects.  As
a trivial example, the function \texttt{allocate-object} in the host
system is used to allocate all standard objects.  But
\texttt{allocate-object} is also a target function for allocating
objects according to the way such objects are represented by the
target system.  These two functions must be available simultaneously.

Most systems solve this problem by using temporary names for target
packages during the bootstrapping process.  For example, even though
in the final target system, the name \texttt{allocate-object} must be
a symbol in the \texttt{common-lisp} package, during the bootstrapping
process, the name might be a symbol in a package with a different
name.

In \sysname{} we solve the problem by using multiple \emph{first-class
  global environments}.

For the purpose of bootstrapping, it is convenient to think of
\texttt{eval} as consisting of two distinct operations:

\begin{itemize}
\item Compile.  A \emph{compilation environment} is used to expand
  macros and for other compilation purposes.  The result of
  compilation is code that is \emph{untied} to any particular
  environment.
\item Tie.  The untied code produced by the first step is \emph{tied}
  to a particular run-time environment.  Tying is accomplished by
  calling the top-level function created by the compilation, passing
  it the result of evaluating \texttt{load-time-value} forms in a
  particular environment.
\end{itemize}

The reason we need to separate these two operations is that for
bootstrapping purposes, the two are going to use distinct global
environments.

\section{Viable image}
\label{sec-bootstrapping-viable-image}

An image I is said to be \emph{viable} if and only if it is possible
to obtain a complete \commonlisp{} system by starting with I and loading a
sequence of ordinary compiled files.

There are many ways of making a viable image.  We are interested in
making a \emph{small} viable image.  In particular, we want the
initial image \emph{not to contain any evaluator} (no compiler, no
interpreter), because we want to be able to load the compiler as a
bunch of compiled files.

By requiring the viable image not to contain the compiler, we place
some restrictions on the code in it.  

One such restriction is that creating the discriminating function of a
generic function is not allowed to invoke the compiler.  Since doing
so is typical in any CLOS implementation, we must find a different
way.  We solve this problem by using a general-purpose discriminating
function that traverses the call history and invokes the effictive
method functions that corresponds to the arguments of the call, and if
no effective method function can be found, calls the machinery to
compute the effective method function and adds the new effective
method function to the call history.

Another such restriction concerns the creation of classes.  Typically,
when a class is created, the reader and writer methods are created by
invoking the compiler.  Again, a different way must be found.  To solve
this problem, the method function of an accessor method is created
as a closure that closes over the index of the slot in the rack.

\subsection{Arguments passed to \texttt{fasl} function}
\label{sec-bootstrapping-arguments-to-fasl-function}

Recall that the top-level function in a compiled file is passed a
certain number of arguments that are functions that can be called as
part of the loading process:

\begin{itemize}
\item A function for allocating a string of a certain length.  This
  allocation can not be accomplished with any more general function
  that requires arguments other than fixnums.
\item The function \texttt{(setf aref)} to fill a string with
  characters.
\item The function \texttt{find-package} to obtain a package from its
  name, which is a string.
\item The function \texttt{intern} to obtain a symbol from its name
  and a package.
\item The function \texttt{cons} to create function names that are
  lists.
\item The function \texttt{fdefinition} to obtain a globally defined
  function, given its name.
\end{itemize}

Other operations, such as obtaining a class metaobject or finding the
value of a special variable, are accomplished by calling functions
obtained using the steps above.

The next thing to determine is exactly what functions must exist in
the minimal viable image in order for it to be possible to create a
complete \commonlisp{} system.

\subsection{Creating a string}
\label{sec-bootstrapping-creating-a-string}

The code in a compiled file for creating a string looks like this:

\begin{verbatim}
(let ((string (funcall make-string <length>)))
  (funcall setf-aref <char-0> string 0)
  (funcall setf-aref <char-1> string 1)
  ...
\end{verbatim}

where \texttt{<length>} is the length of the string, given as a
fixnum.

Here \texttt{funcall} is not a call to the \commonlisp{} function of
that name, but rather a built-in mechanism for calling a function that
is the value of a lexical variable.  The variable \texttt{make-string}
is the first argument to the top-level function of the compiled file
as mentioned in \ref{sec-bootstrapping-arguments-to-fasl-function}.
Similarly, the variable \texttt{setf-aref} contains the function in
the second argument to the top-level function of the compiled file.

\subsection{Finding a package}
\label{sec-bootstrapping-finding-a-package}

To find a package when a compiled file is loaded, the following code
is executed:

\begin{verbatim}
(let* ((name <make-name-as-string>)
       (package (funcall find-package name)))
  ...
\end{verbatim}

where \texttt{<make-name-as-string>} is the code that is shown in
\ref{sec-bootstrapping-creating-a-string}.

The variable \texttt{find-package} is the third argument to the
top-level function of the compiled file.

\subsection{Executing generic functions}
\label{sec-bootstrapping-executing-generic-functions}

The functionality that implements generic function dispatch must also
be in place in the minimal viable image, including:

\begin{itemize}
\item \texttt{compute-discriminating-function}.  In the initial
  imagine, this function will be a general function that scans the
  call history and invokes the corresponding effective method.  If no
  effective method corresponding to the arguments exists, then a new
  one is computed from the applicable methods.
\item \texttt{compute-applicable-methods}.
\item \texttt{compute-applicable-methods-using-classes}.
\item \texttt{compute-effective-method}.  This function returns a form
  to be compiled, but since we do not have the compiler, we instead
  use some more direct method.  We suggest calling a
  \sysname{}-specific function called (say)
  \texttt{compute-effective-method-function} instead.  This call
  returns a closure that can be added to the call history of the
  generic function.  This function does not call the compiler, so it
  can be part of the initial image.  However, this new function can be
  a generic function, so it can take different method combinations
  into account, making it possible to support all the method
  combinations used by the compiler.
\end{itemize}

\subsection{Creating general instances}
\label{sec-bootstrapping-creating-general-instances}

In order to create an instance of a class, the following functions
are called, so they must exist and be executable:

\begin{itemize}
\item \texttt{make-instance}.  Called directly to create the instance.
\item \texttt{allocate-instance}.  Called by \texttt{make-instance} to
  allocate storage for the instance.
\item \texttt{initialize-instance}.  Called by \texttt{make-instance}
  to initialize the slots of the instance.
\item \texttt{shared-initialize}.  Called by
  \texttt{initialize-instance}.
\item \texttt{slot-boundp-using-class}.  Called by
  \texttt{shared-initialize}.  This function works by taking the slot
  index from the effective-slot object and then using the function
  \texttt{standard-instance-access} to check the value of the slot.
  Thus, it does not use any complex generic-function machinery.
\item \texttt{(setf slot-value-using-class)}.  Called by
  \texttt{shared-initialize}.  Like the previous function, this one
  also uses the slot index directly to set the slot value, so no
  complicated machinery is required.
\end{itemize}

Most of these functions are generic functions, so it will use the
functionality described in
\refSec{sec-bootstrapping-executing-generic-functions}.


\subsection{Creating class metaobjects}

A class metaobject is itself a general instance, so the machinery
described in \refSec{sec-bootstrapping-creating-general-instances} is
required.  In addition, all the functionality for initializing class
metaobjects is necessary.  This functionality is fairly simple,
because the complex part is accomplished during class finalization.
Some error checking is typically done, but it does not have to be
present in the initial viable image.  It can be loaded later as
auxiliary methods on \texttt{shared-initialize}.

\subsection{Class finalization}

The class finalization protocol involves calling the following
functions:

\begin{itemize}
\item \texttt{finalize-inheritance}, the main entry point.
\item \texttt{compute-class-precedence-list} to compute the
  precedence list.  The value computed by this function is associated
  with the class metaobject using some unknown mechanism.  We can
  either use \texttt{reinitialize-instance} or a slot writer.
\item \texttt{compute-slots} to compute a list of effective
  slot definition metaobjects.
\item \texttt{effective-slot-definition-class}.  This function is
  called by \texttt{compute-slots} to determine what class to use in
  order to instantiate the effective slot definition metaobjects.
\item \texttt{compute-effective-slot-definition}.  This function is
  also called by \texttt{compute-slots}.
\end{itemize}

\subsection{Creating generic-function metaobjects}

A generic-function metaobject is itself a general instance, so the
machinery described in
\refSec{sec-bootstrapping-creating-general-instances} is required.  In
addition, all the functionality for initializing generic-function
metaobjects is necessary.

Initializing a generic function metaobject is mainly a matter of
checking certain argument combinations for validity, and of supplying
defaults for certain keyword arguments if not supplied.  This
functionality is provided by auxiliary methods on
\texttt{initialize-instance} and \texttt{shared-initialize}, so those
auxiliary methods must be present in the image.

\subsection{Creating method metaobjects}

A method metaobject is itself a general instance, so the machinery
described in \refSec{sec-bootstrapping-creating-general-instances} is
required.  In addition, all the functionality for initializing method
metaobjects is necessary.

Notice that methods will be created when compiled files are being
loaded, so the code in the method will already exist in the compiled
file.  In particular, the function given as the \texttt{:function}
keyword argument to \texttt{make-instance} has already been processed
by \texttt{make-method-lambda} and then compiled by the cross
compiler.  Therefore, none of this machinery needs to exist in the
minimal viable image.

In other words, initializing a method metaobject is mainly a matter of
checking the validity of supplied arguments.

\subsection{Creating slot definition metaobjects}

A slot definition metaobject is itself a general instance, so the
machinery described in
\refSec{sec-bootstrapping-creating-general-instances} is required.  In
addition, all the functionality for initializing slot definition
metaobjects is necessary.

The only potential complication here is the keyword argument
\texttt{:initfunction}.  However, the code for this function has
already been created and compiled by the cross compiler by the time
the compiled file is being loaded.  For the remaining keyword
arguments, it is just a matter of checking their validity.

\section{Bootstrapping stages}

\subsection{Stage 1, bootstrapping CLOS}

\subsubsection{Definitions}

\begin{definition}
A \emph{host class} is a class in the host system.  If it is an
instance of the host class \texttt{standard-class}, then it is
typically created by the host macro \texttt{defclass}.
\end{definition}

\begin{definition}
A \emph{host instance} is an instance of a host class.  If it is an
instance of the host class \texttt{standard-object}, then it is
typically created by a call to the host function
\texttt{make-instance} using a host class or the name of a host class.
\end{definition}

\begin{definition}
An \emph{ersatz instance} is a target instance represented as a host
data structure, using a host structure instance to represent the
\emph{header} and a host simple vector to represent the \emph{rack}.
\end{definition}

\begin{definition}
An \emph{ersatz class} is an ersatz instance that can be instantiated
to obtain another ersatz instance.
\end{definition}

\begin{definition}
A \emph{bridge class} is a representation of a target class as a host
instance.  A bridge class can be instantiated to obtain a bridge
instance. 
\end{definition}

\begin{definition}
A \emph{bridge instance} is a target instance represented just like an
ersatz instance, except that the class object stored in the header is
not an ersatz class, but a bridge class. 
\end{definition}

\begin{definition}
A \emph{host generic function} is a generic function created by the
host macro \texttt{defgeneric}, so it is a host instance of the host
class \texttt{generic-function}.  Arguments to the discriminating
function of such a generic function are host instances.  The host
function \texttt{class-of} is called on some required arguments in
order to determine what methods to call.
\end{definition}

\begin{definition}
A \emph{host method} is a method created by the host macro
\texttt{defmethod}, so it is a host instance of the host class
\texttt{method}.  The class specializers of such a method are host
classes.
\end{definition}

\begin{definition}
An \emph{ersatz generic function} is an instance of the ersatz class
\texttt{generic-function}.  An ersatz generic function can not be
executed.  It can only be translated into a target generic function.
The methods on an ersatz generic function are ersatz methods.
\end{definition}

\begin{definition}
An \emph{ersatz method} is an instance of the ersatz class
\texttt{method}.  The class specializers of such a method are ersatz
classes.
\end{definition}

\begin{definition}
A \emph{bridge generic function} is a generic function which is a host
instance of the host class \texttt{target:bridge-generic-function}.
This class is a subclass of \texttt{host:funcallable-standard-object}
and of the bridge class \texttt{target:standard-generic-function}.
A bridge generic function is executable just like any host function.

Arguments to a bridge generic function are ersatz instances.  The
bridge generic function uses the 
\emph{stamp}
\seesec{sec-generic-function-dispatch-the-discriminating-function} of
the required arguments to dispatch on. 

The methods on a bridge generic function are bridge methods.
\end{definition}

\begin{definition}
A \emph{bridge method} is a a method which is a host instance of the host
class \texttt{target:method}.  The class specializers of such a method are
bridge classes.  The \emph{method function} of a bridge method is an
ordinary host function.
\end{definition}

\subsubsection{Preparation}

Before bootstrapping can start, a certain number of \asdf{} systems
must be loaded, and in particular the system called
\texttt{sicl-clos-boot-support}.

A number of environments are involved in the bootstrapping process,
and it is important to tell them apart.  During the bootstrapping
process, several different \emph{compilation environments} and several
different \emph{run-time environments} are created and used.  The
different compilation environments differ slightly in the way macros
are defined, and the different run-time environments are used to
create classes and generic functions in different phases of the
bootstrapping process.  The different run-time environments will also
have slightly different versions of certain functions that are called
to create these classes and generic functions.

To distinguish between all these environments, we call the different
compile-time environments $C_1$, $C_2$, etc., and the different
run-time environments $R_1$, $R_2$, etc.  The global host environment
is called $H$.

\subsubsection{Phase 1}

The purpose of phase~1 is to create a hierarchy of host standard
classes that has the same structure as the hierarchy of MOP classes.

Three different environments are involved in phase~1:

\begin{itemize}
\item The global host environment $H$ is used to find the host classes
  named \texttt{standard-class}, \texttt{funcallable-standard-class},
  \texttt{built-in-class}, and \texttt{standard-generic-function}.
  These classes are used by \texttt{make-instance} in order to create
  the classes in $R_1$ and the generic functions in $R_2$
\item The run-time environment $R_1$ is where instances of the host
  classes named \texttt{standard-class},
  \texttt{funcallable-standard-class}, and \texttt{built-in-class}
  will be associated with the names of the MOP hierarchy of classes.
  These instances are thus host classes.
\item The run-time environment $R_2$ is where instances of the host
  class named \texttt{standard-generic-function} will be associated
  with the names of the different accessors specialized to host
  classes created in $R_1$.
\end{itemize}

One might ask at this point why generic functions are not defined in
the same environment as classes.  The simple answer is that there are
some generic functions that were automatically imported into $R_1$
from the host, that we still need in $R_1$, and that would have been
overwritten by new ones if we had defined new generic functions in
$R_1$.

Several adaptations are necessary in order to accomplish phase~1:

\begin{itemize}
\item The macro \texttt{defgeneric} is defined in $R_1$ to remove
  unbind the name in the run-time environment (which will be $R_2$)
  before calling \texttt{ensure-generic-function} to create a fresh
  generic function in the run-time environment.  This way, we make
  sure that we have a fresh generic function with no methods as a
  result of a call to \texttt{defgeneric}.  However, we must also make
  sure we make one single call to \texttt{defgeneric}.
\item The function \texttt{ensure-generic-function} is defined in
  $R_1$ and operates in $R_2$.  It checks whether there is already a
  function with the name passed as an argument in $R_2$, and if so, it
  returns that function.  It makes no verification that such an
  existing function is really a generic function; it assumes that it
  is.  It also assumes that the parameters of that generic function
  correspond to the arguments of \texttt{ensure-generic-function}.  If
  there is no generic function with the name passed as an argument in
  $R_2$, it creates an instance of the host class
  \texttt{standard-generic-function} and associate it with the name in
  $R_2$.  We define \texttt{ensure-generic-function} in $R_1$ so that
  it can also be used to find accessor functions when MOP classes are
  instantiated and methods need to be added to these accessor
  functions.
\item The function \texttt{ensure-class} has a special version in
  $R_1$.  Rather than checking for an existing class, it always
  creates a new one.  Furthermore, before calling the host function
  \texttt{make-instance}, it removes the \texttt{:readers} and
  \texttt{:writers} entries from the canonicalized slot specifications
  in the \texttt{:direct-slots} keyword argument.  The result is that
  the class will be created without any slot accessors.  We must do it
  this way to avoid that the class-initialization protocol of the host
  create these accessors in the host global environment $H$.  Instead,
  \texttt{ensure-class} will create instances of the host class
  \texttt{standard-method} using a function that call the host
  functions \texttt{slot-value} and \texttt{(setf slot-value)} and it
  will add these methods to the appropriate generic function found in
  $R_2$.
\end{itemize}

Phase~1 is divided into two steps:

\begin{enumerate}
\item First, the \texttt{defgeneric} forms corresponding to the
  accessors of the classes of the MOP hierarchy are evaluated using
  $R_1$ as both the compilation environment and run-time environment.
  The result of this step is a set of host generic functions in $R_2$,
  each having no methods.
\item Next, the \texttt{defclass} forms corresponding to the classes
  of the MOP hierarchy are evaluated using $R_1$ as both the
  compilation environment and run-time environment.  The result of
  this step is a set of host classes in $R_1$ and host standard
  methods on the accessor generic functions created in step~1
  specialized to these classes.
\end{enumerate}

\subsubsection{Phase 2}

The purpose of phase~2 is to create a hierarchy of bridge classes that
has the same structure as the hierarchy of MOP classes.

Four different environments are involved in phase~2:

\begin{itemize}
\item The compilation environment $C_2$.
\item The run-time environment $R_1$ is used to look up metaclasses to
  instantiate in order to create the bridge classes.
\item The run-time environment $R_2$ is the one in which bridge
  classes will be associated with names.
\item The run-time environment $R_3$ is the one in which bridge
  generic functions will be associated with names.
\end{itemize}

We create bridge classes corresponding to the classes of the MOP
hierarchy.  When a bridge class is created, it will automatically
create bridge generic functions corresponding to slot readers and
writers.  This is done by calling \texttt{ensure-generic-function} of
phase 1.

Creating bridge classes this way will also instantiate the host class
\texttt{target:direct-slot-definition}, so that our bridge classes
contain host instances bridge in their slots. 

In this phase, we also prepare for the creation of ersatz instances.

\subsubsection{Phase 3}

The purpose of this phase is to create ersatz generic functions and
ersatz classes, by instantiating bridge classes.  

At the end of this phase, we have a set of ersatz instances, some of
which are ersatz classes, except that the \texttt{class} slot of the
header object of every such instance is a bridge class.  We also have
a set of ersatz generic functions (mainly accessors) that are ersatz
instances like all the others. 

\subsubsection{Phase 4}

The first step of this phase is to finalize all the ersatz classes
that were created in phase 3.  Finalization will create ersatz
instances of bridge classes corresponding to effective slot
definitions. 

The second step is to \emph{patch} all the ersatz instances allocated
so far.  There are two different ways in which ersatz instances must
be patched.  First of all, all ersatz instances have a bridge class in
the \texttt{class} slot of the header object.  We patch this
information by accessing the \emph{name} of the bridge class and
replacing the slot contents by the ersatz class with the same name.
Second, ersatz instances of the class \texttt{standard-object} contain
a list of effective slot definition objects in the second word of the
rack, except that those effective slot definition objects
are bridge instances, because they were copied form the
\texttt{class-slots} slot of the bridge class when the bridge class
was instantiated to obtain the ersatz instance.  Since all ersatz
classes were finalized during the first step of this phase, they now
all have a list of effective slot definition objects, and those
objects are ersatz instances.  Patching consists of replacing the
second word of the rack of all instances of
\texttt{standard-object} by the contents of the \texttt{class-slots}
slot of the class object of the instance, which is now a ersatz
class. 

The final step in this phase is to \emph{install} some of the
remaining bridge generic functions so that they are the
\texttt{fdefinition}s of their respective names.  We do not install
all remaining bridge generic functions, because some of them would
clobber host generic functions with the same name that are still
needed.  

At the end of this phase, we have a complete set of bridge generic
functions that operate on ersatz instances.  We still need bridge
classes to create ersatz instances, because the \emph{initfunction}
needs to be executed for slots that require it, and only host
functions are executable at this point.

\subsubsection{Phase 5}

The purpose of this phase is to create ersatz instances for all
objects that are needed in order to obtain a viable image, including: 

\begin{itemize}
\item ersatz built-in classes such as \texttt{package}, \texttt{symbol},
  \texttt{string}, etc., 
\item ersatz instances of those classes, such as the required
  packages, the symbols contained in those packages, the names of
  those symbols, etc.
\item ersatz standard classes for representing the global environment
  and its contents.
\item ersatz instances of those classes.
\end{itemize}

\subsubsection{Phase 6}

The purpose of this phase is to replace all the host instances that
have been used so far as part of the entire ersatz structure, such as
symbols, lists, and integers by their ersatz counterparts.

\subsubsection{Phase 7}

The purpose of this phase is to take the simulated graph of objects
used so far and transfer it to a \emph{memory image}.  

\subsubsection{Phase 8}

Finally, the memory image is written to a binary file. 


\subsection{Stage 2, compiling macro definitions}

Stage 1 of the bootstrapping process consists of using the cross
compiler to compile files containing definitions of standard macros
that will be needed for compiling other files. 

When a \texttt{defmacro} form is compiled by the cross compiler, we
distinguish between the two parts of that defmacro form, namely the
\emph{expander code} and the \emph{resulting expansion code}.  The
\emph{expander code} is the code that will be executed in order to
compute the resulting expansion code when the macro is invoked.  The
\emph{resulting expansion code} is the code that replaces the macro
call form. 

As an example, consider the hypothetical definition of the
\texttt{and} macro shown in \refCode{code-defmacro-and}.

\begin{codefragment}
\inputcode{code-defmacro-and.code}
\caption{\label{code-defmacro-and}
Example implementation of the \texttt{and} macro.}
\end{codefragment}

In \refCode{code-defmacro-and}, the occurrences of \texttt{car},
\texttt{cdr}, \texttt{null}, and \texttt{cond} are part of the
\emph{expander code} whereas the occurrence of \texttt{when}, of
\texttt{t}, and the occurrence of \texttt{and} in the last line are
part of the resulting expansion code. 

The result of expanding the \texttt{defmacro} form in
\refCode{code-defmacro-and} is shown in
\refCode{code-macro-expansion-and}. 

\begin{codefragment}
\inputcode{code-macro-expansion-and.code}
\caption{\label{code-macro-expansion-and}
Expansion of the macro call.}
\end{codefragment}

Thus, when the code in \refCode{code-defmacro-and} is compiled by the
cross compiler, it is first expanded to the code in
\refCode{code-macro-expansion-and}, and the resulting code is compiled
instead.  Now \refCode{code-macro-expansion-and} contains an
\texttt{eval-when} at the top level with all three situations (i.e.,
\texttt{:compile-toplevel}, \texttt{:load-toplevel}, and
\texttt{:execute}.  As a result, two things happen to the
\texttt{funcall} form of \refCode{code-macro-expansion-and}:

\begin{enumerate}
\item It is \emph{evaluated} by the \emph{host function}
  \texttt{eval}.
\item It is \emph{minimally compiled} by the cross compiler.
\end{enumerate}

In order for the evaluation by the host function \texttt{eval} to be
successful, the following must be true:

\begin{itemize}
\item All the \emph{functions} and \emph{macros} that are
  \emph{invoked} as a result of the call to \texttt{eval} must exist.
  In the case of \refCode{code-macro-expansion-and}, the function
  \texttt{(setf sicl-environment::macro-function)} must exist, and that
    is all.
\item All the \emph{macros} that occur in macro forms that are
  \emph{compiled} as a result of the call to \texttt{eval} must
  exist.  These are the macros of the expansion code; in our example
  only \texttt{cond}.  Clearly, if only standard \commonlisp{} macros are
  used in the expansion code of macros, this requirement is
  automatically fulfilled.
\item It is \emph{preferable}, though not absolutely necessary for the
  \emph{functions} that occur in function forms that are
  \emph{compiled} as a result of the call to \texttt{eval} exist.  If
  they do not exist, the compilation will succeed, but a warning will
  probably be issued.  These functions are the functions of the
  expansion code; in our example \texttt{car}, \texttt{cdr}, and
  \texttt{null}.  Again, if only standard \commonlisp{} function are used in
  the expansion code of macros, this requirement is automatically
  fulfilled.  It is common, however, for other functions to be used as
  well.  In that case, those functions should preferably have been
  loaded into the host environment first. 
\end{itemize}

In order for the minimal compilation by the cross compiler to be
successful, the following must be true:

\begin{itemize}
\item All the \emph{macros} that occur in macro forms that are
  \emph{minimally compiled} by the cross compiler must exist.  These
  are again the macros of the expansion code; in our example only
  \texttt{cond}.  Now, the cross compiler uses the \emph{compilation
    environment} of the \emph{target} when looking up macro
  definitions.  Therefore, in order for the example in
  \refCode{code-defmacro-and} to work, a file containing the
  definition of the macro \texttt{cond} must first be compiled by the
  cross compiler. 
\item While it would have been desirable for the \emph{functions} that
  occur in function forms that are \emph{minimally compiled} by the
  cross compiler to exist, this is typically not the case.%
  \fixme{Investigate the possibility of first compiling a bunch of
    \texttt{declaim} forms containing type signatures of most
    standard \commonlisp{} functions used in macro expansion code.}
  As a
  result, the cross compiler will emit warnings about undefined
  functions.  The generated code will still work, however.
\end{itemize}

Of the constraints listed above, the most restrictive is the one that
imposes an order between the files to be cross compiled, i.e., that
the macros of the expansion code must be cross compiled first.  It is
possible to avoid this restriction entirely by using \emph{auxiliary
  functions} rather than macros.  The alternative implementation of
the \texttt{and} macro in \refCode{code-defmacro-and-2} shows how this
is done in the extreme case.

\begin{codefragment}
\inputcode{code-defmacro-and-2.code}
\caption{\label{code-defmacro-and-2}
Alternative implementation of the \texttt{and} macro.}
\end{codefragment}

We use the technique of \refCode{code-defmacro-and-2} only when the
expansion code is fairly complex.  An example of a rather complex
expansion code is that of the macro \texttt{loop} which uses mutually
recursive functions and fairly complex data structures.  When this
technique is used, we can even use a macro to implement its own
expansion code.  For instance, nothing prevents us from using
\texttt{loop} to implement the functions of the expander code of
\texttt{loop}, because when the \texttt{loop} macro is used to expand
code in the cross compiler, the occurrences of \texttt{loop} in the
functions called by the expander code are executed by the host.  As it
turns out, we do not do that, because we would like for the \sysname{}
implementation of \texttt{loop} to be used as a drop-in extension in
implementations other than \sysname{}.


