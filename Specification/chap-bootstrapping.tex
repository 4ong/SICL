\chapter{Bootstrapping}

\section{General technique}

\sysname{} is bootstrapped from an existing \commonlisp{}
implementation that, in addition to the functionality required by the
standard, also contains the library \texttt{closer-mop}.  This
\commonlisp{} system is called the \emph{host}.  The result of the
bootstrapping process is an \emph{image} in the form of an executable
file containing a \sysname{} system.  This system is called the
\emph{target}.  The target image does not contain a complete
\commonlisp{} system.  It only contains enough functionality to load
the remaining system from compiled files.
\seesec{sec-bootstrapping-viable-image}.

In general, the target image can be thought of as containing
\emph{graph} of \commonlisp{} objects that have been placed in memory
according to the spaces managed by the memory manager.  To create this
graph, we first generate an \emph{isomorphic} graph of host objects in
the memory of an executing host system.  To generate the target image,
the isomorphic host graph is traversed, creating a target version of
each object in the host graph, and placing that object on an
appropriate address in the target image.

The isomorphic host graph contains objects that are \emph{analogous}
to their target counterparts as follows:

\begin{itemize}
\item A target \texttt{fixnum} is represented as a host integer.
  Whether the integer is a host fixnum or not depends on the fixnum
  range of the host.
\item A target \texttt{character} is represented as a host character.
\item A target \texttt{cons} cell is represented as a host
  \texttt{cons} cell.
\item A target general instance is represented as a host
  \texttt{standard-object} for the \emph{header} and a host
  \texttt{simple-vector} for the \emph{rack}.
\item Target objects such as bignums or floats are not needed at this
  stage of bootstrapping, so they do not have any representation as
  host objects.
\end{itemize}

\section{Global environments for bootstrapping}

During different stages of bootstrapping, a particular \emph{name} (of
a function, class, etc) must be associated with different objects.  As
a trivial example, the function \texttt{allocate-object} in the host
system is used to allocate all standard objects.  But
\texttt{allocate-object} is also a target function for allocating
objects according to the way such objects are represented by the
target system.  These two functions must be available simultaneously.

Most systems solve this problem by using temporary names for target
packages during the bootstrapping process.  For example, even though
in the final target system, the name \texttt{allocate-object} must be
a symbol in the \texttt{common-lisp} package, during the bootstrapping
process, the name might be a symbol in a package with a different
name.

In \sysname{} we solve the problem by using multiple \emph{first-class
  global environments}.

For the purpose of bootstrapping, it is convenient to think of
\texttt{eval} as consisting of two distinct operations:

\begin{itemize}
\item Compile.  A \emph{compilation environment} is used to expand
  macros and for other compilation purposes.  The result of
  compilation is code that is \emph{untied} to any particular
  environment.
\item Tie.  The untied code produced by the first step is \emph{tied}
  to a particular run-time environment.  Tying is accomplished by
  calling the top-level function created by the compilation.  This
  function takes a single argument, namely a ``function-cell finder''
  function.  Calling that argument function with a function name,
  returns a function cell in a particular environment, thereby tying
  the code to that particular environment.
\end{itemize}

The reason we need to separate these two operations is that for
bootstrapping purposes, the two are going to use distinct global
environments.

\section{Viable image}
\label{sec-bootstrapping-viable-image}

An image I is said to be \emph{viable} if and only if it is possible
to obtain a complete \commonlisp{} system by starting with I and loading a
sequence of ordinary compiled files.

\section{Bootstrapping stages}

\subsection{Stage 1, bootstrapping CLOS}

\subsubsection{Definitions}

\begin{definition}
A \emph{simple instance} is an instance of some class, but that is
also neither a class nor a generic function.
\end{definition}

\begin{definition}
A \emph{host class} is a class in the host system.  If it is an
instance of the host class \texttt{standard-class}, then it is
typically created by the host macro \texttt{defclass}.
\end{definition}

\begin{definition}
A \emph{host instance} is an instance of a host class.  If it is an
instance of the host class \texttt{standard-object}, then it is
typically created by a call to the host function
\texttt{make-instance} using a host class or the name of a host class.
\end{definition}

\begin{definition}
A \emph{host generic function} is a generic function created by the
host macro \texttt{defgeneric}, so it is a host instance of the host
class \texttt{generic-function}.  Arguments to the discriminating
function of such a generic function are host instances.  The host
function \texttt{class-of} is called on some required arguments in
order to determine what methods to call.
\end{definition}

\begin{definition}
A \emph{host method} is a method created by the host macro
\texttt{defmethod}, so it is a host instance of the host class
\texttt{method}.  The class specializers of such a method are host
classes.
\end{definition}

\begin{definition}
A \emph{simple host instance} is a host instance that is neither a
host class nor a host generic function.
\end{definition}

\begin{definition}
An \emph{ersatz instance} is a target instance represented as a host
data structure, using a host standard object to represent the
\emph{header} and a host simple vector to represent the \emph{rack}.
In fact, the header is an instance of the host class
\texttt{funcallable-standard-object} so that some ersatz instances can
be used as functions in the host system.
\end{definition}

\begin{definition}
An ersatz instance is said to be \emph{pure} if the class slot of the
header is also an ersatz instance.  An ersatz instance is said to be
\emph{impure} if it is not pure.  See below for more information on
impure ersatz instances.
\end{definition}

\begin{definition}
An \emph{ersatz class} is an ersatz instance that can be instantiated
to obtain another ersatz instance.
\end{definition}

\begin{definition}
An \emph{ersatz generic function} is an ersatz instance that is also a
generic function.  It is possible for an ersatz generic function be
executed in the host system because the header object is an instance
of the host class \texttt{funcallable-standard-object}.  The methods
on an ersatz generic function are ersatz methods.
\end{definition}

\begin{definition}
An \emph{ersatz method} is an ersatz instance that is also a method.
\end{definition}

\begin{definition}
A \emph{bridge class} is a representation of a target class as a
simple host instance.  An impure ersatz instance has a bridge class in
the class slot of its header.  A bridge class can be instantiated to
obtain an impure ersatz instance.
\end{definition}

\begin{definition}
A \emph{bridge generic function} is a target generic function
represented as a simple host instance, though it is an instance of the
host function \texttt{funcallable-standard-object} so it can be
executed by the host.

Arguments to a bridge generic function are ersatz instances.  The
bridge generic function uses the 
\emph{stamp}
\seesec{sec-generic-function-dispatch-the-discriminating-function} of
the required arguments to dispatch on. 

The methods on a bridge generic function are bridge methods.
\end{definition}

\begin{definition}
A \emph{bridge method} is a target method represented by a simple host
instance.  The class specializers of such a method are bridge classes.
The \emph{method function} of a bridge method is an ordinary host
function.
\end{definition}

\subsubsection{Preparation}

In addition to the host environment, eight different \sysname{}
first-class environments are involved in the bootstrapping procedure.
We shall refer to them as $E_0$, $E_1$, $E_2$, $E_3$, $E_4$, $E_5$,
$E_6$, and $E_7$.

\subsubsection{Phase 0}

We define a class named
\texttt{sicl-boot-phase-0:funcallable-standard-class} in the host
environment.  It is defined as a direct subclass of the host class
\texttt{closer-mop:funcallable-standard-class}.  When we
evaluate \texttt{defclass} forms in phase 1, the classes created are
instances of this class.  We could have chosen this class only for
instances that need to be executable in the host, and a subclass of
the host class \texttt{standard-class} for the others, but the host
class \texttt{funcallable-standard-class} can do everything that the
host class named \texttt{standard-class} can, so we simplify the code by
using one single class.

We define an \texttt{:around} method on \texttt{initialize-instance}
in the host environment, specialized to
\texttt{sicl-boot-phase-0:funcallable-standard-class}.  The purpose of
this \texttt{:around} method is to remove the \texttt{:reader} and
\texttt{:accessor} slot options supplied in the \texttt{defclass}
forms that we evaluate in phase 1.  Without this \texttt{:around}
method, the host function \texttt{initialize-instance} would receive
keyword arguments \texttt{:readers} and \texttt{:writers} and it would
then add methods to host generic functions corresponding to the names
given.  Instead, this \texttt{:around} method adds the readers and
writers to the generic function with the corresponding name in
envirnoment $E_2$.  It assumes that this generic function exists, so
we must create it explicitly before a class that defines the accessor
method can be defined.

In environment $E_0$, we define the following classes:

\begin{itemize}
\item \texttt{t}.  This class is the same as the host class
  \texttt{t}.  It will be used as a specializer on method arguments
  that are not otherwise specialized.
\item \texttt{common-lisp:standard-generic-function}.  This class is
  the same as the class with the same name in the host environment.
  It will be used to create host generic functions in environment
  $E_2$.
\item \texttt{common-lisp:standard-method}.  This class is
  the same as the class with the same name in the host environment.
  This class will be used to create methods on the generic functions
  that we create in environment $E_2$.
\item \texttt{common-lisp:standard-class}.  This class is the same as
  the class named
  \texttt{sicl-boot-phase-0:funcallable-standard-class} in the host
  environment.  It will be used to create most classes in $E_1$.
\item \texttt{common-lisp:built-in-class}.  This class is the same as
  the class named
  \texttt{sicl-boot-phase-0:funcallable-standard-class} in the host
  environment.  It will be used to create some classes in $E_1$, for
  example \texttt{t} and \texttt{function}.
\item \texttt{sicl-clos:funcallable-standard-class}.  This class is
  the same as the class named
  \texttt{sicl-boot-phase-0:funcallable-standard-class} in the host
  environment.  It will be used to create some classes in $E_1$, such
  as \texttt{generic-function} and \texttt{standard-generic-function}.
\item \texttt{sicl-clos:standard-direct-slot-definition}.  This class
  is the same as the class named
  \texttt{closer-mop:standard-direct-slot-definition} in the host
  environment.  This class will be used to create slot-definition
  metaobjects for the classes that we create in environment $E_1$.
\end{itemize}

\subsubsection{Phase 1}

The purpose of phase~1 is:

\begin{itemize}
\item to create host generic functions in $E_2$ corresponding to all
  the accessor functions defined by \sysname{} on standard MOP
  classes, and
\item to create a hierarchy in $E_1$ of host standard classes that has
  the same structure as the hierarchy of MOP classes.
\end{itemize}


Three different environments are involved in phase~1:

\begin{itemize}
\item Environment $E_0$ is used to find find host classes to
  instantiate.  The class named \texttt{standard-class} in $E_0$ is
  used to instantiate most of the classes in environment $E_1$, for
  example, \texttt{standard-class}, \texttt{built-in-class},
  \texttt{slot-definition}, etc.  The class named
  \texttt{built-in-class} in $E_0$ is used to create classes
  \texttt{t} and \texttt{function} as well as some non-MOP classes in
  $E_1$.  The class named \texttt{funcallable-standard-class} in $E_0$
  is used to create classes \texttt{generic-function} and
  \texttt{standard-generic-function}, also in environment $E_1$.  The
  class named \texttt{standard-direct-slot-definition} in $E_0$ is
  used to define slot-definition metaobjects for the classes in
  environment $E_1$.  Environment $E_0$ is also used to find host
  classes \texttt{standard-generic-function} and
  \texttt{standard-method} to instantiate in order to create generic
  functions in environment $E_2$ as well as methods on those generic
  functions.
\item The run-time environment $E_1$ is where instances of the host
  classes named \texttt{standard-class},
  \texttt{funcallable-standard-class}, and \texttt{built-in-class} in
  environment $E_0$ will be associated with the names of the MOP
  hierarchy of classes.  These instances are thus host classes.  The
  entire MOP hierarchy is created as are some built-in classes such as
  \texttt{cons} and some of the number classes.
\item The run-time environment $E_2$ is where instances of the host
  class named \texttt{standard-generic-function} will be associated
  with the names of the different accessors specialized to host
  classes created in $E_1$.
\end{itemize}

One might ask at this point why generic functions are not defined in
the same environment as classes.  The simple answer is that there are
some generic functions that were automatically imported into $E_1$
from the host, that we still need in $E_1$, and that would have been
overwritten by new ones if we had defined new generic functions in
$E_1$.

Several adaptations are necessary in order to accomplish phase~1:

\begin{itemize}
\item A special version of the function
  \texttt{ensure-generic-function} is defined in environment $E_2$.
  It checks whether there is already a function with the name passed
  as an argument in $E_2$, and if so, it returns that function.  It
  makes no verification that such an existing function is really a
  generic function; it assumes that it is.  It also assumes that the
  parameters of that generic function correspond to the arguments of
  \texttt{ensure-generic-function}.  If there is no generic function
  with the name passed as an argument in $E_2$, it creates an instance
  of the host class \texttt{standard-generic-function} and associate
  it with the name in $E_2$.  To create such an instance, it calls the
  host function \texttt{make-instance}.
\item The function \texttt{ensure-class} has a special version in
  $E_1$.  Rather than checking for an existing class, it always
  creates a new one.
\end{itemize}

Phase~1 is divided into two steps:

\begin{enumerate}
\item First, the \texttt{defgeneric} forms corresponding to the
  accessors of the classes of the MOP hierarchy are evaluated using
  $E_1$ as both the compilation environment and run-time environment.
  The result of this step is a set of host generic functions in $E_2$,
  each having no methods.
\item Next, the \texttt{defclass} forms corresponding to the classes
  of the MOP hierarchy are evaluated using $E_1$ as both the
  compilation environment and run-time environment.  The result of
  this step is a set of host classes in $E_1$ and host standard
  methods on the accessor generic functions created in step~1
  specialized to these classes.
\end{enumerate}

\subsubsection{Phase 2}

The purpose of phase~2 is to create a hierarchy of bridge classes that
has the same structure as the hierarchy of MOP classes.

Three different environments are involved in phase~2:

\begin{itemize}
\item The run-time environment $E_1$ is used to look up metaclasses to
  instantiate in order to create the bridge classes.
\item The run-time environment $E_2$ is the one in which bridge
  classes will be associated with names.
\item The run-time environment $E_3$ is the one in which bridge
  generic functions will be associated with names.
\end{itemize}

We create bridge classes corresponding to the classes of the MOP
hierarchy.  When a bridge class is created, it will automatically
create bridge generic functions corresponding to slot readers and
writers.  This is done by calling \texttt{ensure-generic-function} of
phase 1.

Creating bridge classes this way will also instantiate the host class
\texttt{target:direct-slot-definition}, so that our bridge classes
contain host instances bridge in their slots. 

In this phase, we also prepare for the creation of ersatz instances.

\subsubsection{Phase 3}

The purpose of this phase is to create ersatz generic functions and
ersatz classes, by instantiating bridge classes.  

At the end of this phase, we have a set of ersatz instances, some of
which are ersatz classes, except that the \texttt{class} slot of the
header object of every such instance is a bridge class.  We also have
a set of ersatz generic functions (mainly accessors) that are ersatz
instances like all the others. 

\subsubsection{Phase 4}

The first step of this phase is to finalize all the ersatz classes
that were created in phase 3.  Finalization will create ersatz
instances of bridge classes corresponding to effective slot
definitions. 

The second step is to \emph{patch} all the ersatz instances allocated
so far.  There are two different ways in which ersatz instances must
be patched.  First of all, all ersatz instances have a bridge class in
the \texttt{class} slot of the header object.  We patch this
information by accessing the \emph{name} of the bridge class and
replacing the slot contents by the ersatz class with the same name.
Second, ersatz instances of the class \texttt{standard-object} contain
a list of effective slot definition objects in the second word of the
rack, except that those effective slot definition objects
are bridge instances, because they were copied form the
\texttt{class-slots} slot of the bridge class when the bridge class
was instantiated to obtain the ersatz instance.  Since all ersatz
classes were finalized during the first step of this phase, they now
all have a list of effective slot definition objects, and those
objects are ersatz instances.  Patching consists of replacing the
second word of the rack of all instances of
\texttt{standard-object} by the contents of the \texttt{class-slots}
slot of the class object of the instance, which is now a ersatz
class. 

The final step in this phase is to \emph{install} some of the
remaining bridge generic functions so that they are the
\texttt{fdefinition}s of their respective names.  We do not install
all remaining bridge generic functions, because some of them would
clobber host generic functions with the same name that are still
needed.  

At the end of this phase, we have a complete set of bridge generic
functions that operate on ersatz instances.  We still need bridge
classes to create ersatz instances, because the \emph{initfunction}
needs to be executed for slots that require it, and only host
functions are executable at this point.

\subsubsection{Phase 5}

The purpose of this phase is to create ersatz instances for all
objects that are needed in order to obtain a viable image, including: 

\begin{itemize}
\item ersatz built-in classes such as \texttt{package}, \texttt{symbol},
  \texttt{string}, etc., 
\item ersatz instances of those classes, such as the required
  packages, the symbols contained in those packages, the names of
  those symbols, etc.
\item ersatz standard classes for representing the global environment
  and its contents.
\item ersatz instances of those classes.
\end{itemize}

\subsubsection{Phase 6}

The purpose of this phase is to replace all the host instances that
have been used so far as part of the entire ersatz structure, such as
symbols, lists, and integers by their ersatz counterparts.

\subsubsection{Phase 7}

The purpose of this phase is to take the simulated graph of objects
used so far and transfer it to a \emph{memory image}.  

\subsubsection{Phase 8}

Finally, the memory image is written to a binary file. 


\subsection{Stage 2, compiling macro definitions}

Stage 1 of the bootstrapping process consists of using the cross
compiler to compile files containing definitions of standard macros
that will be needed for compiling other files. 

When a \texttt{defmacro} form is compiled by the cross compiler, we
distinguish between the two parts of that defmacro form, namely the
\emph{expander code} and the \emph{resulting expansion code}.  The
\emph{expander code} is the code that will be executed in order to
compute the resulting expansion code when the macro is invoked.  The
\emph{resulting expansion code} is the code that replaces the macro
call form. 

As an example, consider the hypothetical definition of the
\texttt{and} macro shown in \refCode{code-defmacro-and}.

\begin{codefragment}
\inputcode{code-defmacro-and.code}
\caption{\label{code-defmacro-and}
Example implementation of the \texttt{and} macro.}
\end{codefragment}

In \refCode{code-defmacro-and}, the occurrences of \texttt{car},
\texttt{cdr}, \texttt{null}, and \texttt{cond} are part of the
\emph{expander code} whereas the occurrence of \texttt{when}, of
\texttt{t}, and the occurrence of \texttt{and} in the last line are
part of the resulting expansion code. 

The result of expanding the \texttt{defmacro} form in
\refCode{code-defmacro-and} is shown in
\refCode{code-macro-expansion-and}. 

\begin{codefragment}
\inputcode{code-macro-expansion-and.code}
\caption{\label{code-macro-expansion-and}
Expansion of the macro call.}
\end{codefragment}

Thus, when the code in \refCode{code-defmacro-and} is compiled by the
cross compiler, it is first expanded to the code in
\refCode{code-macro-expansion-and}, and the resulting code is compiled
instead.  Now \refCode{code-macro-expansion-and} contains an
\texttt{eval-when} at the top level with all three situations (i.e.,
\texttt{:compile-toplevel}, \texttt{:load-toplevel}, and
\texttt{:execute}.  As a result, two things happen to the
\texttt{funcall} form of \refCode{code-macro-expansion-and}:

\begin{enumerate}
\item It is \emph{evaluated} by the \emph{host function}
  \texttt{eval}.
\item It is \emph{minimally compiled} by the cross compiler.
\end{enumerate}

In order for the evaluation by the host function \texttt{eval} to be
successful, the following must be true:

\begin{itemize}
\item All the \emph{functions} and \emph{macros} that are
  \emph{invoked} as a result of the call to \texttt{eval} must exist.
  In the case of \refCode{code-macro-expansion-and}, the function
  \texttt{(setf sicl-environment::macro-function)} must exist, and that
    is all.
\item All the \emph{macros} that occur in macro forms that are
  \emph{compiled} as a result of the call to \texttt{eval} must
  exist.  These are the macros of the expansion code; in our example
  only \texttt{cond}.  Clearly, if only standard \commonlisp{} macros are
  used in the expansion code of macros, this requirement is
  automatically fulfilled.
\item It is \emph{preferable}, though not absolutely necessary for the
  \emph{functions} that occur in function forms that are
  \emph{compiled} as a result of the call to \texttt{eval} exist.  If
  they do not exist, the compilation will succeed, but a warning will
  probably be issued.  These functions are the functions of the
  expansion code; in our example \texttt{car}, \texttt{cdr}, and
  \texttt{null}.  Again, if only standard \commonlisp{} function are used in
  the expansion code of macros, this requirement is automatically
  fulfilled.  It is common, however, for other functions to be used as
  well.  In that case, those functions should preferably have been
  loaded into the host environment first. 
\end{itemize}

In order for the minimal compilation by the cross compiler to be
successful, the following must be true:

\begin{itemize}
\item All the \emph{macros} that occur in macro forms that are
  \emph{minimally compiled} by the cross compiler must exist.  These
  are again the macros of the expansion code; in our example only
  \texttt{cond}.  Now, the cross compiler uses the \emph{compilation
    environment} of the \emph{target} when looking up macro
  definitions.  Therefore, in order for the example in
  \refCode{code-defmacro-and} to work, a file containing the
  definition of the macro \texttt{cond} must first be compiled by the
  cross compiler. 
\item While it would have been desirable for the \emph{functions} that
  occur in function forms that are \emph{minimally compiled} by the
  cross compiler to exist, this is typically not the case.%
  \fixme{Investigate the possibility of first compiling a bunch of
    \texttt{declaim} forms containing type signatures of most
    standard \commonlisp{} functions used in macro expansion code.}
  As a
  result, the cross compiler will emit warnings about undefined
  functions.  The generated code will still work, however.
\end{itemize}

Of the constraints listed above, the most restrictive is the one that
imposes an order between the files to be cross compiled, i.e., that
the macros of the expansion code must be cross compiled first.  It is
possible to avoid this restriction entirely by using \emph{auxiliary
  functions} rather than macros.  The alternative implementation of
the \texttt{and} macro in \refCode{code-defmacro-and-2} shows how this
is done in the extreme case.

\begin{codefragment}
\inputcode{code-defmacro-and-2.code}
\caption{\label{code-defmacro-and-2}
Alternative implementation of the \texttt{and} macro.}
\end{codefragment}

We use the technique of \refCode{code-defmacro-and-2} only when the
expansion code is fairly complex.  An example of a rather complex
expansion code is that of the macro \texttt{loop} which uses mutually
recursive functions and fairly complex data structures.  When this
technique is used, we can even use a macro to implement its own
expansion code.  For instance, nothing prevents us from using
\texttt{loop} to implement the functions of the expander code of
\texttt{loop}, because when the \texttt{loop} macro is used to expand
code in the cross compiler, the occurrences of \texttt{loop} in the
functions called by the expander code are executed by the host.  As it
turns out, we do not do that, because we would like for the \sysname{}
implementation of \texttt{loop} to be used as a drop-in extension in
implementations other than \sysname{}.


