\chapter{Object system}

\sysname{} will implement the full metaobject protocol (MOP) as
described by the Art of the Metaobject Protocol (AMOP)
\cite{Kiczales:1991:AMP:574212}, in as far it does not conflict with
the \hs{}.

\section{Classes of class metaobjects}

The AMOP stipulates the existence of four class metaclasses, namely:

\begin{itemize}
\item \texttt{standard-class}.  This is the default metaclass for classes
  created by \texttt{defclass}.  It is also the metaclass for all
  classes in the metaobject protocol except \texttt{t},
  \texttt{function}, \texttt{generic-function}, and
  \texttt{standard-generic-function}. 
\item \texttt{funcallable-standard-class}.  This is the metaclass for
  \texttt{generic-function}, and \texttt{standard-generic-function},
  and of course for user-defined subclasses of those classes. 
\item \texttt{built-in-class}.  This is the metaclass for all built-in
  classes.  More about built-in classes in
  \refSec{object-system-built-in-classes}. 
\item \texttt{forward-reference-class}.  This is the metaclass for
  classes that have been referred to as superclasses, but that have
  not yet been created by \texttt{defclass}.
\end{itemize}

\subsection{Standard classes}
\label{object-system-standard-classes}

Instances of \texttt{standard-class} (i.e. ``standard classes'') are
typically created by \texttt{defclass}.  When no superclasses are
given to \texttt{defclass}, the class \texttt{standard-object} is
automatically made a superclass of the new class.  

Perhaps the most interesting features of standard classes is that they
can be redefined even though there are existing instances of them.
Without this features, using \cl{} interactively would not be as
obvious as it is, so in some ways, this features is totally essential
for any interactive language.%
\footnote{By ``interactive language'', we mean a language in which a
  program is built up by a sequence of interactions that augment and
  modify the state of some \emph{global environment}.}

When there are existing instances of a standard class that is
modified, the \hs{} gives us very specific rules concerning how
those instances are to be updated.  The \hs{} is also very clear
that existing instances do not need to be updated immediately.  But
they must be updated no later than immediately before an access to any
of the slots of the instance is attempted.   The reason for that rule
is so that implementations would not have to maintain a reference from
a class to each of its instance.  Such references would be costly in
terms of space, and would have to be \emph{weak}%
\footnote{A \emph{weak} reference is a reference that does not
  sufficient for the garbage collector to keep the object alive.  The
  object is kept alive only if there are \emph{strong} (i.e. normal)
  references to it as well.}
so as to avoid memory leaks. 

Instead of keeping weak references from classes to instances,
implementations solve the problem of updating obsolete instance by
keeping some kind of \emph{version information} in each instance.
When an access to some slot of the instance is attempted, the version
information is checked against the current version of the class.  If
the instance is obsolete, it is first updated according to the new
definition of the class.  Furthermore, the version information must
contain enough information of the class as it was when the instance
was created to determine whether slots have been added or removed. 

In \sysname{} the contents vector of each instance of
\texttt{standard-object} contains a reference to the \emph{list of
  effective slots} of the class as it was when the instance was
created.  This method makes the contents vector of the instance
completely self contained.  It allows the garbage collector to trace
the obsolete instance, or update it before tracing.  It allows for an
inspector to inspect the obsolete instance if this should be
required.  

Another very handy feature of standard classes, but a much simpler
one, is that it allows for instances to \emph{change class}.  In other
words, without changing the \emph{identity} of the instance, the class
that is an instance of can be changed to a different class.  Again,
the \hs{} gives very specific rules about how the instance must be
transformed in order to be a legitimate instance of the new class.  No
special mechanism is required for this feature to work, other than the
ability to modify all aspects of an instance except its identity.  The
identity is preserved by the fact that the \emph{header object}
remains the same. 

\subsection{Built-in classes}
\label{object-system-built-in-classes}

The \hs{} contains a significant number of classes that every
conforming implementation must contain.  Most (all?) of these classes
are referred to as \emph{system classes}.  Some example of system
classes are \texttt{symbol}, \texttt{package}, \texttt{list},
\texttt{stream}, etc.  The \hs{} tells us that by \emph{system
  class} is meant ``a class that may be of type built-in-class in a
conforming implementation and hence cannot be inherited by classes
defined by conforming programs.'' 

Language implementers are thus given a choice as to whether a system
class is really a standard class
\seesec{object-system-standard-classes}, a structure class 
\seesec{object-system-built-in-classes}, or a built-in class.  

Some of the decisions are determined by the AMOP.  For instance, the
classes \texttt{standard-class} and \texttt{built-in-class}, labeled
by the \hs{} as system classes are required by the AMOP to be
standard classes.  Any implementation that wants to have an
implementation of the metaobject protocol as close as possible to what
the AMOP requires should take this fact into account. 

However, most system classes are not mentioned at all by the AMOP, so
there we have a choice.  In \sysname{} all these classes will be
implemented as either built-in classes or standard classes.  None of
them will be structure classes. 

Even though \sysname{} will implement many of the system classes as
built-in classes, this does not mean that we have to use
special-purpose ways of implementing them.  We plan to take advantage
of features of the metaobject protocol such as inheritance and class
finalization to define our built-in classes.  

Even for system classes where instances do not all have the same size,
notably the \texttt{array} class and its subclasses, we plan to take
advantage of the metaobject protocol by creating a function named
\texttt{make-built-in-instance} that takes a \emph{size} argument in
addition to ordinary initialization arguments.  We also plan to create
a macro named \texttt{define-built-in-class} similar to
\texttt{defclass}, and with the main difference that the default
superclass is \texttt{t} instead of \texttt{standard-object}.  This
method allows us to concentrate all important features of a built-in
class and its instances in one place, which will simplify
maintenance. 

\subsection{Condition classes}
\label{object-system-condition-classes}

The \hs{} defines an entire hierarchy of classes with the class
\texttt{condition} as the root class.  This hierarchy is not mentioned
by the AMOP.  

We plan to implement this hierarchy by defining a class named
\texttt{condition-class} analogous to \texttt{standard-class} for
standard objects.  The class \texttt{condition} plays a role analogous
to the class \texttt{standard-object} for instances of
\texttt{standard-class}. 

As with built-in classes \seesec{object-system-built-in-classes}, we
plan to take advantage of the very complete set of tools provided by
the metaobject protocol to implement condition classes.  In
particular, we want to allow for condition classes to be redefined
even though existing instances may be present, just the way instances
of \texttt{standard-object} may exist even though the class is being
modified.  However, we do not plan to make it possible for an instance
of a condition class to have its class changed (i.e. by using
\texttt{change-class}).  

With respect to bootstrapping, the hierarchy of condition classes can
be created fairly late in the process.  The reason for this is that we
plan to define a dumbed-down version of \texttt{error} during the
bootstrapping process, and that version will not create any condition
instances.  Furthermore, all \sysname{} code calls \texttt{error} and
the other condition-signaling functions with the \emph{name} of a
condition (which is a symbol) rather than with a \emph{condition
  instance}, again to allow us to create this hierarchy later in the
bootstrapping process. 

\subsection{Structure classes}
\label{object-system-structure-classes}

Just like condition classes \seesec{object-system-condition-classes},
structure classes are not mentioned at all in the AMOP.  In addition,
their description in the \hs{} is limited to the dictionary entry
for \texttt{defstruct}.  

No part of \sysname{} uses structure classes.  The main reason is that
they are difficult to work with due to the fact that conforming
implementations are allowed to make it impossible to redefine
existing structure classes.%
\footnote{I seem to remember reading somewhere that implementers are
  encouraged to make it possible to modify existing structure classes
  in the same way that it is possible to modify standard classes even
  though there are existing instances, but I don't remember where I
  read this, and I can't seem to find the place.  It might have been
  in CLtL2 rather than the \hs{}, but I can't find it there
  either.  Oh well, when I find it, I will remove this footnote.}

However, since we plan for \sysname{} to be a conforming
implementation, we naturally plan to include structure classes as
well. 

The main reason for using structure classes rather than standard
classes is \emph{performance}.  Structure classes are supposed to be
implemented in the ``most efficient way possible''.%
\footnote{Again, I forget where I read this, and I can't find it.}
Presumably, the restrictions on structure classes exist to allow for
an implementation to represent instances as a pointer directly to the
vector of slots and avoid any indirection, which saves a memory
access.  However, in \sysname{} \emph{all} heap-allocated objects
(other than \texttt{cons} cells) are represented as a two-word
\emph{header object} and a \emph{contents vector} for reasons of
simplicity and in order to allow our memory-management strategy to
work.

Since \sysname{} represents instances of structure classes this way,
there is no reason to keep the restriction that structure classes can
not be modified.  For that reason, we plan to avoid that restriction.

Structure classes have another interesting restriction, namely that
they allow only single inheritance.  This restriction allows slot
accessors to be non-generic, because it becomes possible for a slot to
have the same physical position in all subclasses.  We may or may not
take advantage of this possibility.  The higher priority for
\sysname{} is to make accessors for standard objects fast, rather than
to work on an efficient implementation of structures. 

With respect to bootstrapping, since \sysname{} does not use structure
classes at all for its implementation, implementing \texttt{defstruct}
can be done fairly late.  In fact, we may omit it in initial version
of the system. 

\section{Implementing \texttt{slot-value} and \texttt{(setf slot-value)}}

The functions \texttt{slot-value} and \texttt{(setf slot-value)} play
a central role in the object system, because they are used by slot
accessor functions, either directly or indirectly to determine a
\emph{slot location} in an instance to be used.  

In \emph{the Art of the Metaobject Protocol (AMOP)}
\cite{Kiczales:1991:AMP:574212}, these functions are mentioned as
prime examples of issues of \emph{metastability}.  The scenario that
is cited is that \texttt{slot-value} of an instance calls
\texttt{class-slots} on the class of the instance, and
\texttt{class-slots} is an accessor which calls \texttt{slot-value}.
The metastability problem is resolved by recognizing that the
recursion must eventually reach \texttt{standard-class}, so that
treating \texttt{standard-class} as a special case resolves the
problem.

However, the scenario cited above represents a simplification of the
real one.  The specification requires \texttt{slot-value} to call
\texttt{slot-value-using-class} with the instance, the class of the
instance, and an \emph{effective slot definition metaobject}.  In
order for \texttt{slot-value} to find the right effective slot
definition metaobject, it has to traverse the list of effective slot
definition metaobjects until one is found that has the \emph{name} of
the slot given as an argument.  To find the name of an effective slot
definition metaobject, \texttt{slot-value} has to call
\texttt{slot-value} which is again a metastability issue. 

We solve the problems indicated above%
\fixme{We still need to prove that these special cases actually solve
  the problem by preventing an infinite recursion.}
by recognizing as special cases
the following argument combinations to \texttt{slot-value}:

\begin{itemize}
\item The slot name is the name of the slot that stores the value
  returned by \texttt{class-slots} when applied to the class named
  \texttt{standard-class}, and the class of the instance is the class
  named \texttt{standard-class}.
\item The slot name is the name of the slot that stores the value
  returned by \texttt{slot-definition-name} when applied to an
  instance of the class \texttt{standard-effective-slot-definition},
  and the class of the instance is the class named
  \texttt{standard-effective-slot-definition}. 
\item The slot name is the name of the slot that stores the value
  returned by \texttt{slot-definition-location} when applied to an
  instance of the class \texttt{standard-effective-slot-definition},
  and the class of the instance is the class named
  \texttt{standard-effective-slot-definition}. 
\end{itemize}

To handle the special cases, we store the classes named
\texttt{standard-class} and
\texttt{standard-effective-slot-definition} in global variables, and
we store the \emph{slot location} (as returned by
\texttt{slot-definition-location}) of the relevant slots, also in
global variables.  The special case is handled by calling
\texttt{standard-instance-access} with the instance and the relevant
location.

With these special cases, it is now possible to compute the
\emph{method function} of any reader by calling the function
\texttt{make-reader-method-function} shown in
\refCode{code-make-reader}.  However, implementing readers this way is
very slow, so this method is used only for bootstrapping purposes.

\begin{codefragment}
\inputcode{code-make-reader.code}
\caption{\label{code-make-reader}
Function for creating method functions for readers.}
\end{codefragment}

When none of the special cases apply, \texttt{slot-value} first calls
\texttt{class-slots} of the class of the instance.  It then calls
\texttt{find} to find the correct effective slot definition
metaobject, passing it the class name and the \texttt{key} function
being \texttt{(function slot-definition-name)} which is a reader.
Finally, it calls \texttt{slot-value-using-class} as required by the
specification.

The standard methods on \texttt{slot-value-using-class} call the
reader function \texttt{slot-definition-location} on the effective
slot definition and use that location to call
\texttt{standard-instance-access} on the instance.

The cases of \texttt{(setf slot-value)} and \texttt{(setf
  slot-value-using-class)} are handled in the analogous way.

\section{Bootstrapping}

\subsection{Definitions}

\begin{definition}
A \emph{host class} is a class in the host system.  If it is an
instance of the host class \texttt{host:standard-class}, then it is
typically created by \texttt{host:defclass}.
\end{definition}

\begin{definition}
A \emph{host instance} is an instance of a host class.  If it is an
instance of the class \texttt{host:standard-object}, then it is
typically created by a call to \texttt{host:make-instance} using a
host class or the name of a host class.
\end{definition}

\begin{definition}
A \emph{target class} is a class metaobject represented as a target
instance of some target class. 
\end{definition}

\begin{definition}
A \emph{target instance} is an instance of a target class.
\end{definition}

\begin{definition}
A \emph{bridge class} is a representation of a target class as a host
instance.
\end{definition}

\begin{definition}
A \emph{bridge instance} is a target instance, except that the class of
the instance is not a target class but a bridge class.
\end{definition}

Notice that according to this definition, a bridge instance is not an
instance of a bridge class.

\begin{definition}
A \emph{host generic function} is a generic function created by
\texttt{host:defgeneric}, so it is a host instance of the host class
\texttt{host:generic-function}.  Arguments to the discriminating
function of such a generic function are host instances.
\texttt{host:class-of} is called on some required arguments in order
to determine what methods to call.
\end{definition}

\begin{definition}
A \emph{host method} is a method created by \texttt{host:defmethod},
so it is a host instance of the host class \texttt{host:method}.  The
class specializers of such a method are host classes.
\end{definition}

\begin{definition}
A \emph{target generic function} is an instance of the target class
\texttt{target:generic-function}.  Arguments to the discriminating
function of such a generic function are target instances.
\texttt{target:class-of} is called on some required arguments in order
to determine what methods to call, and the function
\texttt{target:class-unique-number} is called to determine the unique
number of the class.
\end{definition}

\begin{definition}
A \emph{target method} is an instance of the target class
\texttt{target:method}.  The class specializers of such a method are
target classes.
\end{definition}

\begin{definition}
A \emph{bridge generic function} is a generic function which is a host
instance of the host class \texttt{target:generic-function}.  Arguments to
the discriminating function of such a generic function are
typically target instances (but not necessarily, because of our
definition of "target instance").  \texttt{target:class-of} is called on
some required arguments in order to determine what methods to
call, and \texttt{target:class-unique-number} is called to determine the
unique number of the class.
\end{definition}

\begin{definition}
A \emph{bridge method} is a a method which is a host instance of the host
class \texttt{target:method}.  The class specializers of such a method can
be bridge classes or target classes depending on context, but the
class accessors (\texttt{target:class-direct-superclasses}, etc.) as well
as \texttt{target:class-unique-number} must be defined in that context to
work on the specializers. 
\end{definition}

\subsection{Bootstrapping stages}

\subsubsection{Stage 1}

We use \texttt{host:defclass} to create a hierarchy of host classes
corresponding to the hierarchy of MOP classes.  The names of those host
classes will be \texttt{target:standard-object},
\texttt{target:metaobject}, etc., except that we do not create a class
named \texttt{T} because that would require shadowing the symbol
\texttt{T} which would create many other problems.  It is however not
required to have \texttt{T} as a superclass for these classes.  All
that matters is that these classes have the right slots, and
\texttt{T} does not supply any slots anyway.  These host classes have
the analogous superclasses compared their corresponding target MOP
classes, and they define the analogous slots.  Their metaclasses are
all wrong of course, because each one of them has
\texttt{host:standard-class} as metaclass.  The inheritance
finalization protocol of the host will make sure that instances of
these host classes have the effective slots determined by class
inheritance.  In other words, each of the classes define slots with
the right names, the right initargs, etc.

\subsubsection{Stage 2}

The purpose of this stage is to make it possible to create bridge
generic functions.  In order to create a generic function, we need the
existence of other generic functions.  We break this cycle by using
host generic functions to create bridge generic functions.  Examples 
of host generic functions that are defined in this stage are
\texttt{ensure-generic-function-using-class},
\texttt{compute-applicable-methods-using-classes},
\texttt{compute-discriminating-function}, etc.  Furthermore, in this
stage, we define \texttt{:after} method on
\texttt{host:initialize-instance} specialized for the host classes
\texttt{target:generic-function} and \texttt{target:method}.  

\subsubsection{Stage 3}

The purpose of this stage is to create bridge classes corresponding to
the classes of the MOP hierarchy.  We do this by redefining
\texttt{defclass} to call \texttt{target:ensure-class} which calls
\texttt{host:make-instance} of the appropriate host class.  When a
bridge class is created, it will automatically create generic
functions corresponding to slot readers and writers.  This is done by
calling \texttt{host:make-instance} directly or indirectly through
\texttt{ensure-generic-function}.  Thus the generic functions created
this way are bridge generic functions, and the methods are bridge
methods.

Creating bridge classes this way will also instantiate the host class
\texttt{target:direct-slot-definition}, so that our bridge classes
contain bridge instances in their slots. 

Before creating the bridge classes, we must define \texttt{:after}
methods on \texttt{host:initialize-instance} specialized for the host
classes \texttt{target:standard-class} and
\texttt{target:funcallable-standard-class}.  

\subsubsection{Stage 4}

