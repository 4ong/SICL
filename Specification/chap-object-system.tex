\chapter{Object system}

\sysname{} will implement the full metaobject protocol (MOP) as
described by the Art of the Metaobject Protocol (AMOP)
\cite{Kiczales:1991:AMP:574212}, in as far it does not conflict with
the \hs{}.

\section{Classes of class metaobjects}

The AMOP stipulates the existence of four class metaclasses, namely:

\begin{itemize}
\item \href{http://www.metamodular.com/CLOS-MOP/class-standard-class.html}
  {\texttt{standard-class}}.
  This is the default metaclass for classes
  created by \texttt{defclass}.  It is also the metaclass for all
  classes in the metaobject protocol except \texttt{t},
  \texttt{function}, \texttt{generic-function}, and
  \texttt{standard-generic-function}. 
\item \href{http://www.metamodular.com/CLOS-MOP/class-funcallable-standard-class.html}
{\texttt{funcallable-standard-class}}.
This is the metaclass for
\href{http://www.metamodular.com/CLOS-MOP/class-generic-function.html}
{\texttt{generic-function}}, 
and 
\href{http://www.metamodular.com/CLOS-MOP/class-standard-generic-function.html}
{\texttt{standard-generic-function}},
and of course for user-defined subclasses of those classes. 
\item \href{http://www.metamodular.com/CLOS-MOP/class-built-in-class.html}
{\texttt{built-in-class}}.  
This is the metaclass for all built-in
classes.  More about built-in classes in
\refSec{object-system-built-in-classes}. 
\item \href{http://www.metamodular.com/CLOS-MOP/class-forward-referenced-class.html}
{\texttt{forward-referenced-class}}.  This is the metaclass for
  classes that have been referred to as superclasses, but that have
  not yet been created by \texttt{defclass}.
\end{itemize}

In addition, the \hs{} requires the existence of a class metaclass
named \texttt{structure-class}.  Whether conditions have their own
metaclass is not specified. 

In \sysname{}, every class (i.e., ever instance of a class metaclass)
contains a \emph{unique number} which is an integer assigned
sequentially from $0$ as a class is created or modified.  When a class
is modified, its old unique number is never reused, leaving
\emph{holes} in the sequence corresponding to numbers that no longer
correspond to any classes.%
\footnote{It might be possible for the garbage collector to change the
  unique numbers of the classed, compacting the sequence, but that
  probably will not be necessary.}

Every \emph{general instance}%
\footnote{Recall that a \emph{general instance} is an instance
  allocated on the heap, excluding \texttt{cons}
  cells. \seechap{chap-data-representation}} contains a \emph{stamp},
which is the unique number of its class as it was when the instance
was created.  This number is always the first element of the rack of
the general instance.  Even though built-in classes can not be
redefined, general instances of built-in classes contain a stamp.

\subsection{Standard classes}
\label{object-system-standard-classes}

Instances of \texttt{standard-class} (i.e. ``standard classes'') are
typically created by \texttt{defclass}.  When no superclasses are
given to \texttt{defclass}, the class \texttt{standard-object} is
automatically made a superclass of the new class.  

Perhaps the most interesting features of standard classes is that they
can be redefined even though there are existing instances of them.
Without this features, using \commonlisp{} interactively would not be as
obvious as it is, so in some ways, this features is totally essential
for any interactive language.%
\footnote{By ``interactive language'', we mean a language in which a
  program is built up by a sequence of interactions that augment and
  modify the state of some \emph{global environment}.}

When there are existing instances of a standard class that is
modified, the \hs{} gives us very specific rules concerning how
those instances are to be updated.  The \hs{} is also very clear
that existing instances do not need to be updated immediately.  But
they must be updated no later than immediately before an access to any
of the slots of the instance is attempted.   The reason for that rule
is so that implementations would not have to maintain a reference from
a class to each of its instance.  Such references would be costly in
terms of space, and would have to be \emph{weak}%
\footnote{A \emph{weak} reference is a reference that is not
  sufficient for the garbage collector to keep the object alive.  The
  object is kept alive only if there is at least one \emph{strong}
  (i.e. normal) reference to it as well.}  so as to avoid memory
leaks.

Instead of keeping weak references from classes to instances,
implementations solve the problem of updating obsolete instance by
keeping some kind of \emph{version information} in each instance.
When some operation on the instance is attempted, the version
information is checked against the current version of the class.  If
the instance is obsolete, it is first updated according to the new
definition of the class.  Furthermore, the version information must
contain enough information of the class as it was when the instance
was created to determine whether slots have been added or removed.

In \sysname{} the first location of the rack of each
general instance contains a \emph{stamp}, which is the unique number
of the class as it was when the instance was created.  In addition,
if the instance is \emph{flexible}, it also contains a reference to the
\emph{list of effective slots} of the class as it was when the
instance was created.  This method makes the rack of the
instance completely self contained.  It allows the garbage collector
to trace the obsolete instance, or update it before tracing.  It
allows for an inspector to inspect the obsolete instance if this
should be required.  The main purpose of the list of effective slots,
however, is making it possible to update an obsolete instance.

Another very handy feature of standard classes, but a much simpler
one, is that it allows for instances to \emph{change class}.  In other
words, without changing the \emph{identity} of the instance, the class
that is an instance of can be changed to a different class.  Again,
the \hs{} gives very specific rules about how the instance must be
transformed in order to be a legitimate instance of the new class.  No
special mechanism is required for this feature to work, other than the
ability to modify all aspects of an instance except its identity.  The
identity is preserved by the fact that the \emph{header object}
remains the same. 

\subsection{Built-in classes}
\label{object-system-built-in-classes}

The \hs{} contains a significant number of classes that every
conforming implementation must contain.  Most (all?) of these classes
are referred to as \emph{system classes}.  Some example of system
classes are \texttt{symbol}, \texttt{package}, \texttt{list},
\texttt{stream}, etc.  The \hs{} tells us that by \emph{system
  class} is meant ``a class that may be of type built-in-class in a
conforming implementation and hence cannot be inherited by classes
defined by conforming programs.'' 

Language implementers are thus given a choice as to whether a system
class is really a standard class
\seesec{object-system-standard-classes}, a structure class 
\seesec{object-system-structure-classes}, or a built-in class.

Some of the decisions are determined by the AMOP.  For instance, the
classes \texttt{standard-class} and \texttt{built-in-class}, labeled
by the \hs{} as system classes are required by the AMOP to be
standard classes.  Any implementation that wants to have an
implementation of the metaobject protocol as close as possible to what
the AMOP requires should take this fact into account. 

However, most system classes are not mentioned at all by the AMOP, so
there we have a choice.  In \sysname{} all these classes will be
implemented as either built-in classes or standard classes.  None of
them will be structure classes. 

Even though \sysname{} will implement many of the system classes as
built-in classes, this does not mean that we have to use
special-purpose ways of implementing them.  The \sysname{} object
system takes advantage of features of the metaobject protocol such as
inheritance and class finalization to define built-in classes as well
as standard classes.

Even for system classes where instances do not all have the same size,
notably the \texttt{array} class and its subclasses, we plan to take
advantage of the metaobject protocol by allowing
\texttt{make-instance} to take a \emph{size} argument in addition to
ordinary initialization arguments.  We also plan to allow
\texttt{defclass}, to define built-in classes by passing it
\texttt{built-in-class} as a metaclass.  In that case, the default
superclass is \texttt{t} instead of \texttt{standard-object}.  This
technique allows us to concentrate all important features of a
built-in class and its instances in one place, which will simplify
maintenance.

\subsection{Condition classes}
\label{object-system-condition-classes}

The \hs{} defines an entire hierarchy of classes with the class
\texttt{condition} as the root class.  This hierarchy is not mentioned
by the AMOP.  

We plan to implement this hierarchy by defining a class named
\texttt{condition-class} analogous to \texttt{standard-class} for
standard objects.  The class \texttt{condition} plays a role analogous
to the class \texttt{standard-object} for instances of
\texttt{standard-class}. 

As with built-in classes \seesec{object-system-built-in-classes}, we
plan to take advantage of the very complete set of tools provided by
the metaobject protocol to implement condition classes.  In
particular, we want to allow for condition classes to be redefined
even though existing instances may be present, just the way instances
of \texttt{standard-object} may exist even though the class is being
modified.  However, we do not plan to make it possible for an instance
of a condition class to have its class changed (i.e. by using
\texttt{change-class}).  

With respect to bootstrapping, the hierarchy of condition classes can
be created fairly late in the process.  The reason for this is that we
plan to define a dumbed-down version of \texttt{error} during the
bootstrapping process, and that version will not create any condition
instances.  Furthermore, all \sysname{} code calls \texttt{error} and
the other condition-signaling functions with the \emph{name} of a
condition (which is a symbol) rather than with a \emph{condition
  instance}, again to allow us to create this hierarchy later in the
bootstrapping process. 

\subsection{Structure classes}
\label{object-system-structure-classes}

Just like condition classes \seesec{object-system-condition-classes},
structure classes are not mentioned at all in the AMOP.  In addition,
their description in the \hs{} is limited to the dictionary entry
for \texttt{defstruct}.  

No part of \sysname{} uses structure classes.  The main reason is that
they are difficult to work with due to the fact that conforming
implementations are allowed to make it impossible to redefine
existing structure classes.%
\footnote{I seem to remember reading somewhere that implementers are
  encouraged to make it possible to modify existing structure classes
  in the same way that it is possible to modify standard classes even
  though there are existing instances, but I don't remember where I
  read this, and I can't seem to find the place.  It might have been
  in CLtL2 rather than the \hs{}, but I can't find it there
  either.  Oh well, when I find it, I will remove this footnote.}

However, since we plan for \sysname{} to be a conforming
implementation, we naturally plan to include structure classes as
well. 

The main reason for using structure classes rather than standard
classes is \emph{performance}.  Structure classes are supposed to be
implemented in the ``most efficient way possible''.%
\footnote{Again, I forget where I read this, and I can't find it.}
Presumably, the restrictions on structure classes exist to allow for
an implementation to represent instances as a pointer directly to the
vector of slots and avoid any indirection, which saves a memory
access.  However, in \sysname{} \emph{all} heap-allocated objects
(other than \texttt{cons} cells) are represented as a two-word
\emph{header object} and a \emph{rack} for reasons of
simplicity and in order to allow our memory-management strategy to
work.

Since \sysname{} represents instances of structure classes this way,
there is no reason to keep the restriction that structure classes can
not be modified.  For that reason, we plan to avoid that restriction.

Structure classes have another interesting restriction, namely that
they allow only single inheritance.  This restriction allows slot
accessors to be non-generic, because it becomes possible for a slot to
have the same physical position in all subclasses.  We may or may not
take advantage of this possibility.  The higher priority for
\sysname{} is to make accessors for standard objects fast, rather than
to work on an efficient implementation of structures. 

With respect to bootstrapping, since \sysname{} does not use structure
classes at all for its implementation, implementing \texttt{defstruct}
can be done fairly late.  In fact, we may omit it in initial version
of the system. 

\section{Generic function dispatch}
\label{sec-generic-function-dispatch}

\subsection{Call history}
\label{sec-generic-function-dispatch-call-history}

Each generic function contains a \emph{call history}.  The call
history is a simple list of \emph{call history entries}.  A call
history entry associates a list of classes (those of the classes of
the required arguments to the generic function) with a \emph{list of
  applicable methods} and an \emph{effective method}.

As permitted by the AMOP, when the generic function is invoked, its
discriminating function first consults the call history (though, for
performance reasons, not directly) in order to see whether an existing
effective method can be reused, and if so, it invokes it on the
arguments received.

If the call history does not contain an entry corresponding to the
classes of the required arguments, as required by the AMOP, the
discriminating function then first calls
\texttt{compute-applicable-methods-using-classes}, passing it the
classes of the required arguments.  If the second value returned by
that call is \textit{true}, then the effective method is computed by
calling \texttt{compute-effective-method}.  The resulting effective
method is combined with the classes of the arguments, and the list of
applicable methods into a call history entry which is added to the
call history, and the effective method is invoked on the arguments
received.  If the second value returned by the call is \textit{false},
then the discriminating function calls
\texttt{compute-applicable-methods} with the list of the arguments
received, and then the effective method is computed by calling
\texttt{compute-effective-method} and finally invoked.

When a method is added to the generic function, the call history is
traversed to see whether there exists a call history entry such that
the new method would be applicable to arguments with the classes of
the entry.  If so, the entry is removed.  If any entry was removed, a
new discriminating function is computed and installed. 

When a method is removed from the generic function, the call history
is traversed to see whether there exists a call history entry such
that the method to be removed is in the list of applicable methods
associated with the entry.  If so, the entry is removed. If any entry
was removed, a new discriminating function is computed and installed.

When a class metaobject is reinitialized, that class metaobject and
all of its subclasses are traversed.  For each class metaobject
traversed, \texttt{specializer-direct-methods} is called to determine
which methods contain that class as a specializer.  By definition, any
such method will be associated with a generic function.  The call
history of that generic function is traversed to determine whether
there is an entry containing that method, and if so, the entry is
removed from the call history.  The AMOP allows the implementation to
keep the entry if the \emph{precedence list} of the class does not
change as a result of being reinitialized, but for reasons explained
below, we remove the entry independently of whether this is the case.
If an entry was removed, a new discriminating function is computed and
installed.

\subsection{The discriminating function}
\label{sec-generic-function-dispatch-the-discriminating-function}

The discriminating function of a generic function is computed from the
call history.

If the call history has relatively few entries, then the
discriminating function computes the \emph{identification}%
\footnote{Recall that the \emph{identification} of an object is either
  the \emph{stamp} of the object if it is a general instance, or the
  \emph{unique number} of the class of the object if it is a special
  instance.}
of each of the required arguments.  It then uses numeric comparisons
in a tree-shaped computation to determine which (if any) effective
method to invoke.  In effect, the discriminating function becomes a
very simple \emph{automaton} where each transition is determined by a
comparison between two small integers.  The class numbers become
constants inside the compiled code of the discriminating function,
making comparison fast.  Each argument identification is tested from
left to right, without taking the \emph{argument precedence order} of
the generic function into account.  For each argument, the set of
possible effective methods is filtered by a binary search.  The search
is based on \emph{intervals of class numbers} as opposed to individual
class numbers.  This optimization can speed up the dispatch
considerably when an interval of class numbers yield the same
effective method.  Since it is common that the unique class numbers of
the classes in an inheritance subtree cluster into contiguous
intervals, this optimization is often pertinent, and in this
case, only two tests (for the upper and the lower bound of the
interval of class numbers) are required to determine whether that
method is applicable.

The automaton of the discriminating function can not contain class
numbers that were discarded as a result of classes being
reinitialized, simply because whenever a class is reinitialized, the
call history of every generic function specializing on that class or
any of its subclasses is updated and the discriminating function is
recomputed.

When a generic function is invoked on some arguments, the first step
is to compute the \emph{identification} of each required argument.
The identification is computed as follows:

\begin{itemize}
\item If the object is a \emph{general instance}, then it is the
  \emph{stamp} of the instance, i.e. the unique number of the class of
  the instance as it was when the instance was created.  The stamp is
  stored in the first element of the rack of the instance.
\item Otherwise (i.e., if the object is a \emph{special instance}, it
  is the \emph{unique number} of the class of the object.
\end{itemize}

The identifications are then used by the automaton to find an
effective method to invoke.  If the automaton fails to find an
effective method, the following steps are taken:

\begin{enumerate}
\item The identification is checked against the unique
  number of the class of the object.  If they are not the same, then
  the object is a general instance, and it is \emph{obsolete}.  The
  machinery for updating the instance is invoked, and then a second
  attempt with the automaton is made. 
\item If the object identification and the unique number of the
  class of the object are the same, then
  \texttt{compute-applicable-methods-using-classes} is called.  If the
  first return value is not the empty list and the second return value
  is \textit{true}, then an effective method is computed and a new
  entry is added to the call history and the automaton is recomputed.
  Finally the effective method is invoked.
\item If the first value is the empty list and the second value is
  still \textit{true}, then \texttt{no-applicable-method} is called.
\item If the second return value is \textit{false}, then 
 \texttt{compute-applicable-methods} is called.  If the result is the
 empty list, then \texttt{no-applicable-method} is called.  Otherwise
 an effective method is computed and invoked. 
\end{enumerate}

Notice that in most cases, no explicit test is required to determine
whether an instance is obsolete.  Also notice that for up-to-date
general instances, there is no need to access the class of the
instance in order to determine an effective method to call.  For
objects other than general instances, there is a small fixed number of
possible classes, so determining the identification of an object can
be open coded.

Notice also that many interesting optimizations are possible here when
the automaton is computed from the call history.

\begin{itemize}
\item If there is a single entry in the call history, the automaton
  can be turned into a sequence of equality tests (one for each
  required argument).  In particular, for an \emph{accessor method},
  the automaton degenerates into a single test and a call to either
  a method that directly accesses the slot or to
  \texttt{no-applicable-method}. 
\item In the case of a single entry in the call history, and a single
  applicable accessor method for that entry, the slot access can be
  open coded in the automaton. 
\item In the case above and when in addition the specializer of the
  accessor method is a class for a special instance (such as
  \texttt{fixnum}, \texttt{character}, or \texttt{cons}), determining
  the unique number of the class object is not required.  Instead, the
  discriminating function can be a simple test for tag bits.
\end{itemize}

If the call history has a large number of entries, a different technique
may be used.  The generic function \texttt{print-object} may be such a
function.  A simple hashing scheme might be better in that case.

\subsection{Accessor methods}
\label{sec-generic-function-dispatch-accessor-methods}

Accessor methods are treated specially when an effective method is
computed from a list of applicable methods.  Rather than applying the
default scheme of generating a call to the method function, when any
of the methods returned by
\texttt{compute-applicable-methods-using-classes} is an accessor
method, \texttt{compute-discriminating-function}, replaces such a
method by one that makes a direct access to the slot of the instance.
It does this by determining the \emph{slot location} of the slot in
instances of the class of the argument.  The substitution is
\emph{not} done by \texttt{compute-applicable-methods-using-classes}
itself, because the list of (sorted) methods returned by that function
is used for the purpose of caching in order to avoid recomputing an
effective method. 

Since the location of a slot may change when the class is
reinitialized, an effective method computed this way may become
invalid as a result.  For that reason, whenever a class is
reinitialized, any call history entries with methods specializing on
that class or any of its subclasses are removed.  This way, a call to
\texttt{compute-applicable-methods-using-classes} will be forced, and
a new location will be determined.

\section{Dealing with metastability issues}
\label{sec-object-system-dealing-with-metastability-issues}

The AMOP gives a few examples of metastability issues that need to be
dealt with.  It also suggests solutions to these problems, based on
recognizing \emph{special cases} such as when \texttt{class-slots} is
called with the class named \texttt{standard-class}.  

In \sysname{} we use a different method, called \emph{satiation}. 
Satiation, in effect, turns metastability problems into
\emph{bootstrapping problems} which are much easier to deal with.

\begin{definition}
A generic function $F$ is said to be \emph{satiated} with respect to a
set of classes $C$ if and only if for every combination of classes of
required arguments of $F$ for which an effective method can be
computed, if a call is made to $F$ with such a combination, then an
effective method exists in the cache of $F$ so that no additional
generic function needs to be invoked in order for the corresponding
effective method to be called.
\end{definition}

As part of bootstrapping the object system, every specified%
\footnote{By the AMOP.} generic function is satiated with respect to
the set of specified classes.  Furthermore, since specified classes
can not be redefined, we can make sure that every specified generic
function always remains satiated with respect to the set of specified
classes.

\begin{theorem}
If every specified generic function is satiated with respect to the
set of specified classes, then every call to \texttt{class-slots} will
terminate.
\end{theorem}

Proof: Base case: If \texttt{class-slots} is called with an instance
of \texttt{standard-class}, then by definition of satiation, the call
will terminate.  If \texttt{class-slots} is called with some other
class, then calls are made to
\texttt{compute-applicable-methods-using-classes},
\texttt{compute-effective-method}, and
\texttt{compute-discriminating-function} in order to compute a new
discriminating function for \texttt{class-slots}.  But these functions
are satiated with respect to \texttt{standard-generic-function} and
\texttt{class-slots} is a standard-generic-function, so those calls
will terminate. 

\section{Implementing \texttt{slot-value} and \texttt{(setf slot-value)}}

For reasons of brevity, the following discussion is about the function
\texttt{slot-value}, but the case of \texttt{(setf slot-value)} is
entirely analogous. 

In \emph{the Art of the Metaobject Protocol (AMOP)}
\cite{Kiczales:1991:AMP:574212}, these functions are mentioned as
prime examples of issues of \emph{metastability}.  The scenario that
is cited is that \texttt{slot-value} of an instance calls
\texttt{class-slots} on the class of the instance, and
\texttt{class-slots} is an accessor which calls \texttt{slot-value}.
In the AMOP, the metastability problem is resolved by recognizing that
the recursion must eventually reach \texttt{standard-class}, so that
treating \texttt{standard-class} as a special case resolves the
problem.

However, the scenario cited above represents a simplification of the
real one.  The specification requires \texttt{slot-value} to call
\texttt{slot-value-using-class} with the instance, the class of the
instance, and an \emph{effective slot definition metaobject}.  In
order for \texttt{slot-value} to find the right effective slot
definition metaobject, it has to traverse the list of effective slot
definition metaobjects until one is found that has the \emph{name} of
the slot given as an argument.  To find the name of an effective slot
definition metaobject, \texttt{slot-value} has to call
\texttt{slot-definition-name} which is an accessor which calls
\texttt{slot-value} which is again a metastability issue.

However, in \sysname{}, as
\refSec{sec-generic-function-dispatch-accessor-methods} explains,
accessors in \sysname{} do not call \texttt{slot-value}, so the
scenario from the AMOP does not apply.  Furthermore, since the
accessors \texttt{class-slots} and \texttt{slot-definition-name} are
\emph{satiated}
\seesec{sec-object-system-dealing-with-metastability-issues} these
functions do not have to call \texttt{slot-location} for the base
case. 

The standard methods on \texttt{slot-value-using-class} call the
reader function \texttt{slot-definition-location} on the effective
slot definition and use that location to call
\texttt{standard-instance-access} on the instance.
