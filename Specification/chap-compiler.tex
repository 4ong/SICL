\chapter{Compiler}

The compiler should be as portable as possible.  It should use
portable Common Lisp for as many of the passes as possible.  

The compiler should keep information about which registers are live,
and how values are represented in live registers, for all values of
the program counter.  This information is used by the garbage
collector to determine what registers should be scanned, and how.   It
is also used by the debugger.  

The compiler should do some extensive type inferencing.  It should be
able to eliminate code for which the result of executing it is known
as a result of the contents of the compilation environment.  

\section{Phase 1}

In phase 1 of the compilation, each expression is first converted into
an abstract syntax tree.  Alternatively, the reader supplies this
abstract syntax tree including information about source location of
every sub-expression.  Initially, the only types of nodes in the tree
correspond to constants, variables, and compound expressions.  

In the following step, the abstract syntax tree is traversed relative
to an \emph{environment} which determines what different types of
expressions mean, i.e. whether a variable is really a symbol-macro,
and whether a compound expression is a function call, a macro call, or
a special form.  In this step, macros and symbol macros are expanded,
and nodes representing special forms are specialized into a type of
node that is particular to each special operator.  References to
variables are replaced by entries into the environment, so that the
name of a variable no longer influences its scope. 

After conversion to class instances, each nested expression is
\emph{normalized} in several steps.  In the first step, each
expression that supplies a single value to another expression is
converted into first normal form as follows:

\begin{itemize}
\item A variable reference is already in first normal form.
\item A constant expression \texttt{c} is converted into first normal
  form by creating a \texttt{let} form \texttt{(let ((v c)) b)} where
  \texttt{v} is a new binding. 
\item An expression such as \texttt{(setq var expr)} is converted into
  first normal form by converting \texttt{expr} into first normal
  form.
\item An expression such as \texttt{(if test then else)} is converted
  into first normal form by converting each sub-expression into first
  normal form.
\item An expression such as \texttt{(let ((v1 e1) (v2 e2) ... (vn en))
  body)} is converted into first normal form by converting each
  \texttt{ei} and the \texttt{body} into first normal form.
\item An expression such as \texttt{(progn e1 e2 ... en)} is converted
  to first normal form by converting \texttt{en} to first normal
  form.  The other expressions do not supply values to other
  expressions.  However, these other expressions may contain other
  expressions that need to be converted to first normal form.
\item A function call expression such as \texttt{(f a1 a2 ... an)} is
  converted to first normal form by creating a \texttt{let} form
  \texttt{(let ((v1 aa1) (v2 aa2) ... (vn aan)) (f v1 v2 ... vn))}
  where \texttt{vi} are new bindings, and \texttt{aai} is \texttt{ai}
  converted into first normal form.
\end{itemize}

A second normalization phase gathers up all bindings to the outermost
level, and converts all the \texttt{let} forms into \texttt{progn}
forms as follows: \texttt{(let ((v1 e1) (v2 e2) ... (vn en)) body)}
becomes \texttt{(progn (setq v1 e1) (setq v2 e2) ... (setq vn en)
  body)}.

\section{Inlining}
\label{section-inlining}

The SICL compiler will use inlining in order to make type inferencing
more effective so that several tests can be avoided.  Implementors of
SICL features other than the compiler can assume that inlining is done
when it seems reasonable.  Therefore, even fairly low-level functions
such as \texttt{aref} can be implemented as portable Lisp code. 

