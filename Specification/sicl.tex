\documentclass{article}
\usepackage[T1]{fontenc}
\usepackage[utf8]{inputenc}
\usepackage{alltt}
\usepackage{moreverb}
\usepackage{epsfig}
% \usepackage{makeidx}

\setlength{\parskip}{0.3cm}
\setlength{\parindent}{0cm}

\def\sysname{SICL}

\def\inputfig#1{\input #1}
\def\inputtex#1{\input #1}

\title{{\Huge \sysname{}}\\
Building blocks for implementers of\\Common Lisp systems.}

\author{Robert Strandh}

\begin{document}

\maketitle

\begin{abstract}
\sysname{} (which doesn't mean anything in particular; pronounce it
like ``sickle'') is a project with the purpose of creating a
collection of highly-portable high-performance ``modules'' for
developers of Common Lisp systems.  Such modules include ``standard
libraries'' such as high-level functions that operate on lists,
high-level functions that operate on sequences, the \texttt{format}
function, the \texttt{loop} macro, the \texttt{read} function, the
\emph{pretty printer}, etc.  Other possible modules include code that
makes it easier to write compilers, such as parsers of lambda lists,
modules for manipulating declarations, etc.  Other planned modules
include a module that provides localized docstrings for all of Common
Lisp. 

Great care will be taken to make the modules as portable as possible,
and to decrease interdependence as much as possible in order to make
bootstrapping easier, and to allow for modules to serve as drop-in
replacements in different implementations with as few modifications as
possible.  The lower layers should use as few primitives as
possible, without sacrificing performance.  We think it is important
that the code of \sysname{} be of very high quality.  To that end, we
would like for error messages to be as explicit as possible.  Macros,
for instance, would have to do extensive syntax analysis so as to
prevent error messages from being phrased in terms of expanded code.

To gain wide acceptance, \sysname{} should be in the public domain, or
be distributed according to a license that serves the same purpose in
places where it is not possible for works to be explicitly placed in
the public domain.

Ultimately, this collection of modules might be used to produce a new
implementation of Common Lisp.  
\end{abstract}

\section{Introduction}

We think it is possible to improve on existing open-source Common Lisp
systems in several ways, and we hope \sysname{} will be able to
accomplish that, provided that great care is taken to create code with
a combination of characteristics:

\begin{itemize}
\item The code should be layered, so that different Common Lisp
  implementations can choose to include \sysname{} modules that
  represent gaps in their system or improvement on their existing
  code, without having to include parts for which they prefer their
  own, implementation-specific code. 

  Upper layers will contain code that is not performance critical.
  This code will be written entirely in Common Lisp.  To avoid
  circular references, we will specify what lower-level Common Lisp
  primitives can be used to write functions in the upper layer.  If
  done well, code in this layer could be used by all free Common Lisp
  implementations, which could save considerable maintenance effort.
  Examples of functionality in this layer would be formatted output,
  pretty-printing, and macros that can expand to portable Common Lisp
  code.

  Intermediate layers will contain code that needs careful tuning to
  obtain performance, but where the tuning can be handled by writing
  different versions of the code for different cases.  For instance,
  functions that work on all kinds of sequences might have special
  versions for lists and vectors.  Similarly, such functions might
  have special versions for common values of the \texttt{:test} (such
  as the Common Lisp functions \texttt{\#'eq}, \texttt{\#'eql}, etc.)
  and \texttt{:key} arguments (such as \texttt{\#'identity},
  \texttt{\#'car}, etc).  These special cases should be handled by
  using compiler macros.

  Lower layers will have to rely more and more on
  implementation-specific details, and will have to introduce
  implementation-specific primitives to be used in implementations of
  standard Common Lisp constructs.  We might provide several different
  versions of code in this layer for different low-level assumptions.

\item The code itself should have very high quality.  By this, we do
  not only mean that it should be bug-free as much as possible, but
  also that it should have good documentation strings and clear
  comments when required.  Error messages should be as explicit as
  possible, and code should be written to capture as many exceptional
  situations as is possible without performance penalties.  We plan to
  use very specific conditions that are subclasses of ones stipulated
  by the Common Lisp HyperSpec for condition signaling, so as to allow
  for implementations to take advantage of such specific conditions
  for better error reporting.  Macro expansion code should do
  everything possible to capture as many errors as possible so that
  error reporting can be done in terms of code written by the
  programmer, as opposed to in terms of expanded code.

\end{itemize}

\section{Coding style}

\subsection{Commenting}

In most programs, comments introduce unnecessary redundancies that can
then easily get out of sync with the code.  This is less risky for an
implementation of a specification that is not likely to change.
Furthermore, we would like \sysname{} to be not only a high-quality
implementation, but we would like for its code to be very readable.
For that reason, we think it is preferable to write \sysname{} in a
``literate programming'' style, with significant comments explaining
the code. 

\subsection{Designators for symbol names}

Always use uninterned symbols (such as \texttt{\#:hello}) whenever a
string designator for a symbol name is called for.  In particular,
this is useful in \texttt{defpackage} and \texttt{in-package} forms.

Using the upper-case equivalent string makes the code break whenever
the reader is case-sensitive (and it looks strange that the designator
has a different case from the way symbol that it designates is then
used), and using keywords unnecessarily clutters the keyword package.

\subsection{Docstrings}

We believe that it is a bad idea for an implementation of a Lisp
system to have docstrings in the same place as the definition of the
language item that is documented, for several reasons.  First, to the
person reading the code, the docstring is most often noise, because it
is known from the standard what the language item is about.  Second,
it often looks ugly with multiple lines starting in column 1 of the
source file, and this fact often discourages the programmer from
providing good docstring.  Third, it makes internationalization
harder.

For this reason, we will provide language-specific files containing
all docstrings of Common Lisp in the form of calls to (SETF
DOCUMENTATION). 

\subsection{Naming and use of slots}

In order to make the code as safe as possible, we typically do not
want to export the name of a slot, whereas frequently, the reader or
the accessor of that slot should be exported.  This restriction
implies that a slot and its corresponding reader or accessor cannot
have the same name.  Several solutions exist to this problem.  The one
we are using for \sysname{} is to have slot names start with the
percent character (`\%').  Traditionally, a percent character has been
used to indicate some kind of danger, i.e. that the programmer should
be very careful before directly using such a name.  Client code that
attempts to use such a slot would have to write
\texttt{package::\%name} which contains two indicators of danger,
namely the double colon package marker and the percent character.

Code should refer to slot names directly as little as possible.  Even
code that is private to a package should use an internal protocol in
the form of readers and accessors, and such protocols should be
documented and exported whenever reasonable. 

\subsection{Standard functions}

Standard functions should always check the validity of their arguments
and of any other aspect of the environment.  If such a function fails
to accomplish its task, it should signal an appropriate condition.  

We would like error messages to be phrased in terms of the code that
was directly invoked by user code, as opposed to in terms of code that
was indirectly invoked by system code.  As an example, consider a
sequence function such as \texttt{substitute}.  If it is detected that
a dotted list has been passed to this function, it should not be
reported by \texttt{endp} or any other system function that was not
directly called by user code, but instead it should be reported by
\texttt{substitute} in terms of the sequence that was originally
passed as an argument.  On the other hand, if substitute invokes a
user-supplied test that fails, we would like the error message to be
reported in terms of that user-supplied code rather than by
\texttt{substitute}.  This is how we are currently imagining solving
this problem:

\begin{itemize}
\item Standard functions do not call any other standard functions
  directly, other than if it is known that no error will be signaled.
  When a call from a standard function $f$ to a standard function $g$
  might result in an error being signaled by $g$, that call is
  replaced by a call to a special version of the standard function,
  say $h$ that signals a more specific condition than is dictated by
  the Common Lisp HyperSpec.
\item If acceptable in terms of performance, a standard function such
  as $f$ that calls other functions that may signal an error, handles
  such errors by signaling an error that is directly related to $f$. 
\item Error reporting is done in terms of the name and arguments to
  $f$. 
  
\end{itemize}

\subsection{Standard macros}

Standard macros must do extensive syntax analysis on their input so as
to avoid compilation errors that are phrased in terms of expanded
code.  

As with standard functions, standard macros that expand into other
system code that may signal an error should not use other standard
functions or other standard macros directly, but instead special
versions that signal more specific conditions.  The expanded code
should then contain a handler for such errors, which signals an error
in terms of the name and the arguments of the macro. 

\subsection{Compiler macros}

{\sysname} will make extensive use of compiler macros.  Compiler
macros are part of the standard, so this mechanism must be part of a
conforming compiler anyway.  In many cases, instead of encoding
special knowledge in the compiler itself, we can use compiler macros.
By doing it this way, we simplify the compiler, and we provide a set
of completely portable macros that any implementation can use. 

Compiler macros should be used whenever the exact shape of the call
site might be used to improve performance of the callee.  For
instance, when the callee uses keyword arguments, we can eliminate the
overhead of keyword-value parsing at runtime and instead call a
special version of the callee that does not have to do any such
parsing.  

Similarly, functions that take a \texttt{\&rest} argument can provide
special cases for different common sizes of the \texttt{\&rest}
argument.

We propose using compiler macros at least for the following
situations: 

\begin{itemize}
\item to convert calls to \texttt{list} and \texttt{list*} into nested
  calls to \texttt{cons};
\item to convert simple calls to some built-in functions that accept
  \texttt{:test} and \texttt{:key} keyword arguments (such as
  \texttt{find}, \texttt{member}, etc) into calls to
  special versions of these procedures with particularly simple
  functions for these keyword arguments (\texttt{identity},
  \texttt{car}, \texttt{eq}, etc);
\item to convert calls to some functions that accept optional
  arguments such as \texttt{last} and \texttt{butlast} into calls to
  special versions when the optional argument is not given.
\end{itemize}

Compiler macros should not be used in the place of inlining (see
section \ref{section-inlining}).

\subsection{Conditions and restarts}

\sysname{} functions should signal conditions whenever this is
required by the Lisp standard (of course) and whenever it is
\emph{allowed} by the Lisp standard and reasonably efficient to do so.
If the standard allows for subclasses of indicated signals (I think
this is the case), then \sysname{} should generate as specific a
conditions as possible, and the conditions should contain as much
information as possible in order to make it as easy as possible to
find out where the problem is located.

\sysname{} function should also provide restarts whenever this is
practical. 

\subsection{Condition reporting}

Condition reporting should be separate from the definition of the
condition itself.  Separating the two will make it easier to customize
condition reporting for different languages and for different
systems.  An integrated development environment might provide
different condition reporters from the normal ones, that in addition
to reporting a condition, displays the source-code location of the
problem. 

Every \sysname{} module will supply a set of default condition
reporters for all the specific conditions defined in that module.
Those condition reporters will use plain English text. 

\subsection{Internationalization}

We would like for {\sysname} to have the ability to report messages in
the local language if desired.  The way we would like to do that is to
have it report conditions according to a \texttt{language} object.  To
accomplish this, condition reporting trampolines to an
implementation-specific function \texttt{sicl:report-condition} which
takes the condition, a stream, and a language as arguments.

The value of the special variable \texttt{sicl:*language*} is passed
by the condition-reporting function to \texttt{sicl:report-condition}.

In other words, the default \texttt{:report} function for conditions is:

\begin{verbatim}
   (lambda (condition stream) 
     (sicl:report-condition condition stream sicl:*language*))
\end{verbatim}

Similarly, the Common Lisp function \texttt{documentation} should
trampoline to a function that uses the value of
\texttt{sicl:*language*} to determine which language to use to show
the documentation. 

\subsection{Package structure}

{\sysname} has a main package containing and exporting all Common Lisp
symbols.  It contains no other symbols.  A number of implementation
packages import the symbols from this package, and might define
internal symbols as well.  Implementation packages may export symbols
to be used by other implementation packages.

This package structure allows us to isolate implementation-dependent
symbols in different packages.  

\subsection{Assertions}

\subsection{Threading and thread safety}

Consider locking free.  We predict that a technique call ``speculative
lock elision'' will soon be available in all main processors. 

\section{Planned modules}

\subsection{Reader}

The reader can be implemented in an almost entirely portable way.  

We are planning to provide a reader with a very fast tokenizer in the
form of a state machine that builds up the token while it is being
read.  

We are also planning to include an entry point to the reader that,
instead of creating ordinary Common Lisp objects, will create an
abstract syntax tree that contains information about source-code
position.  This information could then be used by a compiler for
reporting compilation errors, or by the runtime system for reporting
sources of errors. 

The reader will use the method created by David Gay for accurate and
fast reading of floating-point numbers. 

The only problem as far as portability is concerned is with the reader
macros \texttt{\#\#} and \texttt{\#=}.  When an object labeled with
\texttt{\#=} is being read (and so has not been constructed yet) and a
reference to it is being encountered (thus creating a circular
structure), some temporary value must be put in until the object has
been constructed.  Then the object must be scanned for that temporary
value.  However scanning arbitrary objects for some substructure
cannot be done portably.

\subsection{Pretty printer}

\subsection{Printer}

A large part of the printer can be written portably without
performance penalty.  We intend to supply standard methods for
\texttt{print-object}, and code for functions such as \texttt{princ},
\texttt{prin1}, and \texttt{print}. 

We are planning to use the method created by Burger and Dybvig to
print floating-point numbers so that they can be read back to the
exact same number. 

\subsection{Format}

Again we are planning to use the method created by Burger and Dybvig
for printing floating-point numbers. 

\subsection{Loop}

Our \texttt{loop} module will use all available Common Lisp functions
for its analysis of syntax and semantics.  We believe this is not a
problem, even though we assume the existence of \texttt{loop} for many
other modules, because the code in this module will be executed during
macro-expansion time, and for a new Common Lisp system, this would be
another full Common Lisp implementation. 

For systems without a full implementation of \texttt{loop} we are
planning to provide macro-expanded versions of modules that would
otherwise require \texttt{loop} in order to be compiled. 

Our \texttt{loop} module will use only standard Common Lisp code in
its expansion, so that macro-expanded uses of \texttt{loop} will not
require any other \sysname{} module in order to work.  

\subsection{High-level functions on lists}

This module will implement high-level functions on lists, using
essentially only \texttt{car}, \texttt{cdr}, and \texttt{cons}.  For
its implementation, it will use the \texttt{loop} macro.  If any
other functionality is required, it will supply special
implementations of such functionality, so as to avoid dependencies on
other modules. 

High performance will be obtained by identifying important special
cases such as the use of \texttt{:test} function \texttt{eq}, or
\texttt{equal}, or the use of a \texttt{:key} of \texttt{identity}.

Compiler macros will be supplied so as to avoid runtime dispatch
whenever a special-case function can be determined by only looking at
the call site.  This ensures high performance for short lists, where
argument parsing would otherwise represent a significant fraction of
the cost of the call.

A macro-expanded version of this module will be supplied, which will
not require and existing implementation of \texttt{loop} and will also
not require any other \sysname{}-specific module. 

\subsection{Sequence functions}

This module will provide high-performance implementations of the
functions in the ``sequences'' chapter of the HyperSpec.  High
performance will be obtained by identifying important special cases
such as the use of \texttt{:test} function \texttt{eq}, or
\texttt{equal}, or the use of a \texttt{:key} of \texttt{identity}. 

Compiler macros will be supplied so as to avoid runtime dispatch
whenever a special-case function can be determined by only looking at
the call site.  This ensures high performance for short sequences,
where argument parsing would otherwise represent a significant
fraction of the cost of the call.

\subsection{Type declarations of standard Common Lisp functions}

This module contains portable type declarations for all standard
Common Lisp functions.  It could be used by implementers of Common
Lisp compilers to accomplish error checking and type inferencing. 

\subsection{Docstrings for all Common Lisp symbols}

As mentioned elsewhere, we believe that docstrings should be separate
from code, because they do not address the same audience.  In
addition, separating the two allows us to distribute the docstrings as
a separate module Many implementations have substandard docstrings, so
this is an important module that can be used as a drop-in replacement
for existing ones.

We will provide the infrastructure for allowing internationalization
of docstrings, but we probably will not provide different versions for
different languages. 

\subsection{Strings}

\subsection{Packages}

\subsection{Hash tables}

\subsection{Unicode support}

We currently do not plan to supply a module for Unicode support.
Instead we are relying on the support available in the Unicode library
by Edi Weitz.

\subsection{Streams}

We currently do not plan to supply a module for streams.  Instead we
are relying on the support available in the Flexi-stream library by
Edi Wietz.

\section{Towards a complete Common Lisp implementation}

\subsection{TYPECASE and CASE}

The compiler should generate very fast code for TYPECASE, at least
when the types tested are known to be mutually exclusive and typically
encoded in tag bits.  For this reason, TYPECASE cannot be a macro that
expands to COND or IF (well, I guess it could, but that seem backwards).

The compiler should also generate very fast code for CASE, at least
when the cases represent a dense and compact range of integers or
object that can be associated with such integers.

Higher-level code should be able to rely on the fast implementation of
these primitives. 

\subsection{Low-level special operators}

(this section is contrary to what I wrote in the first version.  I am
now convinced that modern optimization technology can handle arbitrary
tagbodies)

Iteration primitives, including LOOP should compile to TAGBODY.  From
a TAGBODY, the compiler then generates basic blocks that can then be
optimized by known compiler technology.  The TAGBODY can contain only
function calls, SETQs, GOs, and a conditional goto in the form of
(when <var> (go <tag>)) [is this possible?].

The special operator LET* is considered as simple as possible.  The
LET operator is expanded to LET* (using variable renaming).  **Is this
really important?

\subsection{Runtime information}

The compiler will generate runtime information available both to the
debugger and to the garbage collector.  For each value of the program
counter, all registers and stack frame locations will have accessible
type information.  Maintaining this type information does not require
any runtime overhead.  All that is required is a mapping from a
program counter value to a block of runtime information. 

A location (register, stack frame location) can have one of different
types of values.  On the topmost level of abstraction, there are three
possibilities for each location:

\begin{itemize}
\item Unused.  The location does not contain any accessible object.
\item Tagged Lisp value.  This is the most general type.  It covers
  every possible Lisp value.  The garbage collector must trace the
  object contained in this location according to its type, which the
  garbage collector itself has to test for. 
  \begin{itemize}
  \item Immediate Lisp value.  This type is given for tagged Lisp values
    that are not required to be traced by the garbage collector. 
  \item Lisp pointer.  This type is given for all tagged Lisp values
    that are represented by pointers to heap-allocated objects.  
  \end{itemize}
\item Raw machine value.  No location will be tagged with this type,
  but instead with any of the subtypes given below.
  \begin{itemize}
  \item Raw immediate machine value
    \begin{itemize}
      \item Raw integer.
      \item Raw Unicode character.
    \end{itemize}
  \item Raw machine pointer
    \begin{itemize}
    \item Raw machine pointer to a cons cell.  
    \item Raw machine pointer to the first element of a simple array. 
    \item Raw machine pointer to the beginning of an instance of
    standard-class. 
    \item Raw machine pointer that may point inside another object.
    In this case, the location has to be indicated as \emph{tied} to
    another location that contains either a Lisp pointer or a raw
    machine pointer to a known type of object.  This possibility will
    be used when (say) a pointer to an array is stored in some
    location, and this location contains an offset into that array.
    The garbage collector will modify this pointer value by the same
    amount as the one it is tied to. 
    \end{itemize}
  \end{itemize}
\end{itemize}

Low-level functions such as masking out tag bits will generate raw
pointer values.  The compiler must generate code to keep a tagged
pointer to the original object, unless there is a special type from
which the type information of the original object can be inferred. 

A number of functions will be provided that manipulate raw values,
such as arithmetic on raw integers, pointer arithmetic, loading from
and storing to memory, etc.  In order to be able to use such a
function, the compiler must be given enough type information to infer
the correct type of the arguments.  Conversely, functions that return
a value of a type other than a subtype of tagged Lisp value, must be
known to the compiler so as to forbid any use of such a value in a
context that requires a tagged Lisp value, and the compiler must
apply enough type inference to detect such cases. 

\subsection{Compiler}

The compiler should be as portable as possible.  It should use
portable Common Lisp for as many of the passes as possible.  

The compiler should keep information about which registers are live,
and how values are represented in live registers, for all values of
the program counter.  This information is used by the garbage
collector to determine what registers should be scanned, and how.   It
is also used by the debugger.  

The compiler should do some extensive type inferencing.  It should be
able to eliminate code for which the result of executing it is known
as a result of the contents of the compilation environment.  

\subsubsection{Phase 1}

In phase 1 of the compilation, each expression is first converted into
an abstract syntax tree.  Alternatively, the reader supplies this
abstract syntax tree including information about source location of
every sub-expression.  Initially, the only types of nodes in the tree
correspond to constants, variables, and compound expressions.  

In the following step, the abstract syntax tree is traversed relative
to an \emph{environment} which determines what different types of
expressions mean, i.e. whether a variable is really a symbol-macro,
and whether a compound expression is a function call, a macro call, or
a special form.  In this step, macros and symbol macros are expanded,
and nodes representing special forms are specialized into a type of
node that is particular to each special operator.  References to
variables are replaced by entries into the environment, so that the
name of a variable no longer influences its scope. 

After conversion to class instances, each nested expression is
\emph{normalized} in several steps.  In the first step, each
expression that supplies a single value to another expression is
converted into first normal form as follows:

\begin{itemize}
\item A variable reference is already in first normal form.
\item A constant expression \texttt{c} is converted into first normal
  form by creating a \texttt{let} form \texttt{(let ((v c)) b)} where
  \texttt{v} is a new binding. 
\item An expression such as \texttt{(setq var expr)} is converted into
  first normal form by converting \texttt{expr} into first normal
  form.
\item An expression such as \texttt{(if test then else)} is converted
  into first normal form by converting each sub-expression into first
  normal form.
\item An expression such as \texttt{(let ((v1 e1) (v2 e2) ... (vn en))
  body)} is converted into first normal form by converting each
  \texttt{ei} and the \texttt{body} into first normal form.
\item An expression such as \texttt{(progn e1 e2 ... en)} is converted
  to first normal form by converting \texttt{en} to first normal
  form.  The other expressions do not supply values to other
  expressions.  However, these other expressions may contain other
  expressions that need to be converted to first normal form.
\item A function call expression such as \texttt{(f a1 a2 ... an)} is
  converted to first normal form by creating a \texttt{let} form
  \texttt{(let ((v1 aa1) (v2 aa2) ... (vn aan)) (f v1 v2 ... vn))}
  where \texttt{vi} are new bindings, and \texttt{aai} is \texttt{ai}
  converted into first normal form.
\end{itemize}

A second normalization phase gathers up all bindings to the outermost
level, and converts all the \texttt{let} forms into \texttt{progn}
forms as follows: \texttt{(let ((v1 e1) (v2 e2) ... (vn en)) body)}
becomes \texttt{(progn (setq v1 e1) (setq v2 e2) ... (setq vn en)
  body)}.

\subsubsection{Tracking source code locations}

When a macro call is expanded, clearly the abstract syntax tree cannot
be used, because the Common Lisp standard requires macro functions to
work on raw Common Lisp expressions.  In interesting problem then is
to track source-code locations through macro expanders.  Specifically,
we need to make an educated guess about the origin of a form in the
expansion with respect to the sub-expressions of the macro call.

Compound expressions and complex atoms (atoms other than numbers,
characters, and symbols) pose relatively few difficulties.  We build a
dictionary of all such sub-expressions in the original expression,
together with the corresponding abstract syntax tree.  Then we scan
the expanded form for sub-expressions that are \texttt{eq} to those
sub-expressions in the original form.  Clearly, those sub-expressions
were directly inserted into the expansion by the macro function.

Next, we try to find out whether any of the sub-expression found in
the first step were taken apart by the macro function, We do this by
successively replacing each sub-expression in the original form by a
an atom of a different type than the original sub-expression, and then
we call the macro function again.  If it succeeds without an error, we
are fairly sure that the corresponding sub-expression is
\emph{opaque}, i.e., that no other sub-expression in the expanded form
originated inside such a sub-expression.

The tricky part is to identify the origins of simple atoms (numbers,
characters, and symbols) in the expanded form.  We again build a
dictionary of sub-expressions in the original expression, but this
time containing such atoms.  We exclude opaque sub-expressions from
this search, as we are fairly sure that those cannot contain atoms
that appear in the expanded expression.

Then the expanded form is scanned for simple atoms.  If any atom in
the expanded form does not have an entry in the dictionary, then it
originated from the macro expander, either as a literal or as a
computed value.

For the remaining atoms, we try to guess whether they really came from
the original expression, or whether it is just by accident that an
atom in the original expression was also inserted by the macro
function.  We do this by successively replacing each suspect simple
atom in the original expression with a different atom of the same
type.  An integer is replaced by its predecessor or successor, a
character by the next one or previous one in the code order, a symbol
is replaced by a generated symbol.  Then we macro-expand again and
check whether the suspect atom in the expanded code is now
\texttt{eql} to the replacement.  If so, we are fairly sure that the
atom came from the original expression.

Any sub-expression in the expanded form that has not been determined
by the procedure above to come from the original expression is
converted to an abstract syntax tree with unknown source code
location.

This method clearly is not totally accurate.  It would be easy to
write a macro function that inspects a sub-expression of an expression
only if it has been determined to be compound.  Our method will
incorrectly mark such a compound expression as opaque and will not
find sub-expressions in the expanded expression that originated inside
such a compound expression.  We are guessing that such situations will
be fairly rare. 

\subsection{Inlining}
\label{section-inlining}

The SICL compiler will use inlining in order to make type inferencing
more effective so that several tests can be avoided.  Implementors of
SICL features other than the compiler can assume that inlining is done
when it seems reasonable.  Therefore, even fairly low-level functions
such as \texttt{aref} can be implemented as portable Lisp code. 

\subsection{Generic-function dispatch}

\subsection{Garbage collection}

We think it would be good to use a per-thread nursery combined with a
global allocator for older objects.  This technique has been published
by Doligez and Leroy in their paper ``A concurrent, generational
garbage collector for a multithreaded implementation of ML'' and
perhaps in other papers as well.

\subsubsection{Global collector}
The global allocator can be generational or not.  We imagine the use
of a fake-copying collector so as to avoid problems with objects
moving around.  As Paul Wilson noted, fragmentation is a problem only
in theory, and only when inaccurate statistical models of the behavior
of real programs are used.  In the collector that Paul Wilson uses,
objects are grouped by size and organized into doubly-linked lists.
We think this might be a bit too wasteful for small objects, and some
internal fragmentation must also be accepted so as to avoid a
doubly-linked list for each possible size.  Perhaps using bitmaps
together with an efficient way of accessing them in order to determine
free zones of memory.

Here is how we imagine the details so far.  These ideas have not been
tested yet. 

We use a bitmap with a bit for each double word in the heap.  This
method wastes a very small amount of memory compared to the size of
the heap.  The bitmap is organized as a vector of consecutive words.
Initially, the bitmap contains all `1's indicating that all memory is
free.  During tracing, bits are cleared in the bitmap
whenever an address contains a live object.

When tracing is finished, we scan the bitmap to collect blocks of free
memory.  This can be done very efficiently without looking at each bit
of a block as follows: We start by finding the first word that is not
all `0's.  Anything before that is in use.  We then start collecting a
free block, initially of size 0. We do this by using
\texttt{integer-length} of both the word in the bitmap vector and of
the \texttt{lognot} of that word.  This information will tell us what
prefix of the word contains a block, and whether that block is free or
not.  If the entire word contains `1's then we add $2n$ to the size of
the free block, where $n$ is the word length of the processor, and we
continue with the next word.  If the word does not contain all `1's,
but there is a prefix of `1's (as indicated by the fact that
\texttt{integer-length} of the word is $n$), then we add the
difference between the $n$ and the \texttt{integer-length} of the
\texttt{lognot} of the word to the block size, and we put the block on
a free-list as indicated below.  Finally we clear the prefix of `1's,
and we iterate the procedure for this word.  If the word contains a
prefix of `0's and we are currently accumulating a block, we close the
block, skip the prefix and start over.  

We manage the free-lists as follows:  We have a relatively small
number of free-lists of specific sizes.  These sizes are such that the
binary representation contains from $1$ to $3$ initial `1's, and the
rest is all `0's.  Requests for blocks are rounded up to the next
higher size of this form.  For sizes less than $8$, we have free-lists
for sizes $2$, $4$, and $6$ (but not for $7$ because it is odd).  This
method generates just under $4n$ free-lists, where $n$ is again the
word size of the processor.  We organize the free-lists in a vector
by increasing block size.  It is trivial to round up any size to an
existing block size, and it is trivial to compute the index of the
vector, given the block size.  Each free list contains a linked list
of blocks.  The first word of the block is the exact size of the block
in words.  The second word is a pointer to the next block in the
list.  When a request for allocation is made, we round up the size and
index the vector to see whether a block of that size exists.  If not,
we search at higher indexes in the vector until a free block is
found.  We return the pointer to that block, subtract its size from
the real size requested (rounded up to an even size), and insert the
remaining words of the block (if any) into an appropriate free-list.  
This method wastes at most $1/8$ of the total memory, and on the
average only $1/16$, which is very little compared to schemes that
round up to powers of $2$, or use a Fibonacci series for block sizes. 

During tracing, if a request for allocation is made, we clear the
corresponding bits in the bitmap, indicating that this block is no
longer free (we ``allocate black'').  The only time when a request for
allocation must wait is when the free-lists are constructed from the
bitmap. 

Since the nursery collector (see corresponding section below) preserves
the allocation order of objects, and since it is known that ``objects
that are allocated together, die together'' (see Wilson), it is likely
that the heap will contain large consecutive blocks (whether live or
dead), making it very fast to build the free lists. 

\subsubsection{Nursery collector}
For the nursery, we imagine a copying collector managing small (a few
megabytes) linear space.  Instead of promoting objects that survive a
collection, we would like to investigate the possible use of a sliding
collector in the nursery.  Such a collector gives a very precise idea
of the age of different objects, so objects would always be promoted
in the order of the oldest to the youngest.  This technique avoids the
problem where the allocation of some intermediate objects is
immediately followed by a collection, so these objects are promoted
even though they are likely to die soon after the collection.  In a
sliding collector, promotion will happen only when a collection leaves
insufficient space in the nursery, at which point only the number of
objects required to free up enough memory would be promoted, and in
the strict order of oldest to youngest. 

\subsubsection{Promotion}

Promotion could happen for reasons other than age.  Objects that are
too large for the nursery would be allocated directly in the global
allocator.  Objects to which references from foreign-language code are
about to be created would first be promoted to the global collector
where they would no longer move.

\subsection{Debugger}
\label{section-debugging}

Part of the reason for SICL is to have a system that provides
excellent debugging facilities for the programmer.  We imagine that a
thread will be debugged by a debugger running in a different thread.
The debugger should be able to set breakpoints before and after
expressions, and at the beginning or the end of a function.  
We imagine this being done by a conditional branch at strategic points
in the code, perhaps implemented as a ``skip the next instruction if
condition is false'' instruction, if that should turn out to be faster
than a normal conditional branch.  The condition tests a flag in the
current thread.  If the flag is set, then a call is made to determine
whether there is a breakpoint at this place.  If not, it returns and
execution continues as usual.  If there is a breakpoint, then
execution stops and the debugger thread is given the possibility to
inspect the state of the debugged thread. 

For the highest debug level, the conditional branch should be
generated before and after each expression, including each variable
reference.  This can be a bit costly for local variables because it
would slow down execution significantly.  Lower debug levels may
generate the conditional branch only when the cost of the branch is
negligible compared to the cost of evaluating the expression. 

In order not to slow down the execution too much, there should be a
quick test to determine that there is no breakpoint at a particular
place in the code.  We still have to think of ways of doing that.  One
might imagine a conservative test that is quick but possibly not
entirely accurate, perhaps based on intervals of values of the program
counter (say, take PC, shift it right by some number of bits, check a
bitvector to see if the corresponding bit is set, if not, there is no
breakpoint here). 

For each possible breakpoint, the system must keep a description of
the lexical environment.  This includes mappings from variable names
to registers or stack locations, information about liveness of
registers and stack location, how a variable is stored in a location
(immediate value, pointer, with or without type tag, etc). 

\subsection{Source code tracking}

The SICL reader provides an entry point that returns a second value
which contains a syntax tree in which each expression is associated
with source-code location.  The compiler calls this entry point, and
creates its abstract syntax tree with this information in it.  

After the compiler has called a macro expander, the resulting
expansion is traversed in the search of expressions that are eq to the
arguments given to the macro expander.  Such expressions are
associated with the source code location of the argument. 

\subsection{Tracing}

Tracing should not use the method used by some Common Lisp
implementations to encapsulate the traced function inside a
trace-reporting function.  Instead it could use the same mechanism as
the one used for debugging (see section \ref{section-debugging}). 

\subsection{Metering}

Just as was the case with the Multics system, we think it is important
to be able to meter as many system functions as possible.  We suggest
adding such meters in every place where something measurable is going
on, and where adding a meter does not significantly degrade
performance.  

\end{document}
