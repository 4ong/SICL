\chapter{Processing return values}

In this chapter, we describe how processing return values is
accomplished by inserting HIR instructions immediately after HIR code
is generated from an abstract syntax tree.  As with the code for
processing arguments, by doing it this way, we obtain several
advantages:

\begin{itemize}
\item We simplify the translation of HIR code to LIR later on the
  translation process.
\item HIR transformations such as constant hoisting and
  \texttt{fdefinition} hoisting can be applied to the
  argument-processing code, thereby simplifying this code.
\item The HIR instructions introduced are subject to various HIR
  transformations such as value numbering, constant propagation,
  etc.
\end{itemize}

\begin{figure}
\begin{center}
\inputfig{fig-process-values.pdf_t}
\end{center}
\caption{\label{fig-process-values}
Processing return values.}
\end{figure}
