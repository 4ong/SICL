\chapter{General \commonlisp{} style guide}

\section{Different meanings of \texttt{nil}}

Consider the following local variable bindings:

\begin{verbatim}
(let ((x '())
      (y nil)
      z)
  ...)  
\end{verbatim}

To the compiler, the three are equivalent.  To a person reading the
code, they mean different things, however:

\begin{itemize}
\item The initialization of \texttt{x} means that \texttt{x} holds a
  \emph{list} that is initially empty.
\item The initialization of \texttt{y} means that \texttt{y} holds a
  Boolean value or a default value that may or may not change in the
  body of the \texttt{let} form.
\item The absence of initialization of \texttt{y} means that no
  initial value is given to \texttt{z}.  In the body of the
  \texttt{let} form, the variable \texttt{z} will be assigned to
  before it is used. 
\end{itemize}

The following body of the \texttt{let} form corresponds to the
expectations of the person reading the code:

\begin{verbatim}
(let ((x '())
      (y nil)
      z)
  ...
  (push (f y) x)
  ...
  (unless y (setf y (g x)))
  ... 
  (setf z (h x))
  ...)
\end{verbatim}

The following body of the \texttt{let} form violates the expectations
of the person reading the code:

\begin{verbatim}
(let ((x '())
      (y nil)
      z)
  ...
  (push (f y) z)     ; z is used before it is assigned.
  ...
  (unless x          ; x is treated as a Boolean.
    (setf y (g x)))
  ... 
  (push (f x) y)     ; y is treated as a list.
  ...)
\end{verbatim}
