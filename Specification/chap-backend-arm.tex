\chapter{ARM}

\section{Register use}

The ARM standard allocates registers as follows:

\begin{tabular}{|l|l|}
\hline
Name & Used for\\
\hline
\hline
R0 & First argument, first value, scratch\\
R1 & Second argument, second value, scratch \\
R2 & Third argument, scratch \\
R3 & Fourth argument, scratch \\
R4 & Register variable (callee saved) \\
R5 & Register variable (callee saved) \\
R6 & Register variable (callee saved) \\
R7 & Register variable (callee saved) \\
R8 & Register variable (callee saved) \\
R9 & Varies according to the platform\\
R10 & Register variable (callee saved) \\
R11 & Register variable (callee saved) \\
R12, IP & Intra-procedure-call scratch register\\
R13, SP & Stack pointer \\
R14, LR & Link register, scratch \\
R15, PC & Program counter \\
\hline
\end{tabular}

We mostly respect this standard, and allocate as follows:%
\footnote{Unfortunately, because we need to pass the static
  environment and the argument count as arguments, there are only two
  registers left to pass real arguments.  Three would have been
  better.  We might consider using R4 as an additional
  argument/scratch register.}

\begin{tabular}{|l|l|}
\hline
Name & Used for\\
\hline
\hline
R0 & First argument, first value, scratch\\
R1 & Second argument, second value, scratch \\
R2 & Static env. argument, third value, scratch \\
R3 & Argument count, Value count, scratch \\
R4 & Register variable (callee saved) \\
R5 & Register variable (callee saved) \\
R6 & Register variable (callee saved) \\
R7 & Register variable (callee saved) \\
R8 & Register variable (callee saved) \\
R9 & The current thread\\
R10 & Register variable (callee saved) \\
R11 & Frame pointer\\
R12, IP & Code-object argument\\
R13, SP & Stack pointer \\
R14, LR & Link register, scratch \\
R15, PC & Program counter \\
\hline
\end{tabular}
