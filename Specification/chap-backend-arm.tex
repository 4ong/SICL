\chapter{ARM}

\section{Register use}

The ARM standard allocates registers as follows:

\begin{tabular}{|l|l|}
\hline
Name & Used for\\
\hline
\hline
R0 & First argument, first value, scratch\\
R1 & Second argument, second value, scratch \\
R2 & Third argument, scratch \\
R3 & Fourth argument, scratch \\
R4 & Register variable (callee saved) \\
R5 & Register variable (callee saved) \\
R6 & Register variable (callee saved) \\
R7 & Register variable (callee saved) \\
R8 & Register variable (callee saved) \\
R9 & Varies according to the platform\\
R10 & Register variable (callee saved) \\
R11 & Register variable (callee saved) \\
R12, IP & Intra-procedure-call scratch register\\
R13, SP & Stack pointer \\
R14, LR & Link register, scratch \\
R15, PC & Program counter \\
\hline
\end{tabular}

We mostly respect this standard, and allocate as follows:

\begin{tabular}{|l|l|}
\hline
Name & Used for\\
\hline
\hline
R0 & First argument, first value, scratch\\
R1 & Second argument, second value, scratch \\
R2 & Third argument, third value, scratch \\
R3 & Static env. argument, value count, scratch \\
R4 & Register variable (callee saved) \\
R5 & Register variable (callee saved) \\
R6 & Register variable (callee saved) \\
R7 & Register variable (callee saved) \\
R8 & Register variable (callee saved) \\
R9 & Frame pointer\\
R10 & Dynamic environment (callee saved) \\
R11 & Register variable (callee saved)\\
R12, IP & Linkage vector/function argument\\
R13, SP & Stack pointer \\
R14, LR & Link register, scratch \\
R15, PC & Program counter \\
\hline
\end{tabular}

The caller allocates the frame for the callee. 

The callers saves its FP and sets FP to SP.  It then subtracts from SP
4*N where N is the number of arguments to be passed.  Notice that even
if all the arguments to be passed will fit in the argument registers,
the caller still leaves room in the stack frame of the callee.  The
main reason for this way of doing it is so that FP-SP = argcount in
the callee, and we do not have to allocate a register specifically for
the argument count.

In the simple and most common situation where the callee has neither
any \&optional, \&rest nor any \&key parameters, the callee must then
do three things:

\begin{enumerate}
\item Adjust the stack pointer to reflect the size that it requires
  for objects with dynamic extent, possibly for the return address,
  and for the callee-saved registers that need to be saved right
  away, because they are needed in the third step. 
\item Save the callee-saved registers that are guaranteed to be used
  in the body code, or for which it may be impractical to determine
  whether they will be used or not.  
\item Compute a strange kind of permutation so that the arguments are
  in the right initial place to start the body code.  This permutation
  involves the stack frame, the argument register, and the calle-saved
  registers.  In most cases, this permutation will be the identity,
  because arguments will remain in scratch registers until they might
  have to be moved to callee-saved registers or to the stack. 
\end{enumerate}

If there are \&optional parameters that are simple constants or
references to preceding variables, they can also easily be
initialized in case they were not passed by the caller. 

If there are \&optional parameters that require function calls, then
the register allocator has computed the initial permutation so that no
important objects remain in scratch registers.  

Upon function return, R3 contains the number of values as a fixnum.
If there are no more than 3 values, R0, R1, and R2 contain the
values. If there are 4 or more values, R0 and R1 contain the first two
values, and R3 points to the first address containing the remaining
arguments, starting with the third value.

Callee-saved registers that have been used by some function F must be
restored before F returns to its caller, and before F makes a tail
call.

Although listed as callee-saved, R10 containing the dynamic
environment is treated specially.  When some function F needs to push
an entry onto the dynamic environment, it uses the existing value of
R10 as the current dynamic environment and it simply extends it rather
than spilling it.  To restore the original value, F can simply pop off
the top value.  However, F can decide to spill R10 in case it needs it
form some other purpose.  However, before a function call, it needs to
be restored.  A CATCH will store R10 along with other callee-saved
registers so that it can be restored in case of a throw.

R12 is also treated specially.  When an external call is made, the
caller puts the function object to call in this register, and then
calls the function-call code of the class of the object.  That
function-call code replaces the function by the linkage vector, and
also initializes R3.  When an internal call is made, the caller does
not change this register, because it already contains the correct
linkage vector.  Notice that the linkage vector is an ordinary \cl{}
vector with header object and all.  When a function A calls a function
B, if A can be sure that B uses the same code-object and thus the same
linkage vector, then as A is about to call B, A makes sure that this
register contains the same value as it did upon entry to A.  It can do
that either by not using the register for any other purpose, or by
spilling it and restoring it before the call to B.  B guarantees that
the register will contain the same value as it did when B was called.
In the opposite case, i.e., if the compiler can not determine that A
and B use the same code object, then upon calling B, A must save the
register and load the linkage vector of B into it.  Sometime after B
returns and before A itself returns to its caller, A must restore the
register in case A's caller uses the same linkage vector as A.  Notice
that this protocol implies that if A and B use different code objects,
then a call from A to B can never be a tail call.

At any point in time we might need to know what code object is
associated with any stack frame.  The reason is that the code object
contains information about the layout of the stack frame which is
important for debugging purposes, but also for the garbage collector.
However, for performance reasons, we do not want to pass the code
object at each function call, since we are hoping that most function
calls will be to a function using the same code object.  For that
purpose, the linkage vector always contains a back pointer to the code
object. 

The current thread is contained in the dynamic environment as the
binding of a special variable.
