\chapter{Medium-level intermediate representation (MIR)}

The compiler translates an abstract syntax tree into a \emph{graph} of
\emph{instructions} in a language named \textbf{MIR}.  It is a general
graph because an instruction can have zero, one, or more
\emph{successors}, and zero, one, or more \emph{predecessors}.  The
graph can also contain \emph{cycles}.  

An instruction can also have zero, one, or several \emph{inputs} and
zero, one, or more \emph{outputs}.  An input may be an
\emph{immediate} value, a \emph{lexical variable}, or an \emph{entry
  in the linkage vector}.  An output may only be a lexical variable.

Executing a MIR instruction consists of generating the outputs as a
function of the inputs, and also of choosing a successor based on the
inputs.

An execution of a MIR program consists of executing a sequence of
instructions where an instruction in the sequence is the successor
chosen during the execution of the preceding instruction in the
sequence.

\section{Definition of MIR instructions}

\subsection{Instruction \texttt{NOP}}

This instruction has a single successor.  It has no inputs and no
outputs.

Executing this instruction has no effect. 

\subsection{Instruction \texttt{assignment}}

This instruction has a single successor.  It has a single input and a
single output. 

The execution of this instruction results in the input being copied to
the output without any modification. 

\subsection{Instruction \texttt{test}}

This instruction has two successors.  It has a single input. 

The first successor is chosen if the input is \emph{true} and the
second input is chosen if the input is \emph{false}.%
\fixme{Check that this statement is true.}

\subsection{Instruction \texttt{funcall}}

\subsection{Instruction \texttt{get-arguments}}

\subsection{Instruction \texttt{get-values}}

\subsection{Instruction \texttt{put-arguments}}

\subsection{Instruction \texttt{put-values}}

\subsection{Instruction \texttt{enter}}

\subsection{Instruction \texttt{leave-values}}

\subsection{Instruction \texttt{return}}

\subsection{Instruction \texttt{enclose}}

\subsection{Instruction \texttt{memref}}

\subsection{Instruction \texttt{memset}}

\subsection{Instruction \texttt{u+}}

\subsection{Instruction \texttt{u-}}

\subsection{Instruction \texttt{s+}}

\subsection{Instruction \texttt{s-}}

\subsection{Instruction \texttt{neg}}

\subsection{Instruction \texttt{\&}}

\subsection{Instruction \texttt{ior}}

\subsection{Instruction \texttt{xor}}

\subsection{Instruction \texttt{~}}

\subsection{Instruction \texttt{s$<$}}

\subsection{Instruction \texttt{s$<=$}}

\subsection{Instruction \texttt{$u<$}}

\subsection{Instruction \texttt{$u<=$}}

\subsection{Instruction \texttt{catch}}

\subsection{Instruction \texttt{unwind}}
