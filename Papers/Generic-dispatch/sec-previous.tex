\section{Previous work}

Most work on generic dispatch has been done in the context of more
mainstream programming languages such as \cplusplus{} or \java{}.
There are three aspects of \cl{} that complicate the situation with
respect to such languages:

\begin{enumerate}
\item Multiple inheritance.
\item Multiple dispatch.
\item Interactivity.
\end{enumerate}

Multiple inheritance%
\footnote{\cplusplus{} also has multiple inheritance of course.}
makes it possible for a slot to have a different \emph{position} in
the slot vector in instances of different classes.  Slot accessors
must take this possibility into account.

Multiple dispatch makes it more difficult to use table-based
techniques with entries for each class, because the size of the table
grows exponentially in the number of specialized parameters.
Table-compression techniques help overcome some of these problems, at
the cost of a more complicated, and thus more expensive, dispatch
algorithm. 

Many existing techniques are based on the complete program being
available in order to compute dispatch tables.  The interactive nature
of \cl{} makes it more difficult to use such techniques, because every
table may have to be recomputed whenever a small modification is made
to a class.  Though, some techniques may amortize this cost by
\emph{invalidating} some tables and recomputing them when required.

The fact that \cl{} is interactive also makes it possible for the
existence of \emph{obsolete instances}, i.e., instances where the
slots no longer correspond to the definition of the class, so they
have to be \emph{updated} before they are inspected.  

\subsection{PCL}

In PCL\footnote{PCL stands for Portable Common Loops.}
\cite{Kiczales:1990:EMD:91556.91600} a standard object is represented
as a two-word header object where the first words is a pointer to a
\emph{class wrapper} and the second word is a pointer to the
\emph{slot vector} of the instance.  The class wrapper is also a
two-word structure that contains a \emph{hash seed} and a pointer to
the class object. 

Each generic function contains a \emph{memoization table}.  Each entry
of the table contains a class wrapper and the entry point for the
effective method to be called for instances of the corresponding
class.  The memoization table uses a simple hashing mechanism, so that
the hash seed of the class wrapper of the argument is reduced modulo
the size of the memoization table in order to find the corresponding
entry.  The class wrapper in the entry is compared using \texttt{eq}
to the class wrapper of the argument, and if they are the same, the
corresponding effective method is called.  When there is no hit, it
could be that there is a hash collision, or it could be that no entry
exists in the table for the class of the argument.  Thus, if there is
no initial hit, the table is searched sequentially until an entry is
found or all the entries have been examined. 

In the best case then, the following operations are required for a
simple slot reader generic function:

\begin{enumerate}
\item Access the class wrapper of the argument; a memory access.
\item Access the hash seed of the class wrapper; a memory access.
\item Access the size of the memoization table; a memory access.
\item Reduce the hash seed of the class wrapper modulo the size of the
  memoization table; a simple masking operation if the size of the
  table is a power of $2$.
\item Access the memoization table of the generic function; a memory
  access.
\item Access the class wrapper in the memoization table entry; a
  memory access.
\item Compare the class wrapper in the memoization table entry to the
  class wrapper of the argument; a simple register comparison.
\item Access the entry point of the effective method in the
  memoization table entry; a memory access.
\item Jump to the entry point of the effective method; an
  unconditional jump.
\item The effective method accesses the slot vector of the instance; a
  memory access.
\item The slot containing the desired object is read and returned; a
  memory access.
\end{enumerate}

As we can see, there are $8$ memory accesses involved. 

The authors also mention an optimization for slot readers and slot
writers in the case where such a generic function is called with only
a few different classes.  In that case, they suggest replacing the
table lookup with a simple test for the different cases.  However,
since class wrappers are heap-allocated objects, a copying garbage
collector may move them around.  For that reason, class wrappers can
not be inline constants in the code, and must be stored in the generic
function.  If such an optimization is implemented, the mechanism is
reduced to the following steps:

\begin{enumerate}
\item Access the class wrapper of the argument; a memory access.
\item Access one or more class wrappers stored in the generic
  function; at least one memory access.
\item Compare the class wrapper of the argument to the class
  wrapper(s) stored in the generic function; fast register operation.
\item Access the slot vector of the instance; a memory access.
\item The slot containing the desired object is read and returned; a
  memory access.
\end{enumerate}

The minimum number of memory access is reduced to $4$ if the generic
function is called with instances of a single class.  

The technique used by PCL automatically catches obsolete instances.
When a class is modified, the current wrapper is eliminated from all
existing memoization tables, forcing the lookup to fail, and thus
triggering the update of the obsolete instance.  

\subsection{Other work}

The work of Zibin and Gil \cite{Zibin:2002:FAC:582419.582434} also
discusses table-based techniques.  Their paper addresses multiple
inheritance and multiple dispatch, but concentrates on the efficiency
of the algorithm for building the dispatch table and the size of the
resulting dispatch table. 

