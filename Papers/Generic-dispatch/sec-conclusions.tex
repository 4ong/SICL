\section{Conclusions and future work}

We have presented a fast method for generic dispatch in \cl{}.
Clearly, our tests do not represent any scientifically convincing
argument that our method is faster than that of PCL.  Rather, the
presentation of the method itself should be considered essence of the
paper, and the performance simulations should only be viewed as
indications that our method is worth pursuing as the bases of the
generic dispatch mechanism in SICL. 

Having said that, we can still speculate about the impact of our
method, should the results be confirmed in a more realistic setting. 

With our method, the amount of work to be done in a simple slot reader
or slot writer is no greater than the work needed for a non-generic
version, such as \texttt{symbol-name} or \texttt{package-nicknames} in
a typical \cl{} implementation.  Our method therefore makes it
feasible to make such readers and writers generic, and this is exactly
what we do in SICL.  We use the same \clos{} mechanisms (i.e., class
initialization, class finalization, etc.) for built-in classes as
those used for standard classes, even though built-in classes can not
be redefined.  By using the same mechanism, we remove a number of
special cases and we are able to simplify the overall structure of the
system. 

It is even possible to go one step further.  The amount of work
required in our dispatch mechanism is no greater than the work needed
in a binary addition function that tests for the exact type of its
arguments.  Hence it is entirely feasible to make such a function
generic, allowing the user to add methods for other kinds of objects
such as polynomials or other mathematical objects.  For reasonable
performance, it would still be required to capture special cases such
as fixnum or floating-point addition and inline them, but the default
function could very well be an ordinary generic function without any
significant loss of performance. 

Talk about print-object and other special cases.

Talk about the future of SICL.
 
