\documentclass{acm_proc_article-sp}
\usepackage[utf8]{inputenc}

\def\inputfig#1{\input #1}
\def\inputtex#1{\input #1}
\def\inputal#1{\input #1}
\def\inputcode#1{\input #1}

\inputtex{logos.tex}
\inputtex{refmacros.tex}
\inputtex{other-macros.tex}

\begin{document}
\title{Fast generic dispatch for Common Lisp}
\numberofauthors{1}
\author{\alignauthor
Robert Strandh\\
\affaddr{University of Bordeaux}\\
\affaddr{351, Cours de la Libération}\\
\affaddr{Talence, France}\\
\email{robert.strandh@u-bordeaux1.fr}}

\maketitle

\begin{abstract}
We describe a method for generic dispatch that is adapted to modern
computers where accessing memory is potentially quite expensive.
Instead of the traditional hashing scheme used by PCL, we assign a
\emph{unique number} to each class and the dispatch consists of
comparisons of the number assigned to an instance to a certain number
of (usually small) constant integers.  While our implementation (SICL)
is not yet in a state where we are able to get exact performance
figures, a conservative simulation suggests that our method is
significantly fast than the one used in SBCL, which uses PCL.
\end{abstract}

\category{H.4}{Information Systems Applications}{Miscellaneous}
%A category including the fourth, optional field follows...
\category{D.2.8}{Software Engineering}{Metrics}[complexity measures, performance measures]

\inputtex{sec-introduction.tex}
\inputtex{sec-previous.tex}
\inputtex{sec-our-method.tex}
\inputtex{sec-conclusions.tex}

\bibliographystyle{abbrv}
\bibliography{generic-dispatch}
\end{document}
