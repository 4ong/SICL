\documentclass{acm_proc_article-sp}
\usepackage[utf8]{inputenc}
\usepackage{color}

\def\inputfig#1{\input #1}
\def\inputtex#1{\input #1}
\def\inputal#1{\input #1}
\def\inputcode#1{\input #1}

\inputtex{logos.tex}
\inputtex{refmacros.tex}
\inputtex{other-macros.tex}

\begin{document}
\title{Fast generic dispatch for Common Lisp}
\numberofauthors{1}
\author{\alignauthor
Robert Strandh\\
\affaddr{University of Bordeaux}\\
\affaddr{351, Cours de la Libération}\\
\affaddr{Talence, France}\\
\email{robert.strandh@u-bordeaux1.fr}}

\maketitle

\begin{abstract}
We describe a technique for generic dispatch that is adapted to modern
computers where accessing memory is potentially quite expensive.
Instead of the traditional hashing scheme used by PCL
\cite{Kiczales:1990:EMD:91556.91600}, we assign a \emph{unique number}
to each class and the dispatch consists of comparisons of the number
assigned to an instance to a certain number of (usually small)
constant integers.  While our implementation (SICL) is not yet in a
state where we are able to get exact performance figures, a
conservative simulation suggests that our technique is significantly
faster than the one used in SBCL, which uses PCL, and indeed than the
technique used by most high-performance \cl{} implementations.
Furthermore, existing work \cite{Zendra:1997:EDD:263698.263728} using
a similar technique in the context of static languages suggests that
perfomance can improve significantly compared to table-based
techniques.
\end{abstract}

\category{D.3.4}{Programming Languages}{Processors}
[Code generation, Optimization, Run-time environments]

\inputtex{sec-introduction.tex}
\inputtex{sec-previous.tex}
\inputtex{sec-our-method.tex}
\inputtex{sec-performance.tex}
\inputtex{sec-conclusions.tex}
\inputtex{sec-acknowledgements.tex}

\bibliographystyle{abbrv}
\bibliography{generic-dispatch}
\end{document}
