\section{Our method}

A SICL object is either an \emph{immediate object} or a \emph{heap
  object}.  A heap object is either a \texttt{cons} cell or a
\emph{general instance}.  General instances and cons cells have unique
\emph{tags}.  Every general instance is represented by its
\emph{header}.  The header contains two pointers, one to the
\emph{class} of the instance and one to the \emph{rack} of the
instance.  The pointer to the class is a tagged pointer to another
general instance.  The pointer to the rack is a raw machine pointer.  
This representation is shown in \refFig{fig-general-instance}.

\begin{figure}
\begin{center}
\inputfig{fig-general-instance.pdf_t}
\end{center}
\caption{\label{fig-general-instance}
Representation of a general instance.}
\end{figure}

Each class is assigned a \emph{unique number} starting at $0$.  The
number is assigned when the class is finalized, and a new number is
assigned whenever a class is finalized as a result of changes to the
class or any of its superclasses.  Currently, class numbers are never
reused.  This way of allocating class numbers is advantageous because
it often results in a subtree of classes occupying a \emph{dense
  interval} of class numbers, the importance of which is discussed
below. 

The first element of the rack of every general instance is called the
\emph{stamp}.  The stamp is the unique number of the class as it was
when the instance was created updated as a result of changes to its
class.  An instance is \emph{obsolete} if and only if its stamp is not
the same as the unique number of its class.
