\section{Previous work}

\subsection{SBCL}

The \sbcl{} \commonlisp{} implementation unsurprisingly defines the
class \texttt{method\--combination} and then the class
\texttt{standard\--method\--combination} as a subclass of
the class named \texttt{method\--combination}.

More surprisingly, it then defines two subclasses of the class
\texttt{standard\--method\--combination}, namely
\texttt{long\--method\--combination} and
\texttt{short\--method\--combination}, each for use with the different
forms (long and short) of the macro
\texttt{define\--method\--combination}.

\subsection{ECL}

The \ecl{} \commonlisp{} implementation defines the class
\texttt{method-combination}, and method-combination metaobjects are
direct instances of this class, which is not conforming.

The macro \texttt{define-method-combination} does not define a method
combination, nor a new method-combination class.  Instead it defines a
function that we will call the \emph{method-combination function}.
This function computes the effective method of a generic function.
The lambda list of the method-combination function consists of two
required parameters, namely a generic function and a list of
applicable methods, followed by the lambda list given to
\texttt{define-method-combination}.  That lambda list will contain
\texttt{\&optional (order :most-specific-first)} for most built-in
method-combination types.  The resulting function is stored in a hash
table with the name of the method-combination type as a key.

When a generic function is created, a new instance of the
\texttt{method-combination} class is created.  The new instance
contains the method-combination function and a list of the options
given after the method-combination name in the
\texttt{:method-combination} option to \texttt{defgeneric}.

Since a new instance is created each time, two different generic
function that, when created, were given the same method-combination
name and the same method-combination arguments, will have two
different instances of the class \texttt{method-combination}.
