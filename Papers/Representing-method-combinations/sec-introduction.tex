\section{Introduction}

The \commonlisp{} standard contains very little information about
method combinations.  The dictionary entry for the system class
\texttt{method-combination} requires a \emph{method combination
  object} to be an \emph{indirect instance} of the system class named
\texttt{method-combination}.  The standard further requires such an
object to contain information both about the \emph{type} of method
combination and the \emph{arguments} used with that type.

The term \emph{indirect instance}, as explained in the glossary,
excludes the possibility of such a method combination object to be an
immediate instance of the class \texttt{method-combination}.  We can
interpret this requirement as the need to create a subclass, say,
\texttt{standard-method-combination} to parallel the situation for
\texttt{method} vs \texttt{standard-method} and
\texttt{generic-function} vs \texttt{standard-generic-function}, i.e.,
so as to allow the programmer to create very different objects from
those that the \texttt{standard-} version can provide.

Clearly, the text of the dictionary entry means that when the macro
\texttt{defgeneric} is used with the \texttt{:method-combination}
option given, such a method combination object is what the generic
function will contain.  We can confirm this view by examining the
description of the MOP generic function
\texttt{generic-function-method-combination} which states that the
return value is ``a method combination metaobject''.

However, the macro \texttt{define-method-combination} does \emph{not}
define a method combination object.  The reason is of course that no
arguments are supplied to this macro.  The dictionary entry for this
macro also clearly says that the macro is used to define new
\emph{types} of method combination.

The main issue for the person implementing a \commonlisp{} system,
then, is how to interpret the relation between a \emph{method
  combination type} and a \emph{method combination object}.

It is easy to draw the conclusion that a call to the macro
\texttt{define-method-combination} creates a new \emph{class}, as
suggested by the use of the word \emph{type} in the standard, and that
method combination objects of that type are instances of the new
class.  However, this view creates several problems.  In particular,
one must then determine whether each use of the same combination of
the type and the arguments in the \texttt{:method-combination} option
to \texttt{defgeneric} creates a new instance of the class, or whether
existing instances are somehow kept track of and reused.  The first
possibility would have the unfortunate consequence that two calls to
\texttt{generic-function-method-combination} with different
generic-function metaobjects would return two method combination
objects that are not identical.

In this paper, we argue that a \emph{method combination type} is
itself an instance of a completely different class that we shall call
\texttt{method-combination-template}, and that a \emph{method
  combination object} is a \emph{variant} of the template in that it
contains a reference to the template as well as the values of the
\emph{arguments} that this particular method combination type allows.
To conform to the standard, we obviously maintain that method
combination objects are instances of
\texttt{standard-method-combination}.

When a generic function is defined or redefined, it would be desirable
to have the options of the \texttt{method-combination}
\texttt{defgeneric} option checked for validity when the definition or
redefinition occurs.  For the built-in method combination types, most
implementations also handle this check as a special case.  However,
all implementations we have investigated fail to check the options to
a user-defined method-combination type defined by the long-form of
\texttt{define-method-combination}.  Instead, if the options are
incompatible with the define method-combination type, in the best
case, an error is signaled when the method-combination is applied to a
list of applicable methods by the generic function
\texttt{compute-effective-method}.  Furthermore, the error being
signaled can be hard to decipher, as it typically results from invalid
arguments to a function with a particular lambda list.

In this paper, we propose a general mechanism for early detection of
incompatible options to a particular method-combination type.  This
mechanism is available to the creator of custom method-combination
types, and also used to verify the options to the built-in
method-combination types.
