\section{Our technique}

Our technique consists of applying \emph{local graph rewriting} to the
graph of instructions in intermediate code.  Local graph rewriting has
the advantage of being simple, both to implement and when it comes to
proving correctness.

For the purpose of this paper, we assume that some initial phase has
determined that:

\begin{enumerate}
\item there are two instructions, $D$ and $I$, in the program that
  are identical tests,
\item the variable being tested is the same in $D$ and $I$,
\item $D$ dominates $I$, and
\item the variable being tested is not assigned to in any path from
  $D$ to $I$.
\end{enumerate}

In a real compiler, such a phase probably does not exist.  Some items
are simplified if the compiler translates the intermediate code to SSA
form \cite{Cytron:1989:EMC:75277.75280,
  Cytron:1991:ECS:115372.115320}, and some items can be tested during
the execution of our technique.

During the execution of our algorithm, the instruction $I$ will be
\emph{replicated}, so that it is part of some set $S$ in which every
replica remains dominated by $D$.

In our technique, we keep track of the outcome of the test in the
\emph{control arcs} of the graph of intermediate instructions.
We can think of this information as being represented as three
different \emph{colors}:

\begin{itemize}
\item An arc is \emph{black} if we have no information concerning the
  outcome of the test at that point in the program.
\item An arc is \emph{green} if the outcome of the test at that point in
  the program is known to be \emph{true}.
\item An arc is \emph{red} if the outcome of the test at that point in
  the program is known to be \emph{false}.
\end{itemize}

Initially, only the outgoing arcs of $D$ and $I$ have a color
other than black.

Our technique involves the repeated application of the first
applicable rewrite rule in the following list to some arbitrary
element of $S$, say $s$.

\begin{enumerate}
\item If $s$ has no predecessors, then remove it from $S$.
\item If $s$ has a \emph{green} incoming arc, then change the head
  of that arc so that it refers to the successor of $s$ referred to
  by the \emph{green} outgoing arc of $s$.
\item If $s$ has a \emph{red} incoming arc, then change the head
  of that arc so that it refers to the successor of $s$ referred to
  by the \emph{red} outgoing arc of $s$.
\item If $s$ has $n>1$ predecessors, then replicate $s$ $n$ times;
  once for each predecessor.  Every replica is inserted into $S$.
  Colors of outgoing control arcs are preserved in the replicas.
\item Let $p$ be the (unique) predecessor of $s$.  Remove $p$ as a
  predecessor of $s$ so that existing immediate predecessors of $p$
  instead become immediate predecessors of $s$.  Insert a replica of
  $p$ in each outgoing control arc of $s$, preserving the color of
  each arc.
\end{enumerate}

Rewrite rules are applied until the set $S$ is empty.
