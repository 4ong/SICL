\section{Our technique}

\subsection{General description}

Our technique consists of applying \emph{local graph rewriting} to the
graph of instructions in intermediate code.  Local graph rewriting has
the advantage of being simple, both to implement and when it comes to
proving correctness.

For the purpose of this paper, we assume that some initial phase has
determined that:

\begin{enumerate}
\item there are two instructions, $D$ and $I$, in the program that
  are identical tests,
\item the variable being tested is the same in $D$ and $I$,
\item $D$ dominates $I$, and
\item the variable being tested is not assigned to in any path from
  $D$ to $I$.
\end{enumerate}

In a real compiler, such a phase probably does not exist.  Some steps
of the phase are simplified if the compiler translates the
intermediate code to SSA form \cite{Cytron:1989:EMC:75277.75280,
  Cytron:1991:ECS:115372.115320}, and the tests in some of the steps
can be determined during the execution of our technique, avoiding the
need to include them in a separate phase.

During the execution of our algorithm, the instruction $I$ will be
\emph{replicated}, so that it is part of some set $S$ in which every
replica remains dominated by $D$.

In our technique, we keep track of the outcome of the test in the
\emph{control arcs} of the graph of intermediate instructions.
We can think of this information as being represented as three
different \emph{colors}:

\begin{itemize}
\item An arc is \emph{black} if we have no information concerning the
  outcome of the test at that point in the program.
\item An arc is \emph{green} if the outcome of the test at that point in
  the program is known to be \emph{true}.
\item An arc is \emph{red} if the outcome of the test at that point in
  the program is known to be \emph{false}.
\end{itemize}

Initially, only the outgoing arcs of $D$ and $I$ have a color
other than black.

Our technique involves the repeated application of the first
applicable rewrite rule in the following list to some arbitrary
element of $S$, say $s$, that does not itself have an immediate
predecessor in $S$:

\begin{enumerate}
\item If $s$ has no predecessors, then remove it from $S$.
\item If $s$ has a \emph{green} incoming arc, then change the head
  of that arc so that it refers to the successor of $s$ referred to
  by the \emph{green} outgoing arc of $s$.
\item If $s$ has a \emph{red} incoming arc, then change the head
  of that arc so that it refers to the successor of $s$ referred to
  by the \emph{red} outgoing arc of $s$.
\item If $s$ has $n>1$ predecessors, then replicate $s$ $n$ times;
  once for each predecessor.  Every replica is inserted into $S$.
  Colors of outgoing control arcs are preserved in the replicas.
\item Let $p$ be the (unique) predecessor of $s$.  Remove $p$ as a
  predecessor of $s$ so that existing immediate predecessors of $p$
  instead become immediate predecessors of $s$.  Insert a replica of
  $p$ in each outgoing control arc of $s$, preserving the color of
  each arc.
\end{enumerate}

\noindent
Rewrite rules are applied until the set $S$ is empty, or until each
element of $S$ has an immediate predecessor in $S$.

\subsection{A simple example}

Let us see how our technique works on the example in
\refSec{sec-introduction}.  The initial situation is shown in
\refFig{fig-rewrite-1}.

\begin{figure}
\begin{center}
\inputfig{fig-rewrite-1.pdf_t}
\end{center}
\caption{\label{fig-rewrite-1}
Initial instruction graph.}
\end{figure}

As \refFig{fig-rewrite-1} shows, the second \texttt{consp} is
dominated by the first, so it becomes the only member of the set $S$.
The last rewrite rule applies to the second \texttt{consp} so that the
\texttt{setq} is replicated as its successors.  Then the last rewrite
rule applies again resulting in the replication of the \texttt{call}.
The result of these two applications is shown in \refFig{fig-rewrite-2}.

\begin{figure}
\begin{center}
\inputfig{fig-rewrite-2.pdf_t}
\end{center}
\caption{\label{fig-rewrite-2}
Result after two rewrites.}
\end{figure}

As we can see in \refFig{fig-rewrite-2}, the second \texttt{consp}
now has two predecessors, and both incoming arcs are black.
Therefore, rewrite rule number $4$ applies and the \texttt{consp} is
replicated.  As a result, $S$ now has two members.  The result of
applying this rule is shown in \refFig{fig-rewrite-3}.

\begin{figure}
\begin{center}
\inputfig{fig-rewrite-3.pdf_t}
\end{center}
\caption{\label{fig-rewrite-3}
Result after replicating the test.}
\end{figure}

We now choose the leftmost replica of the second \texttt{consp} to
apply our rules to.  It has a single predecessor with a black incoming
control arc, so the last rewrite rule applies.  We replicate the
\texttt{setq} in both branches of the test, giving us the result shown
in \refFig{fig-rewrite-4}.

\begin{figure}
\begin{center}
\inputfig{fig-rewrite-4.pdf_t}
\end{center}
\caption{\label{fig-rewrite-4}
Result after replicating \texttt{setq}.}
\end{figure}

In \refFig{fig-rewrite-4}, the last rewrite rule applies again, and we
replicate the \texttt{cons-car}, giving us the situation shown in in
\refFig{fig-rewrite-5}.

\begin{figure}
\begin{center}
\inputfig{fig-rewrite-5.pdf_t}
\end{center}
\caption{\label{fig-rewrite-5}
Result after replicating \texttt{cons-car}.}
\end{figure}

As \refFig{fig-rewrite-5} shows, the \texttt{consp} instruction now
has a single predecessor, but the incoming arc has a known outcome of
the test, namely \textit{true} (color green).  therefore, rewrite rule
number $2$ applies.  The left outgoing arc of the first \texttt{consp}
is redirected to go directly to the \texttt{cons-car} instruction.
The result of applying this rule is shown in \refFig{fig-rewrite-6}.

\begin{figure}
\begin{center}
\inputfig{fig-rewrite-6.pdf_t}
\end{center}
\caption{\label{fig-rewrite-6}
Result after short-circuit \texttt{consp}.}
\end{figure}

At this point, the \texttt{consp} that we have been processing has no
predecessor so it is removed from $S$.  Removing all instructions that
can not be reached from the start instruction gives the situation
shown in \refFig{fig-rewrite-7}.

\begin{figure}
\begin{center}
\inputfig{fig-rewrite-7.pdf_t}
\end{center}
\caption{\label{fig-rewrite-7}
Result after removing unreachable instructions.}
\end{figure}

Analyzing \refFig{fig-rewrite-7}, we can see that if the result of the
first \texttt{consp} yields \emph{true}, then no second test is
performed.  Instead, the variable \texttt{a} is set to the result of
the instruction \texttt{cons-car}, the variable \texttt{b} is set to
the result of the call, and the variable \texttt{c} is set to the
result of the instruction \texttt{cons-cdr}.  Applying the same rules
to the remaining \texttt{consp} instruction in $S$ and then to the
second \texttt{null} instruction (which is now dominated by the
first), yields the final result shown in  \refFig{fig-rewrite-8}.

\begin{figure}
\begin{center}
\inputfig{fig-rewrite-8.pdf_t}
\end{center}
\caption{\label{fig-rewrite-8}
Final result.}
\end{figure}

This example represents a control graph that is particularly simple in
that there are no loops between the first and the second
\texttt{consp} instructions.  Our technique must obviously work no
matter the complexity of the control graph, as long as the first test
dominates the second.

\subsection{Proof of correctness and termination}

The correctness of our technique is easy to prove, simply because each
rewrite rule preserve the semantics of the program.  The last rewrite
rule preserves the semantics only under certain circumstances which
are easy to verify:

\begin{itemize}
\item The predecessor does not write to a lexical variable that is
  read by the test instruction.  This condition is respected because
  we have assumed that the variable being tested is not written to
  in any path between the first and the second occurrences of the
  test.
\item The predecessor must not have any other side effect that may
  alter the outcome of the test.  By restricting the test to lexical
  variables, this restriction is also respected.
\end{itemize}

Termination is a bit harder to prove.  One way is to find some
non-negative \emph{metric} that can be shown to strictly decrease as a
result of the application of each rewrite rule.  We have not found any
such metric.  However, this conundrum can be avoided by a simple
\emph{grouping} of the rewrite rules.  This grouping is not required
to be present in the implementation of our technique; only in the
termination proof.

To see how the rewrite rules can be grouped, consider a general case
where the test instruction has some arbitrary number of red, green,
and black incoming control arcs.  Rules number $2$ and $3$ are first
applied a finite number of times.  What happens next depends on the
number $n$ of black incoming control arcs:

\begin{itemize}
\item If $n=0$ the first rewrite
rule applies, in which case the instruction is removed from the set
$S$.
\item If $n=1$, the last rewrite rule is applied.  The crucial
  characteristic of this rewrite rule is that the total number of
  black control arcs decreases by one.
\item If $n>1$, rewrite rule number $4$ is applied.  Notice that the
  number of black control arcs is not modified by the application of
  this rule.
\end{itemize}

For the purpose of this proof, we assume that these steps happen
immediately after each other, so that for a particular instruction,
the green and red incoming control arcs are first eliminated, the same
instruction is then potentially replicated, and finally, the last
rewrite rule is applied to one of the replicas.  However, the
implementation does not have to work that way in order for termination
to be certain.

In other words, we can create groups of rewrite steps, where a group
can be formed in one of the following ways:

\begin{enumerate}
\item It has a finite number of applications of rewrite rules number
  $2$ and $3$, followed by a single application of rewrite rule number
  $1$.
\item It has a finite number of applications of rewrite rules number
  $2$ and $3$, followed by a single application of rewrite rule number
  $5$.
\item It has a finite number of applications of rewrite rules number
  $2$ and $3$, followed by a single application of rewrite rule number
  $4$, followed by a single application of rewrite rule number $5$.
\end{enumerate}

With this information, we can create a metric consisting of a pair
$(B,N)$, where $B$ is the total number of black control arcs of the
program and $N$ is the number of elements of the set $S$.  For two
pairs $(B_1,N_1)$ and $(B_2,N_2)$, $(B_1,N_1)$ is \emph{strictly
  smaller than} $(B_2,N_2)$, written $(B_1,N_1) < (B_2,N_2)$, if and
only if either $B_1 < B_2$ or $B_1 = B_2$ and $N_1 < N_2$.

Theorem: The rewrite algorithm terminates.

Proof: As a result of a rewrite according to a group of type $1$, $B$
remains the same, but $N$ decreases by $1$.  As a result of a rewrite
according to a group of type $2$ or $3$, $B$ decreases by $1$ (but $N$
may increase).
