\section{Conclusions and future work}

We have defined a technique for eliminating redundant tests in
intermediate code.  The technique relies on replication of code paths
between two identical tests.  So far, our technique only defines a
\emph{mechanism} for achieving the result.  It does not yet define a
\emph{policy} stating when the technique should be applied.

The question of policy is an important one, because with a large
number of redundant tests in the intermediate code, there is a
possibility for \emph{exponential blowup} of the code size.
Future work involves defining a reasonable policy to avoid such
pathological cases.

The technique described in this paper will become available as one of
the optimization techniques provided by the Cleavir compiler framework
that is currently part of the \sicl{} project.%
\footnote{See https://github.com/robert-strandh/SICL}  Only then will
it be possible to determine the exact characteristics of our technique in
terms of applicability, computational cost, performance gain of
compiled code, and size increase of typical programs.
