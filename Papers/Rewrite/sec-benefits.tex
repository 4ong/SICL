\section{Benefits of our technique}

The main benefit of our technique is its simplicity.  This simplicity
is important both in terms of its implementation and in terms of
ensuring termination for all possible control graphs.

As the example in \refSec{sec-our-technique-example} shows, redundant
tests can be avoided in cases where it is not possible to express this
redundancy by the use of any portable lower-level constructs.
Situations similar to the one in the example occur naturally in many
programs:

\begin{itemize}
\item If the same variable is used in more than one consecutive
  numeric operation, then there will be redundant tests to determine
  the exact numeric subtype of the contents of that variable.  An
  important special case is to determine whether a particular value is
  of type \texttt{fixnum}.
\item If a variable holding an array is used in more than one
  consecutive operation to access some element, then there will be
  redundant tests to determine various aspects of the array that
  influence the way the indices and the elements are handled, such as
  the upgraded element type and whether the array is simple or not.
\end{itemize}

A particularly interesting special case of these situations occurs
when the dominated test is part of a loop.  It is particularly
interesting, because our technique will then eliminate a test that
would otherwise potentially be executed a large number of times.  This
feature can be taken advantage of in a highly portable version of some
of the \commonlisp{} \emph{sequence functions}.  By duplicating a very
general loop in every branch of a multiway test for keyword arguments
such as \texttt{test}, \texttt{test-not}, and \texttt{end}, each copy
of the loop will automatically be simplified differently according to
the particular branch it occurs in.

Our technique has some disadvantages as well.  First of all, the size
of the code increases, which can have a negative influence on cache
performance, especially when different invocations of the code result
in different results of the test.  In fact, if several variables with
overlapping regions of liveness are processes by our technique, the
result may be an \emph{exponential blowup} of the size of the code in
the overlapping region.  It is outside the scope of this paper to
discuss heuristics that will determine the conditions for applying our
technique, but such conditions are required to avoid such problematic
effects.

The increase of the size of the code automatically means longer
compilation times as well.  Techniques that work on global information
about the program can avoid some of these disadvantages, at the cost
of increased complexity compared to our simple local rewrite
technique.
