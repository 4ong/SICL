\section{Introduction}

In a language such as \commonlisp{}, it is hard to avoid redundant
tests, even if the programmer goes to great lengths to avoid such
redundancies.  The reason is that even the lowest-level operators in
\commonlisp{} require \emph{type checks} to determine the exact way to
accomplish the operation, so that two or more calls to similar
operators may introduce redundant tests that are impossible to
eliminate manually.

As an example of such an introduction of redundant tests, consider the
basic list operators \texttt{car} and \texttt{cdr}.  We can think of
these operators to be defined%
\footnote{For these particular operators, an implementation may use
  some tricks to avoid some of these tests, but such tricks are not
  generally usable for other operators.  We still prefer to use
  \texttt{car} and \texttt{cdr} as examples, because their definitions
  are easy to understand.}
in a way similar to the code below:

\begin{verbatim}
  (defun car (x)
    (cond ((consp x) (cons-car x))
          ((null x) nil)
          (t (error 'type-error ...))))

  (defun cdr (x)
    (cond ((consp x) (cons-cdr x))
          ((null x) nil)
          (t (error 'type-error ...))))
\end{verbatim}

\noindent
where \texttt{cons-car} and \texttt{cons-cdr} are operations that
assume that the argument is of type \texttt{cons}.  These operations
are implementation defined and not available to the application
programmer.

Now consider some typical%
\footnote{In this case, the programmer might use the standard macro
  \texttt{destructuring-bind}, but for reasons of simplicity, that
  macro will very likely expand to calls to \texttt{car} and
  \texttt{cdr}, rather than to some implementation-specific code that
  avoids the redundant tests.}
use of \texttt{car} and \texttt{cdr} such as
in the following code:

\begin{verbatim}
  (let ((a (car x))
        (b (cdr x)))
    ...)
\end{verbatim}

After inlining of the \texttt{car} and \texttt{cdr} operations, the
code looks like this:

\begin{verbatim}
  (let ((a (cond ((consp x) (cons-car x))
                 ((null x) nil)
                 (t (error 'type-error ...)))
        (b (cond ((consp x) (cons-cdr x))
                 ((null x) nil)
                 (t (error 'type-error ...)))
    ...)
\end{verbatim}

We notice that the test for \texttt{consp} occurs twice, and that the
second occurrence is \emph{dominated by} the first one, i.e., every
control path leading to the second occurrence must pass by the first
occurrence as well.
