\section{Introduction}

The \commonlisp{} \cite{ansi:common:lisp} sequence functions are
challenging to implement for several reasons:

\begin{itemize}
\item They take several keyword parameters that modify the behavior in
  different ways.  Several special cases must therefore be taken into
  account according to the value of these keyword parameters.
\item In order for performance to be acceptable, different variations
  may have to be implemented according to the type of the sequence to
  be traversed.
\item When the sequence is a vector, it may be necessary to specialize
  the implementation according to the element type of the vector, and
  according to whether the vector is a \emph{simple array} or not.
\end{itemize}

For reasons of maintainability, it is advantageous to create a small
number of versions, each one containing a single loop over the
relevant elements.  In each iteration of the loop, tests would
determine the exact action based on current values of keyword
arguments.  In the case of a vector, the general array accessor
\texttt{aref} would be used to access the elements.

On the other hand, for reasons of performance, it is preferable to
create a large number of different versions of each function, each
version being specialized according to the exact values of the keyword
arguments given.  In the case of a vector, it is also advantageous to
have versions specialized to the available element types provided by
the implementation.  However, in this case, maintenance is
problematic, because each version has to be maintained and tested
separately.

A compromise used by some implementations is to use the \commonlisp{}
macro system to abstract some of the specialization parameters as
macro arguments.  With this technique, a special version is created by
a call to some general macro, providing different cases for keyword
parameters, element types, test functions, etc.  We find that this
technique results in code that is extremely hard to understand, and
therefore to be perceived as correct by maintainers.

In this paper, we present a different technique.  We use the
\commonlisp{} macro system, but not in order to create macros that,
when called, create special versions of a sequence function.  Instead,
our technique makes it possible to write very few versions of each
sequence function, thus keeping a high degree of maintainability.
Most of our macros have no apparent role in our functions, so do not
require the maintainer to understand them.  Instead, they serve the
sole purpose of allowing the compiler to generate efficient code.
