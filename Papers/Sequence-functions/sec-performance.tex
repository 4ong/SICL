\section{Properties of our technique}

\subsection{Performance}

\subsubsection{Results on vectors}
\inputeps{VECTOR-EGAL-BIT.eps}
\inputeps{VECTOR-EQL-CHAR.eps}
\inputeps{VECTOR-EGAL-UB8.eps}

\subsubsection{Results on lists}
\inputeps{LIST-EGAL-BIGNUM.eps}
\inputeps{LIST-EGAL-CAR.eps}
\inputeps{LIST-EGAL-ENDP.eps}

\subsection{Maintainability}

From the point of view of maintainability, there are clear advantages
to our technique.  With only a small amount of macro code, we are able
to hide the implementation details of the functions, without
sacrificing performance.

The small amount of macro code that is needed to make our technique
work is clearly offset by the considerable decrease in the code size
that would otherwise have been required in order to obtain similar
performance.

\subsection{Disadvantages}

There are not only advantages to our technique.

For one thing, compilation times are fairly long, for the simple
reason that the body of the function is duplicated a large number of
times.  Ultimately, the compiler eliminates most of the code, but
initially the code is fairly large.  And the compiler must do a
significant amount of work to determine what code can be eliminated.
To give an idea of the orders of magnitude, in order to obtain
fully-expanded code on SBCL, we had to increase the inline limit from
$100$ to $10\thinspace000$, resulting in a compilation time of tens of
seconds for a single function.

Another disadvantage of our technique is that it doesn't lend itself
to applications using short sequences.  For such applications, it
would be advantageous to inline the sequence functions, but doing so
would make each call site suffer the same long compilation times as we
now observe for the ordinary callable functions.

Not all compilers are able to optimize the main body of a function
according to some enclosing condition.  For a \commonlisp{}
implementation with a more basic compiler, no performance improvement
would be observed.  In addition, the duplication of the main body of
the function would result in a very large increase of the code size,
compared to a simpler code with the same performance.

For the special case of bit vectors, our technique will not be able to
compete with a good native implementation of the sequence functions.
The reason is that, despite the optimizations that the compiler can
perform with out technique, the body of a typical sequence function
still consists of a loop where each iteration treats a single
element.  For bit vectors, a good native implementation would not
treat a single element in an iteration.  Instead, it would take
advantage of instructions that exist in most processors for handling
an entire \emph{word} at a time, which on a modern processor
translates to $64$ bits.  An implementation that uses our technique
would then typically handle bit vectors as a special case, excluded
from the general technique.
