\section{Previous work}

Most implementations process list elements in reverse order when
\texttt{:from-end} is true only when the specification requires it,
i.e., only for the functions \texttt{count} and \texttt{reduce}.

We designed a technique \cite{Durand:2015:ELS:reverse} that allows us
to always process list elements in reverse order very efficiently when
\texttt{:from-end} is true.  Since that paper contains an in-depth
description of our technique, and in order to keep the presentation
simple, in this paper, no example traverses the sequence from the end.

\subsection{ECL and Clasp}

The sequence functions of ECL have a similar superficial structure to
ours, in that they take advantage of custom macros for managing common
aspects of many functions such as the interaction between the
\texttt{test} and \texttt{test-not} keyword arguments, the existence
of keyword arguments \texttt{start} and \texttt{and}, etc.
But these macros just provide convenient syntax for handling shared
aspects of the sequence functions.  They do not assist the compiler
with the optimization of the body of the code.

For functions for which the \commonlisp{} specification allows the
implementation to process elements from the beginning of the sequence
even when \texttt{from-end} is \emph{true}, ECL takes advantage of
this possibility.  For the \texttt{count} function applied to a list,
ECL simply reverses the list before processing the elements.

The \commonlisp{} code base of Clasp is derived from that of ECL, and
the code for the sequence functions of Clasp is the same as that of
ECL.

\subsection{CLISP}

The essence of the code of the sequence functions of CLISP are written
in \clanguage{}, which makes them highly dependent on that particular
implementation.  For that reason, CLISP is outside the scope of this
paper.
