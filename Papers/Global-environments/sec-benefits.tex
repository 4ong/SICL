\section{Benefits of our method}
 
\subsection{Cross compilation}

\subsection{Native compilation}

The \commonlisp{} standards suggests that the \emph{startup
  environment} and the \emph{evaluation environment} may be
different.%
\footnote{Recall that the startup environment is the global
environment as it was when the compilation was initiated, and that the
evaluation environment is the global environment in which evaluations
initiated by the compiler are accomplished.}
Our method allows most evaluations by the compiler to have no
influence in the startup environment.  It suffices to \emph{clone} the
startup environment in order to obtains the evaluation environment. 

With the method described in the previous section, some evaluations
by the compiler would have side effects in the startup environment.
In particular, the value cells and function cells are shared.
Therefore, executing code at compile time that alters the global
binding of a function or a variable will also be seen in the startup
environment.  

\subsection{Bootstrapping}

\subsection{Sanboxing}

It is notoriously hard to create a so-called \emph{sandbox
  environment} for \commonlisp{}, i.e., an environment that contains
a subset of the full languages.  A typical use case would be to
propose a Read-Eval-Print Loop accessible through a web interface for
educational purposes.  Such a sandbox environment is hard to achieve
because functions such as \texttt{eval} and \texttt{compile} would
have to be removed so that the environment could not be destroyed by a
careless user.  However, these functions are typically used by parts
of the system.  For example, \clos{} might need the compiler in order
to generate dispatch code.

The root of the problem is that in \commonlisp{} there is always a way
for the user of a Read-Eval-Print Loop to access every global function
in the system, including the compiler.

Using first-class global environments solves this problem in an
elegant way.  It suffices to propose a restricted environment in which
there is no binding from the names \texttt{eval} and \texttt{compile}
to the corresponding functions.  These functions can still be
available in some other environment for use by the system itself.

%%  LocalWords:  Sanboxing startup
