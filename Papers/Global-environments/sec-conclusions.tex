\section{Conclusions and future work}

In this paper, we have advocated first-class global environments as a
way of implementing the global environments mentioned in the \hs{}.
We have seen that this technique has several advantages in terms of
flexibility of the system, and that it greatly simplifies certain
difficult problems such as bootstrapping and sandboxing.

We also think that first-class global environments could be an
excellent basis for a multi-user \commonlisp{} system.

In such a system, each user would have an initial, private,
environment.  That environment would contain the standard
\commonlisp{} functionality.  Most standard \commonlisp{} functions
would be shared between all users.  Some functions, such as
\texttt{print-object} or \texttt{initialize-instance} would not be
shared, so as to allow individual users to add methods to them without
affecting other users.

Furthermore, functionality that could destroy the integrity of the
system, such as access to raw memory, would be accessible in an
environment reserved for system maintenance.  This environment would
not be accessible to ordinary users.
