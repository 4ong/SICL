\appendix

\section{Protocol}

\Defgeneric {fboundp} {fname env}

This generic function is a generic version of the \commonlisp{}
function \texttt{cl:fboundp}.

It returns true if \textit{fname} has a definition in
\textit{env} as an ordinary function, a generic function, a
macro, or a special operator.

\Defgeneric {fmakunbound} {fname env}

This generic function is a generic version of the \commonlisp{}
function \texttt{cl:fmakunbound}.

Makes \textit{fname} unbound in the function namespace of
\textit{env}.

If \textit{fname} already has a definition in
\textit{env} as an ordinary function, as a generic function,
as a macro, or as a special operator, then that definition is lost.

If \textit{fname} has a \texttt{setf} expander associated with
it, then that \texttt{setf} expander is lost.

\Defgeneric {special-operator} {fname env}

if \textit{fname} has a definition as a special operator in
\textit{env}, then that definition is returned.  The definition is an
the object that was used as an argument to \texttt{(setf
  special-operator)}.  The exact nature of this object is not
specified, other than that it can not be \texttt{nil}.  If
\textit{fname} does not have a definition as a special operator in
\textit{env}, then \texttt{nil} is returned.

\Defgeneric {(setf special-operator)} {new-def fname env}

Set the definition of \textit{fname} to be a special operator.  The
exact nature of \textit{new-def} is not specified, except that a value
of \texttt{nil} means that \textit{fname} no longer has a definition
as a special operator in \textit{env}.

If a value other than \texttt{nil} is given for \textit{new-def}, and
\textit{fname} already has a definition as an ordinary function, as a
generic function, or as a macro, then an error is signaled.  As a
consequence, if it is desirable for \textit{fname} to have a
definition both as a special operator and as a macro, then the
definition as a special operator should be set first.

\Defgeneric {fdefinition} {fname env}

This generic function is a generic version of the \commonlisp{}
function \texttt{cl:fdefinition}.

If \textit{fname} has a definition in the function namespece of
\textit{env} (i.e., if \texttt{fboundp} returns true), then a call to
this function succeeds.  Otherwise an error of type
\texttt{undefined-function} is signaled.

If \textit{fname} is defined as an ordinary function or a generic
function, then a call to this function returns the associated
function object.

If \textit{fname} is defined as a macro, then a list of the form
\texttt{(cl:macro-function \textrm{\textit{function}})} is returned, where
\textit{function} is the macro expansion function associated with the
macro.

If \textit{fname} is defined as a special operator, then a list of the
form \texttt{(cl:special \textrm{\textit{object}})} is returned, where
  the nature of \textit{object} is currently not specified.

\Defgeneric {(setf fdefinition)} {new-def fname env}

This generic function is a generic version of the \commonlisp{}
function \texttt{cl:fdefinition}.

\textit{new-def} must be a function.  If \textit{fname} already names
a function or a macro, then the previous definition is lost.  If
\textit{fname} already names a special operator, then an error is
signaled.

If \textit{fname} is a symbol and it has an associated \texttt{setf}
expander, then that \texttt{setf} expander is preserved.

\Defgeneric {macro-function} {symbol env}

This generic function is a generic version of the \commonlisp{}
function \texttt{cl:macro-function}.

If \textit{symbol} has a definition as a macro in \textit{env}, then the
corresponding macro expansion function is returned.

If \textit{symbol} has no definition in the function namespace of
\textit{env}, or if the definition is not a macro, then this
function returns \texttt{nil}.


\Defgeneric {(setf macro-function)} {new-def symbol env}

This generic function is a generic version of the \commonlisp{}
function \texttt{(setf cl:macro-function)}.

\textit{new-def} must be a macro expansion function or \texttt{nil}.
A call to this function then always succeeds.  A value of \texttt{nil}
means that the \textit{symbol} no longer has a macro function
associated with it.  If \textit{symbol} already names a macro or a
function, then the previous definition is lost.  If \textit{symbol}
already names a special operator, that definition is kept.

If \textit{symbol} already names a function, then any proclamation of
the type of that function is lost.  In other words, if at some later
point \textit{symbol} is again defined as a function, its proclaimed
type will be T.

If \textit{symbol} already names a function, then any \texttt{inline} or
\texttt{notinline} proclamation of the type of that function is lost.  In other
words, if at some later point \textit{symbol} is again defined as a
function, its proclaimed inline information will be \texttt{nil}.

If \textit{fname} is a symbol and it has an associated \texttt{setf}
expander, then that \texttt{setf} expander is preserved.

\Defgeneric {compiler-macro-function} {fname env}

This generic function is a generic version of the \commonlisp{}
function \texttt{cl:compiler-macro-function}.

If \textit{fname} has a definition as a compiler macro in
\textit{env}, then the corresponding compiler macro function is
returned.

If \textit{fname} has no definition as a compiler macro in
\textit{env}, then this function returns \texttt{nil}.

\Defgeneric {(setf compiler-macro-function)} {new-def fname env}

This generic function is a generic version of the \commonlisp{}
function \texttt{(setf cl:compiler-macro-function)}.

\textit{new-def} can be a compiler macro function or \texttt{nil}.
When it is a compiler macro function, then it establishes
\textit{new-def} as a compiler macro for \textit{fname} and any
existing definition is lost.  A value of \texttt{nil} means that
\textit{fname} no longer has a compiler macro associated with it in
\textit{env}.

\Defgeneric {function-type} {fname env}

This generic function returns the proclaimed type of the function
associated with \textit{fname} in \textit{env}.

If \textit{fname} is not associated with an ordinary function or a
generic function in \textit{env}, then an error is signaled.

If \textit{fname} is associated with an ordinary function or a
generic function in \textit{env}, but no type proclamation for that
function has been made, then this generic function returns T.

\Defgeneric {(setf function-type)} {new-type fname env}

This generic function is used to set the proclaimed type of the
function associated with \textit{fname} in \textit{env} to NEW-TYPE.

If \textit{fname} is associated with a macro or a special operator
in \textit{env}, then an error is signaled.

\Defgeneric {function-inline} {fname env}

This generic function returns the proclaimed inline information of
the function associated with \textit{fname} in \textit{env}.

If \textit{fname} is not associated with an ordinary function or a
generic function in \textit{env}, then an error is signaled.

If \textit{fname} is associated with an ordinary function or a
generic function in \textit{env}, then the return value of this
function is either \texttt{nil}, \texttt{inline}, or \texttt{notinline}.  If no inline
proclamation has been made, then this generic function returns
\texttt{nil}.

\Defgeneric {(setf function-inline)} {new-inline fname env}

This generic function is used to set the proclaimed inline
information of the function associated with \textit{fname} in
\textit{env} to \textit{new-inline}.

\textit{new-inline} must have one of the values \texttt{nil}, \texttt{inline}, or
\texttt{notinline}.

If \textit{fname} is not associated with an ordinary function or a
generic function in \textit{env}, then an error is signaled.

\Defgeneric {function-cell} {fname env}

A call to this function always succeeds.  It returns a CONS cell,
in which the CAR always holds the current definition of the
function named \textit{fname}.  When \textit{fname} has no
definition as a function, the CAR of this cell will contain a
function that, when called, signals an error of type
\texttt{undefined-function}.  The return value of this function is always
the same (in the sense of \texttt{eq}) when it is passed the same (in the
sense of \texttt{equal}) function name and the same (in the sense of \texttt{eq})
environment.

\Defgeneric {function-unbound} {fname env}

A call to this function always succeeds.  It returns a function
that, when called, signals an error of type \texttt{undefined-function}.
When \textit{fname} has no definition as a function, the return
value of this function is the contents of the CONS cell returned
by FUNCTION-CELL.  The return value of this function is always the
same (in the sense of \texttt{eq}) when it is passed the same (in the sense
of \texttt{equal}) function name and the same (in the sense of \texttt{eq})
environment.  Client code can use the return value of this
function to determine whether \textit{fname} is unbound and if so
signal an error when an attempt is made to evaluate the form
(FUNCTION \textit{fname}).

\Defgeneric {function-lambda-list} {fname env}

This function returns two values.  The first value is an ordinary
lambda list, or \texttt{nil} if no lambda list has been defined for
\textit{fname}.  The second value is true if and only if a lambda
list has been defined for \textit{fname}.

\Defgeneric {boundp} {symbol env}

It returns true if \textit{symbol} has a definition in \textit{env} as a
constant variable, as a special variable, or as a symbol macro.
Otherwise, it returns false.

\Defgeneric {constant-variable} {symbol env}

This function is used in order to define \textit{symbol} as a constant
variable in \textit{env}, with \textit{value} as its value.

If \textit{symbol} already has a definition as a special variable or as a
symbol macro in \textit{env}, then an error is signaled.

If \textit{symbol} already has a definition as a constant variable, and its
current value is not \texttt{eql} to \textit{value}, then an error is signaled.

\Defgeneric {(setf constant-variable)} {value symbol env}

This function is used in order to define \textit{symbol} as a constant
variable in \textit{env}, with \textit{value} as its value.

If \textit{symbol} already has a definition as a special variable or as a
symbol macro in \textit{env}, then an error is signaled.

If \textit{symbol} already has a definition as a constant variable, and its
current value is not \texttt{eql} to \textit{value}, then an error is signaled.

\Defgeneric {special-variable} {symbol env}

This function returns two values.  The first value is the value of
\textit{symbol} as a special variable in \textit{env}, or \texttt{nil} if \textit{symbol}
does not have a value as a special variable in \textit{env}.  The
second value is true if \textit{symbol} does have a value as a special
variable in \textit{env} and false otherwise.

Notice that the symbol can have a value even though this function
returns \texttt{nil} and \texttt{nil}.  The first such case is when the symbol has a
value as a constant variable in \textit{env}.  The second case is when
the symbol was assigned a value using \texttt{(setf symbol-value)}
without declaring the variable as SPECIAL.

\Defgeneric {(setf special-variable)} {value symbol env init-p}

This function is used in order to define \textit{symbol} as a special
variable in \textit{env}.

If \textit{symbol} already has a definition as a constant variable or as a
symbol macro in \textit{env}, then an error is signaled.
Otherwise, \textit{symbol} is defined as a special variable in \textit{env}.

If \textit{symbol} already has a definition as a special variable, and
\textit{init-p} is false, then this function has no effect.  The
current value is not altered, or if \textit{symbol} is currently
unbound, then it remains unbound.

If \textit{init-p} is true, then \textit{value} becomes the new
value of the special variable \textit{symbol}.

\Defgeneric {symbol-macro} {symbol env}

This function returns two values.  The first value is a macro
expansion function associated with the symbol macro named by \textit{symbol},
or \texttt{nil} if \textit{symbol} does not have a definition as a symbol
macro.  The second value is the form that \textit{symbol} expands to as a
macro, or \texttt{nil} if symbol does not have a definition as a
symbol macro.

It is guaranteed that the same (in the sense of \texttt{eq}) function is
returned by two consecutive calls as the first argument as long as
the definition of \textit{symbol} does not change.

\Defgeneric {(setf symbol-macro} {expansion symbol env}

This function is used in order to define \textit{symbol} as a symbol macro
with the given \textit{expansion} in \textit{env}.

If \textit{symbol} already has a definition as a constant variable, or
as a special variable, then an error of type \texttt{program-error} is
signaled.

\Defgeneric {variable-type} {symbol env}

This generic function returns the proclaimed type of the variable
associated with \textit{symbol} in \textit{env}.

If \textit{symbol} has a definition as a constant variable in \textit{env},
then the result of calling \texttt{type-of} on its value is returned.

If \textit{symbol} does not have a definition as a constant variable in
\textit{env}, and no previous type proclamation has been made for
\textit{symbol} in \textit{env}, then this function returns T.

\Defgeneric {(setf variable-type)} {new-type symbol env}

This generic function is used to set the proclaimed type of the
variable associated with \textit{symbol} in \textit{env}.

If \textit{symbol} has a definition as a constant variable in \textit{env},
then an error is signaled.

It is meaningful to set the proclaimed type even if \textit{symbol}
has not previously been defined as a special variable or as a symbol
macro, because it is meaningful to use \texttt{(setf symbol-value)} on
such a symbol.

Recall that the \hs{} defines the meaning of proclaiming the
type of a symbol macro.  Therefore, it is meaningful to call this
function when \textit{symbol} has a definition as a symbol macro in
\textit{env}.
