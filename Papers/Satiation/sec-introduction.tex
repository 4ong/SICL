\section{Introduction}

While most \cl{} implementations have their own native \clos{}
implementation, PCL \cite{Bobrow:1986:CML:28697.28700} is still used
in some high-performance implementations, notably \sbcl{}.  PCL was
written so that \clos{} could be added to a pre-\clos{} Common Lisp
implementation such as the one defined in CLtL \cite{Steele:1984:CLL}
without too much effort.  Even \cl{} implementations that do not use
PCL (such as ECL) include \clos{} late in the process of building a
complete system.

\sicl{}\footnote{https://github.com/robert-strandh/SICL} takes a
different approach.  With very few exceptions, \sicl{} is written in
entirely standard \cl{}, and it is designed to be bootstrapped using a
conforming \cl{} implementation, which therefore includes a complete
implementation of \clos{}.  \sicl{} takes advantage of the conforming
host by making extensive use of \clos{}.  In particular, \clos{} is
bootstrapped \emph{first}, using the host \clos{} implementation to
break circularity in definitions. 

The \sicl{} implementation of \clos{} is a truly metacircular
implementation in that very few compromises are necessary because of
bootstrapping or metastability issues.

Since \clos{} is defined as a \clos{} program, there are necessarily
metacircular issues that need to be resolved.  The AMOP
\cite{Kiczales:1991:AMP:574212} divides these issues into two
families:

\begin{itemize}
\item Bootstrapping issues.  The canonical example mentioned in the
  AMOP is that the class \texttt{standard-class} is its own metaclass,
  so it must exist before it is created.  
\item Metastability issues.  Here, the canonical example is the
  function \texttt{slot-value} which must invoke \texttt{slot-value}
  on the metaclass in order to find the slot containing the slot
  descriptors of the instance. 
\end{itemize}
