\documentclass{acm_proc_article-sp}
\usepackage[utf8]{inputenc}

\def\inputfig#1{\input #1}
\def\inputtex#1{\input #1}
\def\inputal#1{\input #1}
\def\inputcode#1{\input #1}

\inputtex{logos.tex}
\inputtex{refmacros.tex}
\inputtex{other-macros.tex}

\begin{document}
\title{Resolving Metastability Issues During Bootstrapping}
\numberofauthors{1}
\author{\alignauthor
Robert Strandh\\
\affaddr{University of Bordeaux}\\
\affaddr{351, Cours de la Libération}\\
\affaddr{Talence, France}\\
\email{robert.strandh@u-bordeaux1.fr}}

\toappear{Permission to make digital or hard copies of all or part of
  this work for personal or classroom use is granted without fee
  provided that copies are not made or distributed for profit or
  commercial advantage and that copies bear this notice and the full
  citation on the first page. Copyrights for components of this work
  owned by others than the author(s) must be honored. Abstracting with
  credit is permitted. To copy otherwise, or republish, to post on
  servers or to redistribute to lists, requires prior specific
  permission and/or a fee. Request permissions from
  Permissions@acm.org.

  ILC '14, August 14 - 17 2014, Montreal, QC, Canada
  Copyright is held by the owner/author(s). Publication rights licensed to ACM.
  ACM 978-1-4503-2931-6/14/08\$15.00.
  http://dx.doi.org/10.1145/2635648.2635656}

\maketitle

\begin{abstract}
The fact that \clos{} is defined as a \clos{} program introduces two
categories of issues that must be addressed, namely
\emph{bootstrapping issues} and \emph{metastability issues}
\cite{Kiczales:1991:AMP:574212}.  Of the two, the latter is the more
difficult one, and also the one that has the most negative impact on
the \emph{elegance} of the code in that it requires base cases to be
handled specially.

We describe \emph{satiation}, a technique by which metastability
issues can be turned into bootstrapping issues, thereby simplifying
them and keeping the code elegant.  Satiation consists of pre-loading
the call history of a generic function with respect to a set of
argument classes so that the base cases are handled without invoking
the full protocol for computing the effective methods at runtime.
\end{abstract}

\category{D.3.4}{Programming Languages}{Processors}
[Code generation, Run-time environments]

\terms{Algorithms, Languages}

\keywords{\clos{}, \cl{}, Bootstrapping, Metastability}

\inputtex{sec-introduction.tex}
\inputtex{sec-previous.tex}
\inputtex{sec-our-method.tex}
\inputtex{sec-conclusions.tex}

\bibliographystyle{abbrv}
\bibliography{satiation}
\end{document}
