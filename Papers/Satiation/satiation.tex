\documentclass{acm_proc_article-sp}
\usepackage[utf8]{inputenc}

\def\inputfig#1{\input #1}
\def\inputtex#1{\input #1}
\def\inputal#1{\input #1}
\def\inputcode#1{\input #1}

\inputtex{logos.tex}
\inputtex{refmacros.tex}
\inputtex{other-macros.tex}

\begin{document}
\title{Resolving metastability issues at bootstrapping}
\numberofauthors{1}
\author{\alignauthor
Robert Strandh\\
\affaddr{University of Bordeaux}\\
\affaddr{351, Cours de la Libération}\\
\affaddr{Talence, France}\\
\email{robert.strandh@u-bordeaux1.fr}}

\maketitle

\begin{abstract}
The fact that \clos{} is defined as a \clos{} program introduces two
categories of issues that must be addressed, namely
\emph{bootstrapping issues} and \emph{metastability issues}
\cite{Kiczales:1991:AMP:574212}.  Of the two, the latter is the most
difficult one, and also the one that has the most negative impact on
the \emph{elegance} of the code, in that it requires base cases to be
handled specially.

We describe \emph{satiation}, a technique by which metastability
issues can be turned into bootstrapping issues, thereby simplifying
them and keeping the code elegant.  Satiation consists of pre-loading
the call history of a generic function with respect to a set of
argument classes so that the base cases are handled without invoking
the full protocol for computing the effective methods at runtime.
\end{abstract}

\category{D.3.4}{Programming Languages}{Processors}
[Code generation, Run-time environments]

\inputtex{sec-introduction.tex}
\inputtex{sec-previous.tex}
\inputtex{sec-our-method.tex}
\inputtex{sec-conclusions.tex}

\bibliographystyle{abbrv}
\bibliography{satiation}
\end{document}
