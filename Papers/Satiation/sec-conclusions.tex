\section{Conclusions and future work}

We have described a technique that allows us to avoid metastability
issues in the implementation of \clos{} in our system \sicl{} by
replacing those issues by simpler bootstrapping issues.  Furthermore,
our technique also simplifies bootstrapping by avoiding special cases
due to the non-existence of generic functions when the system is
bootstrapped, by using the \emph{host} generic function machinery
in early stages of bootstrapping. 

Currently, nothing prevents a specified method on a specified generic
function, specializing on specified classes to be modified or removed,
and nothing prevents a specified class from being redefined.  Should
this happen, ``the game would of course be over.''  We imagine a
mechanism that protects the user from inadvertently invoking such
operations.  It should probably be possible to \emph{toggle} the
mechanism so that system code can make modifications know to be safe.

When this article is written, \sicl{} is not yet finished, nor even in
a state to be executed standalone.  However, most of the difficult
components (such as the compiler, the garbage collector, and of course
\clos{}) are in a fairly advanced stage of development.  The final
verdict on the technique for bootstrapping the system can not be
determined until the system is able to run standalone.

%%  LocalWords:  metastability
