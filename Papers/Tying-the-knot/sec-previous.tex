\section{Previous work}

Portable Common Loops (PCL) seems to be the most common implementation
of the \closs{} (\clos{}) used by different \commonlisp{}
implementations, and it includes a full implementation of the \clos{}
Metaobject protocol, as specified in the book ``The Art of the
Metaobject Protocol'' \cite{Kiczales:1991:AMP:574212} (AMOP).  In
appendix C of that book, the authors explain how the class hierarchy
of \clos{} is bootstrapped.  In particular, they explain that, since
the class \texttt{standard-class} must exist before any other
metaobject can be created using the function
\texttt{allocate-instance}, that class must be created using some
means other than using the normal ones.  In that appendix, the authors
also identify several special cases for various functions, so as to
avoid infinite computations.

It is clear from the appendix that PCL was designed to be bolted onto
an existing pre-\clos{} \commonlisp{} system with minimal required
modifications.

In our article on bootstrapping
\cite{Durand-Strandh:2019:ELS:Bootstrapping}, we explain how
bootstrapping is done in the \sicl{} system.  Our technique is
completely different from the one used by PCL.  These days, we have
several highly conforming \commonlisp{} implementations that also
include full implementations of the metaobject protocol.  \sicl{}
bootstrapping relies on such implementations by having code in the
metaobject protocol be executed by a \emph{host} implementation.

The result of our technique is that the code of the \sicl{}
implementation of \clos{} looks more ``natural'', and that special
cases are largely possible to avoid.  Perhaps the most spectacular
example of how the \sicl{} \clos{} code differs in this respect from
that of other \commonlisp{} implementations is the definition of the
class named \texttt{t}, i.e., the root of the entire \clos{} class
hierarchy:

\begin{verbatim}
    (defclass t ()
      ()
    (:metaclass built-in-class))
\end{verbatim}

Several other previously published techniques provide the basis for
the technique described in this paper.

Our paper on generic dispatch \cite{Strandh:2014:FGD:2635648.2635654}
explains how the \sicl{} implementation of
\texttt{compute-discriminating-function} works, and in particular the
role of the \emph{call history} in this technique.

In our paper on ``satiation'' \cite{Strandh:2014:RMI:2635648.2635656}
we design a technique for filling the call history of a generic
function so as to avoid the kind of metastability issues that are
presented in appendix C of the AMOP book.

Finally, a necessity for the multiple versions of the classes and the
generic functions that are created as part of the \sicl{}
bootstrapping procedure, is the existence of first-class global
environments, explained in \cite{Strandh:2015:ELS:Environments}.
