\section{Main features of the \sicl{} system}
\label{sec-sicl-features}

\sicl{} is a system that is written entirely in \commonlisp{}.  Thanks
to the particular bootstrapping technique
\cite{durand_irene_2019_2634314} that we developed for \sicl{}, most
parts of the system can use the entire language for their
implementation.  We thus avoid having to keep track of what particular
subset of the language is allowed for the implementation of each
module.

We have multiple objectives for the \sicl{} system, including
excellent maintainability and good performance.  However, the most
important purpose in the context of this paper is \emph{support for
  excellent debugging tools}.  We think it is going to be difficult to
adapt existing \commonlisp{} implementations to support the kind of
application debugging that we consider essential for good programmer
productivity.

Another main purpose of the \sicl{} system is \emph{safety}.  There
are many situations described in the \commonlisp{} standard that have
undefined or unspecified behavior, such as:

\begin{enumerate}
\item Many times when a standard function is called with some argument
  that is not of the type indicated by the corresponding dictionary
  entry, the behavior is undefined, allowing the implementation to
  avoid potentially costly tests for exceptional situations.
\item When a non-local transfer is attempted to an exit point that has
  been ``abandoned'', the standard does not require this situation to
  be detected, making it possible for the system to crash or (worse)
  give the wrong result.
\item When some entity is declared \texttt{dynamic-extent}, but some
  necessary condition for this declaration is violated, the
  implementation is again not required to detect the problem, again
  potentially resulting in a crash or an incorrect computation.
\end{enumerate}

Fortunately, most potential situations of this type are not taken
advantage of by the implementation in order to improve performance,
but some are.  We think that the spirit of the \commonlisp{} standard
is to have a safe language, and that many of these situations of
undefined or unspecified behavior exist only to avoid significantly
more work for the system maintainers at the time the standard was
established.

For that reason, in the \sicl{} system, we do not intend to take
advantage of these situations to make the system unsafe for the
purpose of better performance, even though we might have to work
somewhat harder in order to maintain good performance in all
situations.
