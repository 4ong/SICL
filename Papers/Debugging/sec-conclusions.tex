\section{Conclusions and future work}

We believe that a decent development environment for \commonlisp{}
must include a very feature-full debugger for application programs, and
we firmly believe that the best way of accomplishing such an
environment is to have the debugger execute in the same process as the
application, but to have the application react to debugging operations
only when executed under the control of the debugger.

While it may seem like a valid objection that the application
programmer, as a result of detecting a defect, must rerun the
application from the debugger in order to benefit from the features
proposed by our technique, we disagree with this objection.  We simply
think that there is usually no valid reason to run the application
outside the debugger during the development phase.  The only exception
we can think of would be applications with extreme performance
requirements, or applications where response time is part of the
specification.

The validity of the technique described in this paper has been
somewhat verified, in that we have an embryonic implementation of an
interactive debugger, and an implementation of a small subset of the
communication protocol between the application and the debugger.
However, our current environment does not allow us to verify the
ultimate performance of application code run under the control of the
debugger.  We firmly believe that the performance loss for code
without any breakpoints in it will be acceptable, and that the
additional cost when breakpoints are involved is to be expected by the
application programmer.

We have not yet implemented the two-version body idea
\seesec{sec-two-body-versions} in \sicl{},
mainly because the only implementation of \sicl{} that currently works
is the interpreter for intermediate code.  This interpreter was
written for bootstrapping purposes only and performance is not an
issue.  For that reason, we have included only the debugging version
of function bodies.

In addition to producing a native version of the \sicl{} system, we
also need additional work on the \clordane{} debugger, and on the
additional components of the communication protocol between the
debugger and the application.

We think it would be desirable for existing \commonlisp{}
implementations to incorporate the technique described in this paper,
so as to allow for a much more complete development environment for
users.  However, we are convinced that the modifications that would be
required to those implementations would be prohibitive in terms of
time invested by maintainers.  For that reason, we unfortunately do
not hold out much hope for this possibility, and we intend to
concentrate our efforts on making our technique work for the \sicl{}
system.
