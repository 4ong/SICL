\section{Previous work}

With systems like \unix{}, debugging is usually performed as an
interaction between two \emph{processes}.  The debugger runs in one
process and the application in another process.  For a breakpoint, the
code of the application is modified by the debugger so that the
application sends a signal to the debugger when the breakpoint has
been reached.  For this purpose, the debugger maps the code pages of
the application as \emph{copy on write} (or COW), so that instances of
the same application that are not executed under the control of the
debugger are not affected by the modified code.  In particular, with
this technique any application can be debugged, including the
debuggger itself.

More stuff here, in particular about FLOSS commonlisp{} systems, about
commercial \commonlisp{} systems, about Genera, and perhaps about
other interactive systems like Scheme, etc.
