\section{Our technique}
\label{sec-our-technique}

\subsection{Two versions of every function body}
\label{sec-two-body-versions}

We use an additional machine register as a \emph{flag} to determine
whether a function is being debugged or not, i.e. whether it is run
under the control of the debugger or not.  While using an entire
register for this purpose might seem wasteful, this register is used
for our debugging flag only for a very short time, namely across a
function call.  Therefore, any register that is otherwise used as a
scratch register may be used for this purpose.

At the beginning of the execution of each function, a test is
performed to determine the value of the flag.  Based on the result of
this test, a branch is made to one of two \emph{similar} (but not
identical) copies of the entire function body%
\footnote{This idea was suggested by Michael Raskin.}.  As soon as
this choice
has been made, the register is available for normal use by the
register allocator.  When the flag is cleared, the branch is made to
the \emph{ordinary} version of the function body.  When the flag is
set, the branch is made to the \emph{debugging} version of the
function body.

By including both versions in the same function, we make it
unnecessary for the application programmer to recompile the code with
higher debug settings when it is desirable to have more debugging
information than what the compiler would generate by default.

The ordinary function body is compiled using every typical optimization
technique used by a good compiler, including:

\begin{itemize}
\item constant folding,
\item dead code elimination,
\item common sub-expression elimination,
\item loop-invariant code motion,
\item induction-variable optimization, 
\item elimination of in-scope dead variables.
\end{itemize}

Some of these optimization techniques are essential for
high-performance code, but many of them can make it significantly
harder for the application programmer to understand what the
program is doing:

\begin{itemize}
\item Common sub-expression elimination and similar techniques for
  redundancy elimination may make it impossible to set
  a breakpoint in some part of the code, simply because that code has
  been eliminated by the compiler.
\item When a variable is used for the last time, the compiler
  typically reuses the place that it occupies for other purposes, even
  though the variable may still be in scope.  This optimization makes
  it impossible for the application programmer to examine the value of
  a variable that has been eliminated.  An application programmer with
  a poor understanding of compiler-optimization techniques will find
  the result surprising.
\item Loop-invariant code motion results in code being moved from
  inside a loop to outside it.  Any attempt by the application
  programmer to set a breakpoint in such code will fail.
\item Induction-variable optimization will eliminate or replace
  variables in source code by others that are more beneficial for
  the performance of the computation, again making it harder for the
  application programmer to debug the code.
\end{itemize}

To avoid many of these inconveniences to the application programmer,
the debugging version of the function body is compiled in a way that
makes the code somewhat slower, but much more friendly for the purpose
of debugging.  Some of the optimization techniques cited above will
not be performed at all, or only in a less ``aggressive'' form.

Furthermore, the debugging version of the code is compiled so that a
small routine is called immediately before and immediately after the
execution corresponding to the evaluation of a form in the source
code.  This routine performs a short query from a modest size table;
we think with a few hundred entries at most.  This table is maintained
by the debugger, and made available in the reified thread object.  The
debugging version of the function starts by accessing this table and
storing it in a lexical variable, subject to register allocation as
usual.

The table is a bit-vector indexed by the value of the program counter,
modulo the size of the table.  If the corresponding bit is cleared,
there is definitely not a breakpoint at the value of the program
counter in question.  If the bit is set, it might be the case that
there is such a breakpoint.  In most cases, the bit will be cleared,
so the cost of the test is low, even in a thread that is being
debugged.  If the bit is set, another query is performed, this time to
a hash table, mapping values of the program counter to information
about break points.  If it turns out that the value of the program
counter has a breakpoint associated with it, the thread informs the
debugger.  Details of the communication between the application thread
and the debugger are discussed in
\refSec{sec-debugger-application-communication}.

In the normal version of the function body, when a function call is
made, the debug flag is cleared, indicating to the callee that it is
not under the control of the debugger.  In the debugging version of
the function body, on the other hand, when a function call is
made, the debug flag is set.  This mechanism thus automatically
propagates the information about debugging throughout the call chain.

\subsection{Communication between the debugger and the application}
\label{sec-debugger-application-communication}

Debuggers in \unix{} systems have full access to the address space of
the application, including the stack and the lexical variables.  The
debugger can therefore modify any lexical variable and then continue
the execution of the application.  Such manipulations may very well
violate some of the assumptions made by the compiler for a particular
code fragment.  For example, if the code contains a test for the value
of a numeric variable, the compiler may make different assumptions
about this value in the two different branches executed as a result of
the test.

Allowing the debugger to make arbitrary modifications to lexical
variables, let alone to any memory location, in a \commonlisp{}
application program will defeat any attempts at making the system
safe, and safety is one of the purposes of the \sicl{} system as
expressed in \refSec{sec-sicl-features}.  We must therefore come up
with a different communication protocol that keeps the system safe.

Our design contains two essential elements for this purpose:

\begin{enumerate}
\item The debugger consists of an interactive application with a
  \emph{command loop}.  An iteration of this command loop can of
  course be prompted by a user interaction.  However, when the
  application detects a breakpoint by querying the tables described in
  \refSec{sec-two-body-versions}, it injects a command into the
  command loop of the debugger, triggering the execution of code to
  handle the breakpoint.
\item The debugger sends commands to the application on a \emph{queue}
  protected by a \emph{semaphore}.  Once the application has informed
  the debugger about a breakpoint, it \emph{waits} on the semaphore to
  receive instructions from the debugger about what to do next.
\end{enumerate}

The debugger is in charge of taking into account the commands issued
by the application programmer.  When the application programmer
indicates that a certain action should be performed at a particular
place in the source code, the debugger populates the two tables
mentioned in \refSec{sec-two-body-versions}, and records the
particular action the application programmer desires, for example:

\begin{itemize}
\item Stop the execution of the application and take further action
  based on the state of the application data.
\item Print a \emph{trace message} and immediately continue the
  execution.
\item Step the execution of the program in one of several different
  ways (in, out, over, finish).
\item Continue normal expectation of the application program.
\end{itemize}

To allow for the application programmer to examine the state of the
application, when the application thread detects a breakpoint, the
command it injects into the debugger command loop contains a complete
list of live local variables and their values%
\footnote{Clearly, care must be taken so as to avoid the possibility
  of the debugger making arbitrary modifications to structured data,
  and thereby possibly violating the assumptions made by the compiler
  when the application code was compiled.  We have yet to determine
  exactly how to implement such restrictions.}.

\subsection{Debugger commands available to the application programmer}

We have an embryonic implementation of an interactive debugger, called
\clordane{}.  We used the \mcclim{} library for writing graphic user
interfaces as a basis for this tool.  Currently, \clordane{} can show
the source code of an application (one source file at a time) and an
indication of the place of a breakpoint.  The application being
debugged is then compiled with the \sicl{} compiler, generating
high-level intermediate representation (HIR).  The HIR code is then
interpreted using a small program running in a host \commonlisp{}
implementation.

The communication protocol described in
\refSec{sec-debugger-application-communication} has been implemented
and works to our satisfaction, but only a small subset of interactions
have been implemented so far.

We think that the following commands must be implemented in a fully
featured debugger:

\begin{itemize}
\item The application programmer should be able to point to a location
  in the source code to indicate a particular action to be taken at
  that point:
  \begin{itemize}
  \item Stop the execution of the application and wait for further
    actions.
  \item Print a trace message, possibly containing the values of live
    variables, and then continue the execution. 
  \end{itemize}
  It should be possible to make the action conditional, based on some
  arbitrary expression to be evaluated in the debugger thread, and
  which can contain references to live variables in the application.
\item When the application is stopped, the application programmer
  should be able to examine live variables, and (in some cases, with
  restrictions) modify their values.
\item Also, when the application is stopped, the application
  programmer should be able to issue one of several types of
  \emph{stepping} commands, implicitly indicating the next location
  for the application to stop.
\end{itemize}
