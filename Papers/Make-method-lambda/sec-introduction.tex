\section{Introduction}

In the definition of the \commonlisp{} \cite{ansi:common:lisp}
metaobject protocol \cite{Kiczales:1991:AMP:574212}, the generic
function \texttt{make-method-lambda} plays a role that is very
different from most of the other generic functions that are part of
the metaobject protocol.

According to the book, the function has four parameters, all required:

\begin{enumerate}
\item A generic function metaobject.
\item A (possibly uninitialized) method metaobject.
\item A lambda expression.
\item An environment object.
\end{enumerate}

The main difference is that \texttt{make-method-lambda} is called as
part of the expansion code for the \texttt{defmethod} macro, whereas
other generic functions are called at execution time.

The AMOP book states that the generic function passed as the first
argument may be different from the one that the method is ultimately
going to be added to.  This possibility seems to exist to handle the
situation where a \texttt{defgeneric} form is followed by a
\texttt{defmethod} form in the same file.  In this situation, the
\commonlisp{} standard clearly states that the file compiler does not
create the generic function at compile time.  Therefore, when the
corresponding \texttt{defmethod} form is expanded (and therefore
\texttt{make-method-lambda} is called), the generic function does not
yet exist.  It will be created only when the compiled file is loaded
into the \commonlisp{} system.

The book also states that the method object passed as second argument
may be uninitialized, suggesting that the \emph{class prototype} of
the method class to be instantiated may be passed as the second
argument.

The third argument is a lambda expression corresponding to the body of
the \texttt{defmethod} form.  The purpose of
\texttt{make-method-lambda} is to wrap this lambda expression in
another lambda expression that defines the \emph{method function}.  The
default behavior of \texttt{make-method-lambda} is to return a lambda
expression with two parameters.  The first parameter is a list of all
the arguments to the generic function.  The second parameter is a list
of next methods that can be invoked using \texttt{call-next-method}
from the body of the method.  Therefore \texttt{make-method-lambda}
also provides definitions of \texttt{call-next-method} and
\texttt{next-method-p} that are lexically inside the lambda expression
it returns.

Finally, the fourth argument to \texttt{make-method-lambda} is an
environment object.
