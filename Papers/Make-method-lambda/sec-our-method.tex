\section{Our technique}

As permitted by the \commonlisp{} standard, the \texttt{defgeneric}
macro may store information provided in the \texttt{defgeneric} form
so as to make better error reporting possible when subsequent forms
are compiled.  In particular, the standard mentions storing
information about the lambda list given, so that subsequent calls to
the generic function can be checked for correct argument count.
This information is kept in an implementation-specific format that
does not contain the full generic-function metaobject, as this object
is created when the compiled code resulting from the file compilation
is loaded.

However, just as it is possible to keep information about the lambda
list at compile time, it is also possible to keep information about
the \texttt{:method-class} option given, or, when no option was
supplied, the fact that the method class is \texttt{standard-method}.

With this additional information, during the expansion of the
\texttt{defmethod} macro, the name of the method class can be
retrieved, then a class metaobject from the name, and finally a class
prototype from the class metaobject. 

While the first parameter of \mml{} is indicated as a generic-function
metaobject, it is not specifically indicated that this object might be
uninitialized, contrary to the method object that must be passed as
the second argument.  It is, however, indicated that the
generic-function object passed as the first argument may not be the
generic-function object to which the new method will eventually be
added.  Therefore, there is not much information that \mml{} can make
use of.  The exception would be the exact class of the generic
function and the exact method class.  It would be awkward for a method
on \mml{} to access this information explicitly, rather than as
specializers of its parameters.  For that reason, the first argument
to \mml{} might as well be a class prototype, just as the second
argument might be.
