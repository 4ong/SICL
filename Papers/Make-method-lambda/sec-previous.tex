\section{Previous work}

In their article \cite{Constanza:2008}, Costanza and Herzeel give a
simple example of this simple \texttt{defmethod} form:

\begin{verbatim}
(defmethod foo ((x integer) (y integer))
  (do-something x y))
\end{verbatim}

\noindent
and at the end of section 2.1, on page 3, they claim that the
expansion of that form is ``something like'' the follow form:

{\small\begin{verbatim}
(let ((gf (ensure-generic-function 'foo)))
  (multiple-value-bind
      (lambda-expression extra-initargs)
      (make-method-lambda
        gf
        (class-prototype
          (generic-function-method-class gf))
        '(lambda (x y) (do-something x y))
        lexical-environment-of-defmethod-form)
    (add-method
      gf
      (apply #'make-instance
             (generic-function-method-class gf)
             :qualifiers '()
             :lambda-list '(x y)
             :specializers (list (find-class 'integer)
                                 (find-class 'integer))
             :function (compile nil lambda-expression)
             extra-initargs))
\end{verbatim}}

\noindent
except that we have formatted the code to fit the page, and we have
added two missing closing parentheses at the end of the form.

However, this expansion is not possible.  It has two fundamental
problems:

\begin{enumerate}
\item The call to \texttt{make-method-lambda} must be made at
  macro-expansion time, whereas in their example, the call is present
  in the expansion, so it will be made at run time.
\item In their example, the resulting lambda expression, i.e. the
  original lambda expression of the body of the \texttt{defmethod}
  form wrapped in the code added by \texttt{make-method-lambda}, is
  compiled in the null lexical environment.  However, compiling in the
  null lexical environment would violate the semantics of the
  \commonlisp{} standard, which requires that the body of the
  \texttt{defmethod} form be compiled in lexical environment in which
  it appears.
\end{enumerate}
