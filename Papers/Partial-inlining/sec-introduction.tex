\section{Introduction}

Inlining represents an important optimization technique in any modern
compiler.  It avoids the overhead of a full function call, and it
allows further optimization in the calling function, such as type
inference.

However, inlining does not have only advantages.  It increases the
size of the code, with a possible negative impact on processor cache
performance.  It also increases pressure on register allocation,
possibly making it necessary to spill registers to the stack more
often.

Some sources distinguish between \emph{procedure integration} and
\emph{inline expansion} \cite{Muchnick:1998:ACD:286076}.  Both
techniques are often referred to with the abbreviated form
\emph{inlining}.  Our use of the term then corresponds to
\emph{procedure integration}.

Most literature sources define inlining as something similar to
``replacing a call to a function with a copy of the body of the called
function''.
