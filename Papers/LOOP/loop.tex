\documentclass{sig-alternate-05-2015}
\usepackage[utf8]{inputenc}

\def\inputfig#1{\input #1}
\def\inputtex#1{\input #1}
\def\inputal#1{\input #1}
\def\inputcode#1{\input #1}

\inputtex{logos.tex}
\inputtex{refmacros.tex}
\inputtex{other-macros.tex}

\begin{document}
\setcopyright{rightsretained}
\title{A modern implementation of the LOOP macro}
\numberofauthors{1}
\author{\alignauthor
Robert Strandh\\
\affaddr{University of Bordeaux}\\
\affaddr{351, Cours de la Libération}\\
\affaddr{Talence, France}\\
\email{robert.strandh@u-bordeaux1.fr}}


\maketitle

\begin{abstract}
We describe a modern implementation of the \commonlisp{} \texttt{loop}
macro.  This implementation is part of the \sicl{} project.

We use \emph{combinator parsing} to recognize \texttt{loop} clauses,
and we use \clos{} for code generation. 
\end{abstract}

\begin{CCSXML}
  <ccs2012>
  <concept>
  <concept_id>10011007.10011006.10011008.10011024.10011027</concept_id>
  <concept_desc>Software and its engineering~Control structures</concept_desc>
  <concept_significance>500</concept_significance>
  </concept>
  </ccs2012>
\end{CCSXML}

\ccsdesc[500]{Software and its engineering~Control structures}

\printccsdesc

\keywords{\clos{}, \commonlisp{}, Iteration}

\inputtex{sec-introduction.tex}
\inputtex{sec-previous.tex}
\inputtex{sec-our-method.tex}
\inputtex{sec-benefits.tex}
\inputtex{sec-conclusions.tex}
\inputtex{sec-acknowledgments.tex}

\bibliographystyle{abbrv}
\bibliography{loop}
\end{document}
