\section{Previous work}

\subsection{MIT LOOP with variations}
\label{sec-mit-loop}

One of the first implementations of the \commonlisp{} \texttt{loop}
macro is the one that is often referred to as ``MIT LOOP''
\cite{Burke:Moon:MIT.loop}.  A popular variation of this
implementation includes modifications by \symbolics{}.

This implementation of the \texttt{loop} macro is sometimes more
permissive than the \commonlisp{} standard.  For example, the standard
requires all \emph{variable clauses} to precede all \emph{main
  clauses}.  Code such as the one in this example:

\begin{verbatim}
    (loop until (> i 20)
          for i from 0
          do (print i))
\end{verbatim}

\noindent
is thus not conforming according to the standard, since \texttt{until}
is a \emph{main clause} whereas \texttt{for} is a \emph{variable
  clause}.

However, MIT LOOP and its variation accepts the code in the example.

Another example of non-conforming behavior is illustrated by the
following code:

\begin{verbatim}
    (loop for i from 0 below 10
          sum i
          finally (print i))
\end{verbatim}

The \commonlisp{} standard clearly states that the loop variable does
not take on the value of the upper limit, here $10$, so the value
printed in the \texttt{finally} clause should be $9$.  However,
\texttt{loop} implementations derived from MIT LOOP print $10$
instead.

The MIT LOOP implementation is \emph{monolithic} and that holds true
for its variations too.  The code is contained in a single file with
around $2000$ lines of code in it.

Code generation uses a significant number of special variables holding
various pieces of information that are ultimately assembled into the
final expansion of the macro.

\subsection{\ecl{} and \clasp{}}

\ecl{}%
\footnote{\ecl{} stands for ``Embedded Common Lisp.\\
See: //https://gitlab.com/embeddable-common-lisp/ecl}
includes two implementations of the \texttt{loop} macro, namely
the initial MIT LOOP with only minor modifications, and the variation
by \symbolics{} also with minor modifications.

\clasp{}%
\footnote{See: https://github.com/drmeister/clasp}
is a recent implementation of \commonlisp{}.  It is derived
from \ecl{} in that the \clanguage{} code of \ecl{} has been
translated to \cplusplus{} whereas most of the \commonlisp{} code has
been included with no modification, including the code for the
\texttt{loop} macro.

\ecl{} \texttt{loop} being derived from MIT LOOP, the unconforming
example shown in \refSec{sec-mit-loop} is also accepted by \ecl{} and
\clasp{}.

\subsection{\sbcl{}}

\sbcl{}%
\footnote{\sbcl{} stands for Steel-Bank \commonlisp{}.\\
See: http://www.sbcl.org/}
includes an implementation of the \texttt{loop} macro that was
originally derived from MIT LOOP, but that also includes code from the
\texttt{loop} macro of the \genera{} operating system.  Furthermore,
the \sbcl{} implementation of the \texttt{loop} macro has been
modified and improved to allow for user-definable extensions, such as
the extension defined in \cite{Rhodes:2007:USC:1622123.1622138} for
iterating over user-definable sequences.

\sbcl{} \texttt{loop} being derived from MIT LOOP, the unconforming
example shown in \refSec{sec-mit-loop} is also accepted by \sbcl{}.

\subsection{\clisp{}}

\clisp{} has its own implementation of the \texttt{loop} macro.  The
bulk of the implementation can be found in a function named
\texttt{expand-loop}.  This function consists of more than $900$ lines
of code.

\subsection{\ccl{}}

Like many other implementations, \ccl{}%
\footnote{\ccl{} stands for Clozure \commonlisp.\\
See: http://ccl.clozure.com/}
includes the variation of MIT
LOOP containing modifications by \symbolics{}.
