\section{Conclusions and future work}

\subsection{Second clause parser}

As mentioned in \refSec{sec-our-technique}, we are able to signal
appropriate conditions in some cases when the initial attempt is made
to parse the body of the \texttt{loop} form as individual clauses.
However, when a syntax error is detected in some clause, all further
analysis is abandoned.  It would clearly be better if the analysis
could continue with the remaining clauses, and if an appropriate error
condition could be signaled for the faulty clause.

A simple way of improving error reporting would be to add more parsers
for each clause type.  This additional parsers would recognize
incorrect clause syntax and ultimately result in an error being
signaled, but more importantly, they would succeed so that parsing
could continue with subsequent clauses.

Unfortunately, however, while the parsing technique we use has many
advantages as described in refSec{sec-benefits}, it also has the main
disadvantage that parsing gets slower as more parsers need to be
tried, in particular if no care is taken to order the parsers with
respect to probability of success.

We plan to avoid this conundrum by implementing a \emph{second parser}
for parsing individual clauses.  This second parser would be invoked
only when the first one fails.  In that situation, we estimate that
performance is of secondary importance and that emphasis should be on
appropriate error signaling.
