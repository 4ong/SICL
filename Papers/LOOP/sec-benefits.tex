\section{Benefits of our method}
\label{sec-benefits}
 
As already mentioned in \refSec{sec-our-technique-parsing-clauses},
the main advantage of our technique is that it allows for a
\emph{modular} structure of the \texttt{loop} implementation.

The most immediate consequence of this improved modularity is that the
code is easier to maintain than a monolithic code for the same
purpose.  A modification in one module is less likely to break other
modules.

This modularity also makes it very simple for additional clause types
to be added by the \commonlisp{} implementation, such as the extension
for iterating over the user-extensible sequences described by Rhodes
in his paper on user-extensible sequences
\cite{Rhodes:2007:USC:1622123.1622138}.  This extension defines the
new \texttt{loop} keywords \texttt{element} and \texttt{elements} for
this purpose.

Furthermore, since the parsing technology we use does not require any
costly pre-processing, extensions could be added by client code on a
per-module basis, rather than as a permanent extension.  Then, client
code can maintain the capacity of detecting non-conforming constructs
in most code, while allowing for selected extensions in specific
modules.
