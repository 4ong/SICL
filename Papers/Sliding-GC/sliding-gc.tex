\documentclass{acm_proc_article-sp}
\usepackage[utf8]{inputenc}
\usepackage{color}

\def\inputfig#1{\input #1}
\def\inputtex#1{\input #1}
\def\inputal#1{\input #1}
\def\inputcode#1{\input #1}

\inputtex{logos.tex}
\inputtex{refmacros.tex}
\inputtex{other-macros.tex}

\begin{document}
\title{An improvement to sliding garbage collection}
\numberofauthors{1}
\author{\alignauthor
Robert Strandh\\
\affaddr{University of Bordeaux}\\
\affaddr{351, Cours de la Libération}\\
\affaddr{Talence, France}\\
\email{robert.strandh@u-bordeaux1.fr}}

\maketitle

\begin{abstract}
Garbage collection algorithms are divided into two main categories,
namely \emph{mark-and-sweep} and \emph{copying} collectors.  Among the
copying collectors, the treadmill\fixme{cite work} is perhaps the most
commonly used.  However, the frequently overlooked \emph{sliding}
collector has some advantages in that it preserves the \emph{relative
  age} of objects.  The main problem with the sliding collector is
that building address-forwarding tables is costly.  We suggest an
improvement to the existing algorithm that reverses the order between
building the forwarding table and moving the objects.  Our method
improves performance of building the table, making the sliding
collector a better contestant for young generations of objects
(nurseries). 
\end{abstract}

\category{H.4}{Information Systems Applications}{Miscellaneous}
%A category including the fourth, optional field follows...
\category{D.2.8}{Software Engineering}{Metrics}[complexity measures, performance measures]

\inputtex{sec-introduction.tex}
\inputtex{sec-previous.tex}
\inputtex{sec-our-method.tex}
\inputtex{sec-conclusions.tex}

\bibliographystyle{abbrv}
\bibliography{sliding-gc}
\end{document}
