\documentclass{acm_proc_article-sp}
\usepackage[utf8]{inputenc}
\usepackage{color}

\def\inputfig#1{\input #1}
\def\inputtex#1{\input #1}
\def\inputal#1{\input #1}
\def\inputcode#1{\input #1}

\inputtex{logos.tex}
\inputtex{refmacros.tex}
\inputtex{other-macros.tex}

\begin{document}
\title{An Improvement to Sliding Garbage Collection}
\numberofauthors{1}
\author{\alignauthor
Robert Strandh\\
\affaddr{University of Bordeaux}\\
\affaddr{351, Cours de la Libération}\\
\affaddr{Talence, France}\\
\email{robert.strandh@u-bordeaux1.fr}}

\toappear{Permission to make digital or hard copies of all or part of
  this work for personal or classroom use is granted without fee
  provided that copies are not made or distributed for profit or
  commercial advantage and that copies bear this notice and the full
  citation on the first page. Copyrights for components of this work
  owned by others than the author(s) must be honored. Abstracting with
  credit is permitted. To copy otherwise, or republish, to post on
  servers or to redistribute to lists, requires prior specific
  permission and/or a fee. Request permissions from
  Permissions@acm.org.

ILC '14, August 14 - 17 2014, Montreal, QC, Canada
Copyright is held by the owner/author(s). Publication rights licensed to ACM.
ACM 978-1-4503-2931-6/14/08\$15.00.
http://dx.doi.org/10.1145/2635648.2635655}

\maketitle

\begin{abstract}
Garbage collection algorithms are divided into three main categories,
namely \emph{mark-and-sweep}, \emph{mark-and-compact}, and
\emph{copying} collectors.  The collectors in the
\emph{mark-and-compact} category are frequently overlooked, perhaps
because they have traditionally been associated with greater cost than
collectors in the other categories.  Among the compacting collectors,
the \emph{sliding} collector has some advantages in that it preserves
the \emph{relative age} of objects.  The main problem with the
traditional sliding collector by Haddon and Waite \cite{Haddon:1967}
is that building address-forwarding tables is costly.  We suggest an
improvement to the existing algorithm that reverses the order between
building the forwarding table and moving the objects.  Our method
improves performance of building the table, making the sliding
collector a better contestant for young generations of objects
(nurseries).
\end{abstract}

\category{D.3.4}{Programming Languages}{Processors}[Memory management
(garbage collection)]

\terms{Algorithms, Performance}

\keywords{Compaction, Sliding garbage collection}

\inputtex{sec-introduction.tex}
\inputtex{sec-previous.tex}
\inputtex{sec-our-method.tex}
\inputtex{sec-performance.tex}
\inputtex{sec-conclusions.tex}

\bibliographystyle{abbrv}
\bibliography{sliding-gc}
\end{document}
