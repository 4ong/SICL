\section{Our method}

Our method uses a \emph{break table} just like the method by Haddon
and Waite, but instead of building, moving, and sorting the table
while objects are moved, it first moves the objects and then
constructs the table.  For that, additional space in the form of a bit
vector is required.  The bit vector has one bit per word of memory in
the heap, which amounts to less than $2$\% additional memory on a
$64$-bit machine. 

\begin{figure}
\begin{center}
\inputfig{fig-example-a.pdf_t}
\end{center}
\caption{\label{fig-example-a}
Example of initial heap.}
\end{figure}

\refFig{fig-example-a} shows a heap in which shaded areas indicate
live objects and white areas indicate dead objects.  The heap contains
16 word as shown by the addresses. 

