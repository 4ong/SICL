\section{Conclusions and future work}

We have described a technique for type inference of intermediate code
resulting from the translation of \commonlisp{} code to an
intermediate representation that manipulates only objects of the
source language.  Our technique uses a straightforward dataflow
algorithm, which is also the most natural way of thinking about the
evolution of type information in a \commonlisp{} program.

Type inference can be much more effective if the intermediate code is
first converted to SSA form \cite{Cytron:1989:EMC:75277.75280,
  Cytron:1991:ECS:115372.115320} or some other notation that preserves
values of lexical variables across assignments.  Assigning to a
lexical variable may lose existing type information associated with
that variable, so that the type of the new value must be tested again
at some later point in the program.  Our technique works whether the
intermediate code is transformed this way or not, but we have not yet
tested the effect of such transformations on the effectiveness of our
technique.

At present, the type inference only operates within function, and does
not use any information about other functions that isn't explicitly
declared. The algorithm can be extended to incorporate type information
from local functions without much change, but using inferred types of
global functions will require more work due to the semantics of
redefinition.

Determining precise dataflow in a \commonlisp{} program is complicated
when a lexical variable is shared between nested closures, and
arbitrary dataflow may invalidate essential type information, thereby
limiting the effectiveness of type inference.  Currently, Cleavir only
contains a very rudimentary \emph{escape analysis} phase, forcing our
type-inference technique to be very conservative by treating all shared
variables as being of unknown type.  We intend to improve the quality
of the escape analysis so as to improve the effectiveness of our
type-inference technique.

Currently, the dataflow computation is implemented with an ad hoc
approach.  However, Kildall \cite{Kildall:1973:UAG:512927.512945}
designed a general approach for solving a large spectrum of problems,
and our type-inference technique is a good candidate for using that
general approach.  We already have a general implementation of this
approach, and it can be customized by providing a \emph{domain} in the
form of a standard instance%
\footnote{Recall that a \emph{standard instance} is an instance of
  (a subclass of) the class named \texttt{standard-class}.  Some
  authors use the term ``CLOS class'', but that term does not exist in
  the \commonlisp{} standard.}
that provides the details of the required lattice operations.  In
order to improve maintainability, we intend to adapt our
type-inference technique so that it uses this implementation of
Kildall's approach.

Finally, the type lattice used by our technique is currently very
coarse.  A more populated lattice can greatly improve the
effectiveness of our technique, though care must be taken to avoid
excessive processor time in the type inferencer itself.
