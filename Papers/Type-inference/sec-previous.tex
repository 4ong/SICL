\section{Previous Work}

\subsection{The Nimble type inferencer}

Henry Baker describes the Nimble type inferencer
\footnote{Unpublished technical report.  See for instance
  http://www.pipeline.com/~hbaker1/TInference.html.}  His technique
works for the pre-standard \commonlisp{} language, and works by
annotating source code with type information in the form of
\emph{type declarations}.

Nimble tracks control flow in both directions and maintains both
upper and lower bounds on inferred type information, and could take exponential time. This was compensated for by Baker's team by
reducing type computations to specialized linear bit-vector
arithmetic, but the algorithm was still noticeably slow (on the
computers of the time). A comparison of efficiency with our technique
has not been made.

The advantage of Baker's technique is that it can be used with any
conforming \commonlisp{} implementation, except that it would need some
minor work in order to be applicable to code in the standardized
language.

\subsection{\sbcl{}}

The \sbcl{} implementation of \commonlisp{} is known to have excellent
type-inference capabilities. Its ``constraint propagation'' algorithm
is similar to that described here, but instead operates at control-flow
level of basic blocks, and can propagate constraints other than types.
Rather than inlining, \sbcl{} relies on type derivation functions
associated with higher-level built-in operators. These are necessarily
more complex, but removing the necessity of inlining may decrease
code size.
