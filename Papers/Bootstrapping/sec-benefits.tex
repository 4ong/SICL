\section{Benefits of our technique}
\label{sec-benefits}

Appendix C of ``The Art of the Metaobject Protocol''
\cite{Kiczales:1991:AMP:574212} (Living with Circularity) cites a
number of ways in which their system handles circularity and avoids
bootstrapping and metastability issues.

\subsection{Bootstrapping benefits}

The first bootstrapping problem that is mentioned is the fact that
\texttt{standard-class} must exist before it can be created.  Their
solution is to create this class using some special-case mechanism.
Our technique uses the version of \texttt{standard-class} in the
preceding environment, so this problem is avoided altogether.  As a
result, we can freely modify the definition of
\texttt{standard-class} and rerun the bootstrapping procedure.  No
special case has to be considered.

The second bootstrapping problem mentioned is that generic functions
are used for method lookup, but these generic functions can not exist
until a significant part of the protocol has been implemented.  As an
example, take the call to \texttt{ensure-class} made as a result of
executing the expansion of a \texttt{defclass} form.  By having
\texttt{ensure-class} check for the special case when the argument is
\texttt{standard-class} and by supplying a special function for
creating instances of \texttt{standard-class} they avoid bootstrapping
issues, simply because during bootstrapping, all classes created will
be instances of \texttt{standard-class}.  They also supply a special
version of \texttt{finalize-inheritance} that checks for the metaclass
\texttt{standard-class} and calls special-purpose code in this case.
With our technique, no such special case is needed.  All
classes that are instantiated are fully operational in the preceding
environment, as is the \texttt{finalize-inheritance} generic function.

\subsection{Metastability benefits}

The first example of a metastability problem mentioned in the book is
that \texttt{slot-value} calls \texttt{slot-value-using-class} which
then calls \texttt{slot-location} which in turn recursively calls
\texttt{slot-value} on the class metaobject to access the slot
metaobjects of the class.  The authors propose to solve this problem
by arranging for the function \texttt{slot-location} to check for the
special argument
\texttt{effective-slots} and return a predefined
location.  Our technique does not need this kind of special case,
because the function \texttt{class-slots} does not call
\texttt{slot-value} at all.  It accesses the \texttt{effective-slots}
slot directly, using its location.  This location has been compiled in
during the creation of the effective method and discriminating
function for \texttt{class-slots}.

The final issue discussed in the book arises because the function
\texttt{compute-discriminating-function} is also a generic function
that can not be called with itself as an argument when a method has
been added or removed from it.  Again they solve the issue by a
special case whereby a test is made to see whether the argument is a
standard generic function (i.e. an instance of
\texttt{standard-generic-function}) and if so, a special version of\\
\texttt{compute-discriminating-function} which is not a generic
function is called instead.  With our technique, every generic
function, \texttt{compute-discriminating-function} included, has a 
\emph{call cache} that includes an effective method that is able to
handle arguments that are direct instances of
\texttt{standard-generic-function}.  That call cache entry is not
invalidated when
\texttt{compute-discriminating-function} has new
methods added to it, at least not when the
methods added respect the restrictions of the metaobject protocol,
i.e. that user code is not allowed to add methods that are applicable
when given only standard objects as arguments.

\subsection{Other benefits}

In addition to solving the bootstrapping issues and the metastability
issues given in the ``The Art of the Metaobject Protocol'' book, our
technique has several additional benefits.

Since we begin the bootstrapping procedure by
defining the classes and generic functions specified by the metaobject
protocol, we are able to use the \clos{} machinery to define system
classes.  In a system where \clos{} is added late, many system classes
must be defined using some other mechanism.

Furthermore, as already mentioned, our technique has great advantages
to maintenance.  There are no dependencies between \clos{} code and
other code that require duplication of information that must be kept
synchronized when some code is modified.
