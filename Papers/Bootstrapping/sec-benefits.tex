\section{Benefits of our technique}
\label{sec-benefits}

Appendix C of ``The Art of the Metaobject Protocol''
\cite{Kiczales:1991:AMP:574212} (Living with Circularity) cites a
number of ways in which their system handles circularity and avoids
bootstrapping and metastability issues.

\subsubsection{Bootstrapping benefits}

The first bootstrapping problem that is mentioned is the fact that
\texttt{standard-class} must exist before it can be created.  Their
solution is to create this class using some special-case mechanism.
Our technique uses the version of \texttt{standard-class} in the
preceding environment, so this problem is avoided altogether.  As a
result, we can freely modify the definition of
\texttt{standard-class} and rerun the bootstrapping procedure.  No
special case has to be considered.

The second bootstrapping problem mentioned is that generic functions
are used for method lookup, but these generic functions can not exist
until a significant part of the protocol has been implemented.  As an
example, the take \texttt{ensure-class} that is called as a result of
a \texttt{defclass} form.  By having \texttt{ensure-class} check for
the special case when the argument is \texttt{standard-class} and by
supplying a special function for creating instances of
\texttt{standard-class} they avoid bootstrapping issues, simply
because during bootstrapping, all classes created will be instances of
\texttt{standard-class}.  Also, in order to resolve this issue, they
supply a special version of \texttt{finalize-inheritance} that checks
for the metaclass \texttt{standard-class} and calls special-purpose
code in this case.  Again, with our technique, no such special case is
needed.  All classes that are instantiated are fully operational in
the preceding environment, and the \texttt{finalize-inheritance}
generic function is operational in the same way.

\subsubsection{Metastability benefits}

One example shows how \texttt{slot-value} calls
\texttt{slot-value-using-class} which calls \texttt{slot-location} and
which calls \texttt{slot-value} on the class metaobject to access the
slot metaobjects of the class.  They propose a solution to this
problem whereby \texttt{slot-location} checks for the specal argument
\texttt{effective-slots} and returns a predefined location.  Our
technique does not need this kind of special case, because
\texttt{class-slots} does not call \texttt{slot-value} at all.  It
accesses the \texttt{effective-slots} slot directly, using its
location.  This location has been compiled in during the creation of
the effective method and discriminating function for
\texttt{class-slots}.

The final issue discussed in the book has to do with the fact that
\texttt{compute-discriminating} function is a generic function and
that it can not be called with itself as an argument when a method has
been added or removed from it.  Again they solve the issue by a
special case whereby a test is made to see whether the argument is an
instance of \texttt{standard-generic-function} and if so, a special
non-generic version of \texttt{compute-discriminating-function} is
called instead.  With our technique,
\texttt{compute-discriminating-function} has a \emph{call cache} that
includes an effective method that handles arguments that are instances
of \texttt{standard-generic-function} and that call cache entry is not
invalidated when new methods are added to
\texttt{compute-discriminating-function}, at least not when the
methods added respect the restrictions of the metaobject protocol,
i.e. that user code is not allowed to add methods that are applicable
when given only standard objects as arguments.

\subsubsection{Other benefits}

In addition to solving the bootstrapping issues and the metastability
issues given in the ``The Art of the Metaobject Protocol'' book, our
technique has several additional benefits.

For one thing, since we begin the bootstrapping procedure by
defining the classes and generic functions specified by the metaobject
protocol, we are able to use the \clos{} machinery to define system
classes.  In a system where \clos{} is added late, many system classes
must be defined using some other mechanism.

Furthermore, as already mentioned, our technique has great advantages
to maintenance.  There are no dependencies between \clos{} code and
other code that require duplication of information that must be kept
synchronized when some code is modified.
