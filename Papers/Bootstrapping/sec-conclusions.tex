\section{Conclusions and future work}

We have described a technique for bootstrapping a \commonlisp{} system
using an existing conforming \commonlisp{} system that is also
supported by the library \texttt{closer-mop}.  To our knowledge, no
existing \commonlisp{} system is bootstrapped this way.

There are several advantages to our technique:

\begin{itemize}
\item The full \commonlisp{} language can be used in order to
  implement the system, including the compiler, thereby making the
  code more maintainable.
\item By bootstrapping the MOP generic functions and the hierarchy of
  classes first, we eliminate the bootstrapping problems and
  metastability problems cited by the AMOP book
  \cite{Kiczales:1991:AMP:574212}.
\item Also, by bootstrapping the MOP machinery first, we take advantage
  of it by using it to define all the standard system classes, thereby
  eliminating the need for special mechanisms for this purpose.
\item The absence of special mechanisms that are needed in existing
  implementations for defining many aspects of the system itself,
  further contributes to the maintainability of our code.
\end{itemize}

Even though the technique outlined in this paper is known to work,
many more aspects of the system need further work, including the
bootstrapping technique itself, in order for a native executable to be
generated:

\begin{itemize}
\item We must supply a (simple) code generator that translates
  intermediate code to native code.  The amount of work required is
  fairly modest, and mainly consists of creating native code for
  memory operations such as \texttt{car} and
  \texttt{standard-instance-access}, for object allocation, and for
  simple arithmetic on fixnums.
\item Interface code to the operating system must be supplied, in
  particular for input/output operations.
\item We have yet to write the code that translates the host
  representation of the object graph into a native representation.
  Special care must be taken for object types that are imported from
  the host during bootstrapping, such as symbols, numbers, and
  \texttt{cons} cells.
\end{itemize}

However, we are in no hurry to create a native executable system.  The
moment we do, we lose a fairly good environment (namely the host
\commonlisp{} system) for debugging our code.  Instead, we plan to use
the host environment for testing as many aspects of \sicl{} as
possible, and for creating support for better debugging capabilities,
and only later create a native executable.

In terms of future work, there are still several optimization
techniques that need to be implemented for the Cleavir compiler
framework.
