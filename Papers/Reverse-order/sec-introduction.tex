\section{Introduction}

The \commonlisp{} \emph{sequence} functions are defined to work on
lists as well as vectors.  Furthermore, many of these sequence
functions accept a keyword argument \texttt{from-end} that alters the
behavior in that elements toward the end of the sequence are favored
over elements toward the beginning of the sequence.

Most sequence functions are not required to process the elements from
the end of the sequence.  For example, it is allowed for \texttt{find}
to compare elements from the beginning of the sequence and return the
\emph{last} element that \emph{satisfies the test}%
\footnote{The phrase \emph{satisfy the test} has a precise meaning in
  the \commonlisp{} standard as shown in section 17.2 in that
  document.}
even if the test has side effects.  There is one exception this
requirement, however:  The \texttt{count} function is required by the
standard to test the elements from the end of the sequence.  Though if
the test has no side effects and cannot fail, as is the case of
functions such as \texttt{eq} or \texttt{eql}, testing from the
beginning is arguably conforming behavior.

Processing list elements from the beginning to the end could, however,
have a significant additional cost associated with it when processing
from the end would require fewer executions of the test function, and
the additional cost increases with the complexity of the test.

The premise of this paper is that the programmer who supplies a true
value to the keyword argument \texttt{:from-end} has an inkling that
it would be more efficient to process the elements from the end, even
when the standard explicitly allows the implementation to violate that
processing order.  At the very least, a programmer that supplies a
true value would have to take into account that at least \emph{some}
implementation may \emph{actually} process the elements from the end.
Therefore, it is likely that the programmer who definitely does not
want elements to be processed form the end, either does not supply a
true value, or would write a special-case version of the sequence
function in question.

%%  LocalWords:  startup runtime
