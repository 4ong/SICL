\section{Previous work}

We will frequently refer to techniques used by \sbcl{} because of its
reputation as a high-performance implementation.  We will however also
use other high-performance implementations for comparison when we have
information on the techniques used by those implementations, or when
we can reasonably guess these techniques from other evidence.

For its implementation of \texttt{find}, \sbcl{} takes advantage of
the freedom given by the standard, by processing elements from the
beginning, and remembering the last element that satisfies the test.
For implementations where the technique is unknown, it suffices to
write a test function that counts the number of times it is invoked
and run it on a list where only the last element satisfies the test.

For its implementation of \texttt{count}, \sbcl{} uses the simple
technique of reversing the list first and then processing the elements
of the reversed list from the beginning.

To test the stack depth of various implementations, we devised the
following test:

{\small\begin{verbatim}
(defun stack-depth (n)
  (declare (optimize (space 3)))
  (if (zerop n) 0 (1+ (stack-depth (1- n)))))
\end{verbatim}}

The following table shows approximate values of the argument to
\texttt{stack-depth} for which some implementations exhaust the stack:

\begin{tabular}{|l|r|}
\hline
Implementation & argument\\
\hline
\hline
\sbcl{} & 90000\\
\hline
Allegro & ???\\
\hline
LispWork & ???\\
\hline
\end{tabular}

As we already mentioned, we use recursion as the basis of our
technique, because it is quite fast.  We devised the following test to
verify this hypothesis:

{\small\begin{verbatim}
(defun recursive-find (x list)
  (declare (optimize (speed 3) (debug 0) (safety 0)))
  (cond ((null list) nil)
        ((eq x (car list)) x)
        (t (recursive-find x (cdr list)))))
\end{verbatim}}

On \sbcl{} executing this function on a list with $40000$ elements
where the last element is the only one satisfying the test is slightly
faster than an explicit loop from the beginning of the list, and
around $4$ times as fast as calling \texttt{find}.%
\FIXME{There has got to be a way to convince \sbcl{} \texttt{find} to
  run faster on this test.}
This result indicates that we should use recursion whenever the size
of the stack allows it, though there is of course no portable way of
testing how much stack space is available.  However, each
implementation might have a specific way, which would then be good
enough.
