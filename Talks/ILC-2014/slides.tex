\documentclass{beamer}

\mode<presentation>
{
  \usetheme{Warsaw}
}

\usepackage[utf8]{inputenc}
\usepackage{color}
\usepackage{epsfig}
\usepackage{alltt}
\usepackage{moreverb}

\newcommand{\red}[1]{\textcolor{red}{#1}}
\newcommand{\tr}[1]{\texttt{\red{#1}}}

\def\bs{$\backslash$}
\def\inputfig#1{\input #1}
\def\inputtex#1{\input #1}

\title{SICL spinoffs}
\author{Robert Strandh\inst{1}}

\institute{\inst{1}Université de Bordeaux}
\date[ILC 2014]{International Lisp Conference 2014}


\begin{document}

\begin{frame}
  \titlepage
\end{frame}
%-----------------------------------------------------------
\begin{frame}
\frametitle{Three papers}

\begin{itemize}
\item Fast generic dispatch for Common Lisp
\item An improvement to sliding garbage collection
\item Resolving metastabilty during bootstrapping
\end{itemize}

\end{frame}
%-----------------------------------------------------------
\begin{frame}
\frametitle{Contents of talk}

\begin{itemize}
\item lskjfd
\item alkjsdf 
\end{itemize}
\end{frame}
%-----------------------------------------------------------
\begin{frame}
\frametitle{SICL}

SICL does not stand for anything.  Pronounce it ``sickle''.

Reason for SICL:

\begin{itemize}
\item Review design decisions of existing implementations.
\item Use more of CLOS.
\item Make modules implementation independent if possible.
\item Improve algorithms of existing implementations. 
\item Prepare for internationalization. 
\end{itemize}
\end{frame}
%-----------------------------------------------------------
\end{document}
