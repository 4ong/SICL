\documentclass{beamer}

\mode<presentation>
{
  \usetheme{Warsaw}
}

\usepackage[utf8]{inputenc}
\usepackage{color}
\usepackage{epsfig}
\usepackage{alltt}
\usepackage{moreverb}

\newcommand{\red}[1]{\textcolor{red}{#1}}
\newcommand{\tr}[1]{\texttt{\red{#1}}}

\def\bs{$\backslash$}
\def\inputfig#1{\input #1}
\def\inputtex#1{\input #1}

\title{SICL spinoffs}
\author{Robert Strandh\inst{1}}

\institute{\inst{1}Université de Bordeaux}
\date[ILC 2014]{International Lisp Conference 2014}


\begin{document}

\begin{frame}
  \titlepage
\end{frame}
%-----------------------------------------------------------
\begin{frame}
  \frametitle{Three papers}

  \begin{itemize}
  \item Fast generic dispatch for Common Lisp
  \item An improvement to sliding garbage collection
  \item Resolving metastability during bootstrapping
  \end{itemize}
\end{frame}
%-----------------------------------------------------------
\begin{frame}
  \frametitle{Contents of talk}

  \begin{itemize}
  \item Reasons for SICL.
  \item Fast generic dispatch.
  \item Sliding garbage collection.
  \item Resolving metastability.
  \item Future of SICL.
  \end{itemize}
\end{frame}
%-----------------------------------------------------------
\begin{frame}
  \frametitle{SICL}

  SICL does not stand for anything.  Pronounce it ``sickle''.

  Reason for SICL:
  
  \begin{itemize}
  \item Review design decisions of existing implementations.
  \item Use more of CLOS.
  \item Make modules implementation independent if possible.
  \item Improve algorithms of existing implementations. 
  \item Prepare for internationalization. 
  \end{itemize}
\end{frame}
%-----------------------------------------------------------
\begin{frame}
  \frametitle{SICL}
  \framesubtitle{Examples of design decisions revisited}

  \begin{itemize}
  \item Implementation of \texttt{:from-end t} on list sequences.
  \item In-place \emph{merge sort} algorithm. 
  \item Representation of heap-allocated objects.
  \item Garbage collection.
  \item CLOS generic dispatch.
  \item CLOS bootstrapping.
  \end{itemize}
\end{frame}
%-----------------------------------------------------------
\begin{frame}
  \frametitle{Fast generic dispatch}

  Most of the literature discusses table-based methods, which are not
  very well adapted to modern architectures.  Compressing the tables
  increases the number of memory accesses even more.
  \vskip 0.5cm
  PCL uses a simple hash table.
  \vskip 0.5cm
  Zendra et all use a technique similar to ours, but only for static
  languages.   

\end{frame}
%-----------------------------------------------------------
\begin{frame}
  \frametitle{Sliding garbage collection} 
  \framesubtitle{Original idea}

  Original algorithm by Haddon and Waite in 1967.
  \vskip 0.5cm
  Context: existing mark-and-sweep GC.
  \vskip 0.5cm
  Basic idea: construct a \emph{break table} that maps intervals of
  addresses to \emph{deltas} to be added to addresses in order to
  obtain new location. 

\end{frame}
%-----------------------------------------------------------
\begin{frame}
  \frametitle{Sliding garbage collection} 
  \framesubtitle{Example of break table}
  \begin{center}
\inputfig{fig-example-aa.pdf_t}
  \end{center}

  \begin{center}
\inputfig{fig-example-da.pdf_t}
  \end{center}

\end{frame}
%-----------------------------------------------------------
\begin{frame}
  \frametitle{Sliding garbage collection} 
  \framesubtitle{Bitmap for marking}
  \begin{center}
\inputfig{fig-example-a.pdf_t}
  \end{center}
\end{frame}
%-----------------------------------------------------------
%-----------------------------------------------------------
\end{document}

%%  LocalWords:  metastability
