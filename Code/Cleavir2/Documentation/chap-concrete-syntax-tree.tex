\chapter{Concrete Syntax Tree}

Most \commonlisp{} compilers assume that the input has been obtain by
an application of the \commonlisp{} \texttt{read} function to a
sequence of characters, which then returns a nest list structure that
serves as the input to the first pass of the compiler.

The first pass of \sysname{}, on the other hand, takes a
\emph{concrete syntax tree} (or \emph{CST} for short) as its input.  

A CST has the exact same structure as the corresponding tree
representing the source expression, except that some of the nodes in
the tree have been augmented with information on source location.

To obtain a CST from a sequence of characters, the most appropriate
technique is to use the external library
Eclector%
\footnote{https://github.com/s-expressionists/Eclector}
which does exactly that.

The main advantage of a CST over the nested list structure used by
most compilers is that the CST can have \emph{source information}
associated with it.  Eclector allows for client code to customize this
source information so as to adapt it to what the client needs.

While the CST has a structure that is similar to that of the
corresponding nested list structure that it represents, it can not be
manipulated with the ordinary \commonlisp{} functions \texttt{cons},
\texttt{car}, and \texttt{cdr}.  Instead, a separate library named
Concrete Syntax Tree%
\footnote{https://github.com/s-expressionists/Concrete-Syntax-Tree}
contains functions for manipulating CSTs that parallel those
\commonlisp{} functions.
