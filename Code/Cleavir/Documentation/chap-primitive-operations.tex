\chapter{Primitive operations}

\section{Purpose}

Primitive operations (or primops for short) are similar to
\commonlisp{} special operators in that the compiler handles them
specially.  They are different from \commonlisp{} special operators in
that a primop does not necessarily have an evaluation rule that is
different from that of a function call.

Generally speaking, primops should not be used directly in application
code.  Instead, they are used in system code for implementing certain
basic \commonlisp{} function.  So, for example, the \texttt{consp}
primop would typically be used only in the code for the \commonlisp{}
function \texttt{consp}.  That function would then be inlined by the
compiler, so that the resulting AST and ultimately the resulting HIR
instruction that the \texttt{consp} primop translates to will be
present also in application code.

Frequently, the need for a primop comes from arises because some HIR
instruction is needed.  Take, for example, the HIR instruction named
\texttt{eq-instruction} for comparing two pointer values for equality.
The existence of that instruction requires an AST doing the same
thing, and it is called \texttt{eq-ast}.  When compiled to HIR, the
\texttt{eq-ast} generates the an \texttt{eq-instruction}.  Finally, in
order to produce the source for the \commonlisp{} function
\texttt{eq}, it must be possible to produce the \texttt{eq-ast} from
some \commonlisp{} code, which is why the \texttt{eq} primop is
needed.

\section{Package}

All primitive operations have names that are symbols in the package
named \texttt{cleavir-primop}.  This package contains only those
symbols that name the primitive operations.  There is no code directly
associated with the primitive operations.  Instead, the code for
translating primitive operations to abstract syntax trees
\seechap{chap-abstract-syntax-tree} is in the form of methods on the
function \texttt{cleavir-generate-ast:convert-special}.

\section{Existing primitive operations}

\def\defprimop #1 {\dodocv {#1} {Primitive operation}}

\defprimop{eq}

This primitive operation has the same semantics as the \commonlisp{}
function \texttt{eq}.  It should not be used directly in application
code.  Instead, its main purpose is for defining the code for that
\commonlisp{} function.  Typically the \commonlisp{} \texttt{eq}
function will be declared \texttt{inline} so that the abstract syntax
tree and HIR instruction resulting from this primop will eventually
end up in the compiled code of many applications.
