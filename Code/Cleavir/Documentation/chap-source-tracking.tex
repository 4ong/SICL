\chapter{Source tracking}

Source tracking is necessarily implementation-specific.  In this
chapter, we describe a protocol for source tracking that allows each
implementation to configure the protocol according to its methods and
abilities.

The function \texttt{cleavir-ast:generate-ast} takes an additional
optional argument named \texttt{source-info}.  The caller passes an
object that encapsulates source information about the form to be
compiled.

\sysname{} calls two generic functions:

\Defgeneric {begin-source} {source-info expression}

This function is called at the beginning of the processing of
\textit{expression}.  It should return an object representing the
source location for \textit{expression}, or \texttt{nil} if the source
location of \textit{expression} is not known.

\Defgeneric {end-source} {source-info}

This function is called at the end of the processing of the expression
that was most passed to the most recent call to
\texttt{begin-source}.

These functions are called in \emph{last-in first-out} order,
effectively defining a \emph{stack} of nested expressions.  The
implementation can keep track of that stack in order to have a better
idea of the source location of the expression, or it can ignore this
information and just use a hash table, mapping expressions to
locations.
