\chapter{Source tracking}

Source tracking is necessarily implementation-specific.  In this
chapter, we describe a protocol for source tracking that allows each
implementation to configure the protocol according to its methods and
abilities. 

At the heart of the protocol is the generic function \texttt{location}:

\Defgeneric {location} 
            {environment expression parent-location child-number}

Within the dynamic context of compiling a file, \sysname{} calls this
generic function to obtain the source location of some expression that
it is about to convert to an abstract syntax tree.  The function
returns some implementation-defined object that stores the
source-tracking information about an expression.  

The parameter \textit{environment} is the global environment object
that the implementation passed to the top-level compilation function. 
The parameter \textit{expression} is the expression that \sysname{} is
about to process and convert into an AST.  The parameter
\textit{parent-location} is the implementation-defined location object
that was returned by the call to \texttt{location} with the parent
expression of \textit{expression}, or \texttt{nil} if
\textit{expression} is a top-level expression with no parent.  The
parameter \textit{child-number} is the element number of
\textit{expression} in its parent, where elements are numbered from
$0$. 

At any point, the implementation-defined method on \texttt{location}
may fail to determine the location of \textit{expression} perhaps
because there are multiple or no occurrences of the expression in the
source, or perhaps because some reader macro created a complex
structure that can not be analyzed by this method.  In that case, the
implementation-defined method simply returns an object indicating this
fact.  It can be any object other than \texttt{nil}, as long as the
same method recognizes this object when it is passed as the
\textit{parent-location} argument to subsequent calls to
\texttt{location}. 
