\chapter{Abstract syntax tree}
\label{chap-abstract-syntax-tree}

\section{\texttt{constant-ast}}

\begin{tabular}{|l|l|l|}
\hline
Initarg & Reader & Description\\
\hline\hline
\texttt{:value} & \texttt{value} & The corresponding constant.\\
\hline
\end{tabular}

A \texttt{constant-ast} is produced whenever the compiler sees a
constant, and processing is done by \texttt{eval} or \texttt{compile}
(as opposed to by \texttt{compile-file}).  This constant can either be
in the form of a self-evaluating object, a constant variable, or a
quoted object.

At this point, there is no discrimination between the different types
of constant, nor between integers of different magnitude.  Later on
in the compilation process, some backend may replace small integer
constants by immediate values. 

\section{\texttt{global-ast}}

\begin{tabular}{|l|l|l|}
\hline
Initarg & Reader & Description\\
\hline\hline
\texttt{:name} & \texttt{name} & The name of the function.\\  
\hline
\texttt{:storage} & \texttt{storage} & The storage cell for the function.\\
\hline
\end{tabular}

A \texttt{global-ast} is produced whenever the compiler sees a
reference to a \emph{global function}, i.e., a name in a functional
position that is not defined as a local function, as a special
operator, nor as a macro.

\section{\texttt{special-ast}}

\begin{tabular}{|l|l|l|}
\hline
Initarg & Reader & Description\\
\hline\hline
\texttt{:name} & \texttt{name} & The name of the variable.\\  
\hline
\texttt{:storage} & \texttt{storage} & The storage cell for the variable.\\
\hline
\end{tabular}

A \texttt{special-ast} is produced whenever the compiler sees a
reference to a \emph{special variable}, i.e. a name in a variable
position that is not defined as a lexical variable, a constant
variable, nor as a symbol macro.

\section{\texttt{lexical-ast}}

\begin{tabular}{|l|l|l|}
\hline
Initarg & Reader & Description\\
\hline\hline
\texttt{:name} & \texttt{name} & The name of the variable.\\  
\hline
\end{tabular}

A \texttt{lexical-ast} is produced whenever the compiler sees a
reference to a \emph{lexical variable}. 

\section{\texttt{call-ast}}

\begin{tabular}{|l|l|l|}
\hline
Initarg & Reader & Description\\
\hline\hline
\texttt{:callee-ast} & \texttt{callee-ast} & The AST of the function
to be called\\  
\hline
\texttt{:argument-asts} & \texttt{argument-asts} & A list of ASTs, one
for each argument\\
& & to the function.\\
\hline
\end{tabular}

A \texttt{call-ast} is produced whenever the compiler sees a function
call.  It contains ASTs for the function to be called and for the
arguments to that function.  The AST representing the function can be
any AST that produces a function as a value.

\section{\texttt{block-ast}}

\begin{tabular}{|l|l|l|}
\hline
Initarg & Reader & Description\\
\hline\hline
\texttt{:body-asts} & \texttt{body-asts} & The AST of the body.\\
\hline
\end{tabular}

\section{\texttt{eval-when-ast}}
\section{\texttt{function-ast}}

\begin{tabular}{|l|l|l|}
\hline
Initarg & Reader & Description\\
\hline\hline
\texttt{:body-ast} & \texttt{body-ast} & The AST for the body of the function.\\
\hline
\end{tabular}


\section{\texttt{go-ast}}
\section{\texttt{if-ast}}

\section{\texttt{load-time-value-ast}}

A \texttt{load-time-value-ast} is produced whenever the compiler sees
a \texttt{load-time-value} special form, but also when it sees a
\emph{constant} and processing is done by \texttt{compile-file},
rather than by \texttt{compile} or \texttt{eval}.  

This AST always contains a \emph{form}, so that if it is the result of
a \emph{constant}, then it is wrapped in a \texttt{quote} special
form. 

\section{\texttt{progn-ast}}

\begin{tabular}{|l|l|l|}
\hline
Initarg & Reader & Description\\
\hline\hline
\texttt{:form-asts} & \texttt{form-asts} & A list of ASTs of the body.\\
\hline
\end{tabular}

\section{\texttt{return-from-ast}}

\begin{tabular}{|l|l|l|}
\hline
Initarg & Reader & Description\\
\hline\hline
\texttt{:block-ast} & \texttt{block-ast} & The AST of the
corresponding block.\\
\hline
\texttt{:form-ast} & \texttt{form-ast} & The AST of the value form.\\
\hline
\end{tabular}


\section{\texttt{setq-ast}}

\begin{tabular}{|l|l|l|}
\hline
Initarg & Reader & Description\\
\hline\hline
\texttt{:lhs-ast} & \texttt{lhs-ast} & The AST of the left-hand side.\\
\hline
\texttt{:value-ast} & \texttt{value-ast} & The AST of the value form.\\
\hline
\end{tabular}


\section{\texttt{tagbody-ast}}

\begin{tabular}{|l|l|l|}
\hline
Initarg & Reader & Description\\
\hline\hline
\texttt{:items} & \texttt{items} & A list of ASTs corresponding to\\
& & the items of the \texttt{tagbody} form. \\
\hline
\end{tabular}

An item can either be a \texttt{tag-ast} or an AST corresponding to
some \texttt{statement}. 

\section{\texttt{tag-ast}}

\begin{tabular}{|l|l|l|}
\hline
Initarg & Reader & Description\\
\hline\hline
\texttt{:name} & \texttt{name} & The name of the tag.\\
\hline
\end{tabular}

\section{\texttt{the-ast}}
